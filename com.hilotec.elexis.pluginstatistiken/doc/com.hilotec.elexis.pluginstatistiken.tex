%/*******************************************************************************
% * Copyright (c) 2009, A. Kaufmann and Elexis
% * All rights reserved. This program and the accompanying materials
% * are made available under the terms of the Eclipse Public License v1.0
% * which accompanies this distribution, and is available at
% * http://www.eclipse.org/legal/epl-v10.html
% *
% * Contributors:
% *    A. Kaufmann - initial implementation 
% *    
% * $Id: MesswertBase.java 5386 2009-06-23 11:34:17Z rgw_ch $
% *******************************************************************************/

\documentclass[a4paper]{scrartcl}
\usepackage{german}
\usepackage[utf8]{inputenc}
\usepackage[german]{babel}
\usepackage[]{hyperref}
\usepackage{listings}
\usepackage{booktabs}
\usepackage[pdftex]{graphicx}

\lstset{extendedchars=false}

\title{Hilotec-Pluginstatistiken}
\author{Antoine Kaufmann}

\begin{document}
\maketitle
\section{Einführung}
Dieses Plugin dient dazu statistische Auswertungen und Exporte mit
Patientendaten durchzuführen. Dazu wird die IDataAccess-Schnittstelle benutzt,
die von diversen Elexis-Plugins benutzt wird, wie beispielsweise vom
Befunde- oder vom Hilotec-Messwerte-Plugin, wofür dieses Plugin primär
entwickelt wurde. Dabei wird auf das Statistikplugin «Archie» aufgesetzt,
welches die Möglichkeit bietet Diagramme zu generieren, und Daten als CSV zu
exportieren. Welche Daten benutzt werden, wird mit einer XML-Datei beschrieben.
Darin können mehrere Datenquellen mit Namen eingerichtet werden. In der
XML-Konfigurationsdatei kann die Auswahl der Daten nach beliebigen Kriterien
eingeschränkt werden.

\section{Voraussetzungen}
\begin{itemize}
    \item Elexis 2.0
    \item Archie trunk-Version
\end{itemize}

\section{Konfiguration}
Wie oben bereits erwähnt, geschieht die Konfiguration der Datenquellen(Abfragen
oder Englisch «query») über eine XML-Konfigurationsdatei. Diese befindet sich im
Be\-nutz\-er\-da\-ten\-ver\-zei\-chnis:
\begin{description}
    \item[Windows:] \texttt{C:\textbackslash
                                Dokumente und Einstellungen\textbackslash
                                Benutzer\textbackslash Elexis\textbackslash
                                statistiken.xml}
    \item[Linux:] \texttt{/home/benutzer/elexis/statistik.xml}
    % TODO: Mac?
\end{description}

In dieser XML-Datei können die einzelnen Abfragen im Root-Element
\texttt{queries} mit \texttt{query}-Elementen definiert werden.
Zu jeder Abfrage muss eine Liste mit den ge\-wün\-schten Feldern angegeben
werden, und optional kann die Menge der Datensätzen mit beliebigen Bedingungen
eingeschränkt werden. Die XML-Konfiguration könnte bei\-spiels\-wei\-se
fol\-gen\-der\-ma\-ssen aussehen:
\begin{lstlisting}
<?xml version="1.0"?>
<queries>
  <query title="Aelter als 30, Bluthochdruck">
    <cols>
      <col name="alter" source="Patient.Alter" />
      <col name="name" source="Patient.Name" />
      <col name="vorname" source="Patient.Vorname" />
      <col name="bdvorher"
        source="Messwerte:Patient:firstsince=[startdatum]:\
        status.bddia" /> 
      <col name="bdnacher"
        source="Messwerte:Patient:lastbefore=[enddatum]:\
        status.bddia" />
    </cols>
    <where>
      <and>
        <greaterthan a="[alter]" b="30" />
        <greaterthan a="[bdvorher]" b="120" />
      </and>
    </where>
  </query>
</queries>
\end{lstlisting}
\textbf{Hinweis:} Bitte beachten Sie, dass die Zeilen mit
\texttt{\textbackslash} am Ende nur aus Darstellungsgründen umgebrochen wurden,
und in der Konfigurationsdatei unbedingt am Stück und ohne Leerzeichen stehen
müssen.

Auf den ersten Blick sieht diese Konfigurationsdatei womöglich etwas kompliziert
aus, da die IDataAccess-Identifier (also der Teil mit
\texttt{Messwerte:Patient...}) manchmal etwas lang werden. Die Idee, die
dahinter steckt, ist aber einfach. Für genauere In\-for\-ma\-tio\-nen zur
IDataAccess-Schnittstelle kann das Elexis-Handbuch und die Dokumentation des
jeweiligen Plugins zu Rate gezogen werden. Im Folgenden wird genauer auf die
Mög\-lich\-kei\-ten beim Erstellen von Abfragen
eingegangen.

\subsection{Spaltenliste}
Die Spaltenliste besteht nur aus \texttt{col}-Elementen mit zwei Attributen. Das
erste Attribut \texttt{name} ist glechzeitig Spaltenbeschriftung und interne
Bezeichung für die Spalte(zum Beispiel für die where-Bedingungen), während mit
dem zweiten Attribut \texttt{source} ein IDataAccess-Identifier angegeben werden
muss, der angibt, wo die Daten für die Spalte hergenommen werden sollen. Im
Gegensatz zu den IDataAccess-Platzhaltern, wie sie in Briefen und Dokumenten
angegeben werden können, dürfen sie hier nicht in eckigen Klammern
eingeschlossen werden.

Was aber in dem Source-Attribut in eckigen Klammern angegeben werden kann, sind
Abfrageparameter. Dies wird auch oben im Beispiell benutzt, um den Datumsbereich
einzuschränken(\texttt{[startdatum]} bei \texttt{firstsince} und
\texttt{[enddatum]} bei \texttt{lastbefore}\footnote{Für genauere Informationen
dazu, bitte die \texttt{hilotec-messwerte}-Dokumentation konsultieren}). Im
Moment stehen nur \texttt{startdatum} und \texttt{enddatum} als Parameter für
die Abfragen zur Verfügung. Die Werte dieser Parameter werden durch die beiden
Datumsfelder in der Archie-View gesetzt.


\subsection{Bedingungen}
Um die Menge der ausgegebenen Datensätze einzuschränken, können mit dem
\texttt{where}-Element Bedingungen angegeben werden, die angeben welche
Datensätze miteinbezogen werden sollen, und welche nicht. Dabei können Abfragen
aus den folgenden Vergleichs-Operatoren zusammengebaut werden:
\begin{description}
  \item[equal] Vergleicht Attribut \texttt{a} mit Atttribut \texttt{b} und ist
               wahr, wenn die beiden Werte übereinstimmen.
  \item[greaterthan] Vergleicht Attribut \texttt{a} mit Atttribut \texttt{b} und
                     ist wahr wenn \texttt{a} grösser ist als \texttt{b}.
  \item[lessthan] Vergleicht Attribut \texttt{a} mit Atttribut \texttt{b} und
                  ist wahr wenn \texttt{a} kleiner ist als \texttt{b}.
\end{description}
In den Attributen der Operatoren kann auf Werte in Spalten zugegriffen werden,
indem Platzhalter in der Form \texttt{[spaltenname]} benutzt werden.

Weiter stehen auch die üblichen Logischen Operatoren zur Verfügung:
\begin{description}
  \item[not] Negiert den Wahrheitswert des Unterelements, muss genau ein
             Unterelement haben.
  \item[and] Ist genau dann wahr, wenn alle Unterelemente wahr sind, oder keine
             Unterelemente vorhanden sind. Es können beliebig viele
             Unterelemente angegeben werden.
  \item[or] Ist wahr, wenn mindestens ein Unterelement wahr ist. Es können
            beliebig viele Unterelemente angegeben werden.
\end{description}
Damit können beliebig verschachtelte Bedingungen entworfen werden.


\section{Verwendung}
Die Verwendung des Plugins ist relativ selbsterklärend. Wenn es geladen ist,
erscheint in der Statistikliste von Archie eine weitere Statistik mit dem Titel
«Pluginstatistiken», diese muss ausgewählt werden. Sobald das geschehen ist,
erscheint im Parameterfenster von Archie eine Dropdown-Liste mit den vorhandenen
Abfragen. Dort muss die ge\-wünsch\-te Statistik ausgewählt werden, und in den
beiden Datumsfelder müssen ggf. noch die gewünschten Werte eingefüllt werden.
Sobald die Daten zusammengesucht wurde, ist alles weitere Sache von Archie. Hier
können die Daten dann beispielsweise mit Archies Exportfunktion als CSV
exportiert werden.

\end{document}
