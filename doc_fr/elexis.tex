% *******************************************************************************
% * Copyright (c) 2007-2009 by Elexis
% * All rights reserved. This document and the accompanying materials
% * are made available under the terms of the Eclipse Public License v1.0
% * which accompanies this distribution, and is available at
% * http://www.eclipse.org/legal/epl-v10.html
% *
% * Contributors:
% *    G. Weirich, U. Berner, D. Lutz
% *
% *  $Id: elexis.tex 5278 2009-05-07 10:37:57Z rgw_ch $
% *******************************************************************************
% !Mode:: "TeX:UTF-8" (encoding info for WinEdt)
%
% Dies ist das Zentraldokument für die Elexis-Dokumentation. Einzelne
% Kapitel und Unterkapitel können mit \include eingesetzt werden.

\documentclass[paper=a4,BCOR8.25mm,twoside]{scrbook}
\usepackage{german}
\usepackage[utf8]{inputenc}
\usepackage{makeidx}
\usepackage{wrapfig}
\usepackage{graphicx}
%\usepackage{placeins}
\makeindex

\usepackage{floatflt}
\usepackage[]{hyperref}
\usepackage{color}
\author{Gerry Weirich\\Urs Berner\\Daniel Lutz\\traduction française par Bruno Büchel}

\extratitle{
    \vfill
	\begin{center}
		\includegraphics{../ch.elexis/rsc/elexis-logo}
	\end{center}
    \begin{center}
        \textbf{Version 2.1.0}
    \end{center}
    \vfill
}
\title{Le Manuel de Elexis\textsuperscript{\textregistered}}
\begin{document}

\maketitle

\tableofcontents

\part{Introduction}
\chapter{Conditions pour l'utilisation}
Elexis\textsuperscript{\textregistered} est un projet ouvert qui évolue constamment. Le développement continue, et on ne peut pas garantir que ce manuel décrit toujours correctement et/ou de façon exhaustive toutes les qualités de ce programme. Si vous croyez d'avoir trouvé une erreur veuillez nous l'annoncer svp.

Important : Vous ne pouvez utiliser Elexis que si vous êtes d'accord avec la clause de non-résponsabilité suivante et avec les conditions de la licence.


\section{Clause de non-résponsabilité}
On ne peut jamais garantir qu'on logiciel soit sans erreurs\footnote{Déjà en 1930 Kurt Gödel avait prouvé avec son théorème d'incomplétude que il existe une infinité de faits vrais qu'il est impossible de prouver en utilisant la branche des mathématiques en question. L'élargissement de ce théorème par la preuve que l'absence d'erreurs d'un logiciel ne peut jamais être garantie à été fourni en 1936 par Alan Turing. Malgré les percées théoriques de connaissances informatique, que Turing a fourni avec sa machine de Turing hypothétique (des véritables ordinateurs n'existaient pas encore) il n'était par ailleurs pas le permier programmeur. Cette honneur appartient à Lady Ada Augusta Byron, Countess of Lovelace (1815-1852), qui avait développé le premier programme d'ordinateur (pour la analytical engine de Charles Babbage qui n'a jamais été construite)}. Il en est de même pour des projets Closed Source que pour des projets Open Source. Elexis est utilisé et testé dans plusieurs cabinets médicaux. Dans la mesure du 
possible toutes les erreurs ont été éliminées. Mais il ne peut pas être garanti qu'il n' y existent pas d'autres erreurs.
Pour cette raison nous déclinons toute responsabilité pour des dégâts directs ou indirects, matériels ou immatériels, découlant de l'utilisation ou la non-disponibilité de Elexis. Nous ne garantissons pas non plus la fonctionnalité du logiciel sur un ordinateur spécifique ou pour un usage particulier.


\section{Licence}
Vous pouvez utiliser Elexis sur autant d'ordinateurs que vous voulez. Vous pouvez transmettre des copies de Elexis à des tiers. Vous êtes libres d'appliquer toute modification possible à Elexis ou de les faire effectuer par des tiers, à l'exception de celles énumérées dans le paragraphe suivant.


\medskip

Il n'est  \textbf{pas permis }de : supprimer ou modifier des indications de Copyright du programme ou de la documentation. Il n'est  \textbf{pas permis} de créer un logiciel indépendant à base de Elexis sans indiquer l'origine de cette base. Il n'est \textbf{pas non plus permis} de nommer 'Elexis' un autre logiciel ou un logiciel qui a été crée par des modifications et extensions de Elexis.

\chapter{Description et instruction rapide}
Elexis -- die elektronische Praxis(le cabinet électronique) -- est
\begin{itemize}
	\item un projet de développement communautaire d'un logiciel source ouvert (Open Source)spécifiquement adapté aux conditions suisses.

	\item  un logiciel universel : Il est utile pour toutes le professions médicales qu'il s'agisse d'un cabinet médicale ou d'un cabinet de dentiste, d'un cabinet de physiothérapie, de ergothérapie ou de logopédie, de psychothérapie ou de homéopathie. Elexis peut être perfectionné et adapté avec souplesse aux besoins de l'utilisateur.
	\item De la pratique pour la pratique : Elexis est développé par des professionnels du domaine médicale et testé sous les conditions du travail quotidien au sein du cabinet médical.

	\item Fonctionne sous Windows, Linux et MacOS.X\footnote{Pour MacOS X par contre avec la restriction que le traitement de texte doit avoir lieu avec un autre programme , car l'intégration d'Open Office n'est pas encore possible.} et même dans des environnements mixtes avec différents systèmes informatiques sur le même serveur. 
	\item Open Source : Elexis n'est pas un programme propriétaire. Vous n'êtes donc pas dépendant d'un fabricant. Des personnes intéressées peuvent en tout temps avoir accès au code source et peuvent selon besoin augmenter la fonctionnalité du logiciel. La licence sous laquelle Elexis est distribué permet un tel développement par des tiers de façon explicite et sans conditions ou coûts.

	\item Tout en un : le dossier médical du patient, la gestion des stocks, les commandes, la facturation, le contrôle des débiteurs, l'agenda et bien plus encore.
  	\item Sociable : Peut importer les données d'autres logiciels (si prévu par leur producteurs) et peut exporter ses propres données dans des formats d'exportation standardisés.
\end{itemize}
\pagebreak[3]
\textbf{Soutien}
Elexis contient tous les composants afin de pouvoir démarrer directement. Un dossier médical électronique est toutefois une chose complexe et ne peut pas être maîtrisé d'un coup.
Nous essayons de vous faciliter le début par notre documentation. Si vous préférez ne pas traiter avec des internes des ordinateurs, nous vous proposons également du soutien sous forme d'installation et de maintenance. C'est avec plaisir que nous vous proposons une offre sur demande.


\bigskip

\textbf{Forum}
Il existe un forum ouvert (http://www.elexis-forum.ch) , où des questions
et des discussions entre les utilisateurs et les intéressés sont possible et bienvenus.

\bigskip

\textbf{Plugins supplémentaires et développement }
\index{Plugins supplémentaires}\index{Website}
Les nouveautés sont en général d'abord décrit sur le site (http://www.elexis.ch) . Afin de rester au courant de l'actualité concernant des nouveaux plugins nous vous conseillons de consulter de temps en temps ce site.


\bigskip
\index{Commande de développement}\index{Plugins supplémentaires}\index{Plugin!programmer}
\textbf{Développement spécifique }
Si vous avez besoin d'une fonction spécifique, qui n'est pas disponible dans la version de base, nous vous conseillons volontiers sur les possibilités et les coûts d'un développement correspondant.

% *******************************************************************************
% * Copyright (c) 2007 by Elexis
% * All rights reserved. This document and the accompanying materials
% * are made available under the terms of the Eclipse Public License v1.0
% * which accompanies this distribution, and is available at
% * http://www.eclipse.org/legal/epl-v10.html
% *
% * Contributors:
% *    G. Weirich
% *
% *  $Id: elexis.tex 2453 2007-05-30 15:16:13Z rgw_ch $
% *******************************************************************************
% !Mode:: "TeX:UTF-8" (encoding info for WinEdt)

\section{Standard-Installation}
\label{easyistall}
\index{installation}
Es gibt für Windows, Linux und Mac leicht installierbare und sofort lauffähige Pakete. Ausserdem kann eine demo-Datenbank eingerichtet werden, die schon eine Anzahl Pseudo-Patienten enthält.


Vergewissern Sie sich bitte zunächst, ob Ihr System von Elexis unterstützt wird (Anhang \ref{systemvoraussetzungen} dieses Handbuchs, Seite \pageref{systemvoraussetzungen}).\par
\index{Demoversion}
\textbf{Vollversion}: Wenn Sie nicht die Demo-Installation, sondern eine volle Praxisinstallation möchten, fangen Sie trotzdem so wie hier beschrieben an, und fahren dann gemäss der Anleitung im Anhang \ref{vollversion} ab Seite \pageref{vollversion} weiter.
\subsection{Windows 2000/XP/Vista}
\index{Windows}\index{Vista}
\begin{itemize}
	\item Laden Sie \href{http://www.elexis.ch/dl14.php?file=elexis-windows}{www.elexis.ch/dl14.php?file=elexis-windows} herunter (ca. 170MB)
	\item starten Sie die heruntergeladene Datei elexis-windows-x.y.z.exe und folgen Sie den Anweisungen am Bildschirm.
    \item Programm starten: Doppelklick auf das Symbol \glqq Elexis\grqq{} auf Ihrem Desktop oder das Programm \glqq Elexis\grqq{} im Startmenü auswählen.
	\item Deinstallation: Klicken Sie auf \glqq Elexis deinstallieren\grqq im  Startmenü.
\end{itemize}

\subsection{Linux}
(getestet mit SuSE 10.0, Kubuntu 6.06, 6.10, 7.04 und 8.04 sowie  Xubuntu 6.10, 7.04, 7.10., 8.04 und 8.10.)\footnote{Bei Gnome gibt es Probleme mit OpenOffice; wir empfehlen daher zur Zeit nur Kubuntu (KDE) und Xubuntu (XFCE), nicht aber Ubuntu (Gnome).}
\index{Linux}
\begin{itemize}
	\item Laden Sie \href{http://www.rgw.ch/dl14.php?file=elexis-linux}{www.rgw.ch/dl14.php?file=elexis-linux} herunter (ca. 90MB)
	\item Entpacken Sie das	heruntergeladene Archiv	elexis-linux-x.y.z.tgz an eine beliebige Stelle\footnote{Z.B. mit dem Befehl tar -xzf elexis-linux-x.y.z.tgz}
    \item Programm starten: Starten Sie einfach das Programm \footnote{meistens so: ./elexis}. Selbstverständlich können Sie auch unter Linux einen Shortcut auf dem Desktop
    anlegen, das dazu nötige Vorgehen variiert aber je nach desktop environment.

 	\item Textprogramm einrichten: Da bei den empfohlenen Linux-Distributionen
 	Open\-Of\-fi\-ce ohnehin integriert ist, liefern wir es beim Linux-Installer nicht
 	mit. Daher ist dieser auch erheblich kleiner, als der Windows-Installer. Dafür
 	ist noch ein wenig 	\glqq Handarbeit\grqq{} notwendig, um die Zusammenarbeit
 	von Elexis mit OpenOffice einzurichten (Anleitung für Kubuntu/Xubuntu, SuSE
 	analog mit Yast statt mit apt-get):
	\begin{itemize}
	 	\item öffnen Sie eine Konsole und geben Sie ein: \textit{sudo apt-get in\-stall
	 	openoffice.org-office\-bean}
		\item öffnen Sie in Elexis das Menü \textsc{Datei - Einstellungen} und suchen
		Sie dort die Seite \textsc{Textverarbeitung} auf. Markieren Sie den Punkt
		\glqq Open\-Of\-fice Wrap\-per\grqq{}.
		\item Gehen Sie dann im selben Dialog zur Seite \textsc{OpenOffice.org}.
		Suchen Sie mit dem Knopf \textsc{Durchsuchen} Ihr OpenOffice-
		Programmverzeichnis auf (in aller Regel wird dies bei Kubuntu
		/usr/lib/openoffice/program sein). Klicken Sie dann auf \textsc{Anwenden} und
		schliessen Sie den Dialog mit \textsc{OK}
		\item Wichtig: Verlassen Sie Elexis, warten Sie einige Sekunden und starten
		Sie neu.
    \end{itemize}
  \item Deinstallation: Löschen Sie den beim Entpacken erstellten Ordner. Das
 ist alles.
\end{itemize}

\subsection{Apple Macintosh OS-X}\
(Getestet Version 10.4 (Tiger) und 10.5 (Leopard). Einschränkung: Kein integriertes Textsystem)
\index{Macintosh}\index{Apple}
\begin{itemize}
	\item Laden Sie \href{http://www.elexis.ch/dl14.php?file=elexis-macosx}{www.elexis.ch/dl14.php?file=elexis-macosx} herunter (ca. 25 MB)
	\item Entpacken Sie das heruntergeladene Archiv elexis-macosx-x.y.z.zip an eine beliebige Stelle.
    \item Programm starten: Öffnen Sie das herunter\-geladene Verzeichnis und doppel\-klic\-ken Sie auf das Pro\-gramm \glqq Elexis\grqq{}
	\item Deinstallation: Löschen Sie den beim Entpacken erstellten Ordner. Das ist alles.
\end{itemize}

Bei allen Betriebssystemen wird beim ersten Start die Einrichtung des Programms komplettiert und eine leere Datenbank erstellt. Der Benutzername zum ersten Anmelden ist 'Administrator' das Passowort 'admin'. Beachten Sie, dass sie mit dieser leeren Datenbank noch nicht sehr viel anfangen können. Wenn Sie Elexis einfach mal ausprobieren wollen, möchten Sie vielleicht die Demo-Datenbank benutzen:

\subsection{Demo-Datenbank}
(Für alle Betriebssysteme.) Laden Sie \href{http://www.elexis.ch/files/demoDB.zip}{http://www.elexis.ch/files/demoDB.zip} herunter. Entpacken Sie dieses Archiv in Ihr Elexis-Verzeichnis (es enthält einen Ordner namens 'demoDB'). Starten Sie dann Elexis neu. Es wird sich nun mit dieser De
mo-Datenbank verbinden, welche bereits einige Beispielpatienten etc. enthält. Anmeldename ist hier 'test', das Passwort ebenfalls 'test'.

Wir empfehlen, dass Sie dann zum \glqq warmwerden\grqq{} die geführte Tour (s.S.
\pageref{tour}) machen.




%%%%%%%%%%%%%%%%%%%%%%%%%%%%%%%%%%%%%%%%%%%%%%%%%%%%%%%%%%%%%%%%%%%%%%%%%%%%%%
\part{Visite guidée}
\chapter{Saisir une nouvelle patiente}
\label{tour}
	% *******************************************************************************
% * Copyright (c) 2007-2010 by Elexis
% * All rights reserved. This document and the accompanying materials
% * are made available under the terms of the Eclipse Public License v1.0
% * which accompanies this distribution, and is available at
% * http://www.eclipse.org/legal/epl-v10.html
% *
% * Contributors:
% *    G. Weirich - initial implementation
% *
% *  $Id: patneu.tex 4903 2009-01-03 11:44:22Z rgw_ch $
% *******************************************************************************
% !Mode:: "TeX:UTF-8" (encoding info for WinEdt)

Starten Sie Elexis durch Doppelklick auf das Programmsymbol.
Nach einem kurzen Moment erscheint ein Fenster etwa wie in Abb. \ref{fig:startbild} \footnote{Die meisten Abbildungen dieser Tour entstammen Windows XP. Unter anderen Betriebssystemen wird das Aussehen leicht abweichend sein.}.
 \begin{figure}[ht]
    \center
	\includegraphics[width=0.9\textwidth]{images/einf0}
    %\includegraphics{images/einf0}
	\caption{Elexis Startbildschirm}
	\label{fig:startbild}
\end{figure}
\section{Patientendaten erfassen}
\begin{wrapfigure}{r}{8cm}
	\includegraphics[width=6.5cm]{images/einf1}
	\caption{Patientennamen eingeben}\label{fig:patname}
\end{wrapfigure}
Aktivieren sie mit einem Klick die Ansicht 'Patienten' und schreiben Sie in die Eingabefelder Name und Vorname der neuen Patientin.
Falls die Patientin schon einmal erfasst worden ist erscheint ihr Name, in unserem Fall, wo keine Patientin dieses Namens vorhanden ist, werden unten keine Einträge angezeigt (s. Abb. \ref{fig:patname})\footnote{Um die Eingabefelder zu leeren und wieder alle Patienten anzuzeigen, können Sie auf das rote 'x' links neben den Eingabefeldern klicken.}.

\index{Patient!neu}
Klicken Sie dann auf das grüne Plus-Symbol oben rechts, um eine Patientin mit diesen
Daten neu anzulegen. Es erscheint ein Dialogfenster (Abb. \ref{fig:patdata}), wo
Sie die Angaben in die entsprechenden Felder eingeben können.\\
\bigskip

\begin{figure}[ht]
	\includegraphics{images/einf2}
	\caption{Patientendaten ergänzen}
	\label{fig:patdata}
\end{figure}
Sie brauchen die hier verlangten Daten nicht vollständig einzugeben, sondern einfach soweit Sie diese im Moment kennen.
Sie müssen also im Notfalldienst nicht zuerst die vollständigen Daten eingeben, bevor Sie mit der Behandlung beginnen können.
Erfassen Sie z.B. nur Name und Geburtsdatum und überlassen Sie den Rest Ihrer MPA. Für Elexis ist ein neuer Patient in dem Moment bekannt,
wo Sie 'OK' klicken -- egal wieviele Daten zu diesem Zeitpunkt eingegeben sind.

\section{Falldaten erfassen}
Bei einer neuen Patientin müssen Sie zunächst einen \glqq Fall\grqq{} erstellen, dem
die Konsultation zugeordnet werden kann.
\index{Fall!erstellen}

\begin{figure}[htbp]
     \begin{minipage}{0.4\textwidth}
      \centering
       \includegraphics[width=0.8\textwidth]{images/einf3}
       \caption{Fälle-Ansicht}
       	\label{fig:faelle1}
     \end{minipage}\hfill
     \begin{minipage}{0.6\textwidth}
      \centering
       \includegraphics[width=1.0\textwidth]{images/einf4}
       \caption{Fall-Detail}
       \label{fig:falldetail}
     \end{minipage}
   \end{figure}


Ein Fall sammelt alle Konsultationen, die mit einem gemeinsamen Abrechnungssystem erfasst werden. (Vgl. \ref{settings:abrechnungssystem} auf S. \pageref{settings:abrechnungssystem}).
Klicken Sie also in der \glqq Fälle\grqq{}-Ansicht
(Abb. \ref{fig:faelle1}) auf das grüne Plus-Symbol.

Dadurch öffnet sich ein Dialog, in dem Sie wiederum die Angaben eintragen
können, soweit diese Ihnen bekannt sind (Abb. \ref{fig:falldetail})

Spätestens zur Rechnungsstellung müssen dann allerdings die notwendigen Angaben (Debitor, Kostenträger und Versicherungsnummer bzw. Fallnummer)
eingegeben werden.
Nach dem Klick auf OK haben Sie den neuen Fall erstellt. Bei weiteren
Konsultationen kann man sich diesen Schritt natürlich sparen.\\

\bigskip

Als nächstes erstellen wir eine neue Konsultation, wieder mit dem nun schon
bekannten grünen Plus-Symbol (Abb. \ref{fig:neuekons}.

\index{Konsultation!neu}
\begin{figure}[ht]
	\includegraphics{images/einf5}
	\caption{Neue Konsultation}
	\label{fig:neuekons}
\end{figure}
\pagebreak[2]

Danach können wir mit dem KG-Eintrag beginnen (Abb. \ref{fig:KG}).

\section{Krankengeschichte führen}
\begin{figure}
    \begin{center}
	   \includegraphics[width=0.7\textwidth]{images/einf6}
    	\caption{KG-Eintrag}
	   \label{fig:KG}
    \end{center}
\end{figure}
Der KG-Eintrag kann einfache Textformatierungen enthalten, Textbausteine können
beliebig definiert und über eine konfigurierbare shortcut-Taste aufgerufen werden. Die Verrechnung erfolgt dann entweder über ein Tastaturmakro oder per Maus.
Nach Fertigstellung des Eintrags (auch vor oder während des Eintragens) können
Sie durch Klicken auf \glqq Verrechnung\grqq{} die Leistungen-Ansicht öffnen
(Abb. \ref{fig:Verrechnung}). Analog können sie durch klicken auf
\glqq Behandlungsdiagnosen\grqq{} die Diagnosen-Ansicht öffnen.

\begin{figure}[ht]
	\includegraphics[width=6cm]{images/einf7}
	\caption{Verrechnungs-Fenster}
	\label{fig:Verrechnung}
\end{figure}
Dieses Fenster enthält alle im System vorgesehenen Leistungscode-Systeme, sowie eine Seite mit selbstdefinierten Leistungsblöcken.
\index{Leistungen!verrechnen}\index{abrechnen}
Sie können entweder einen ganzen Block oder einzelne Leistungen aus dem Block oder aus einem anderen Leistungsfenster (Tarmed etc.) ins \glqq Verrechnung\grqq{}-Feld ziehen (drag and drop).

Genau gleich lassen sich zur Konsultation auch Diagnosen zuordnen, auch hier hat man die Wahl zwischen allen im System integrierten Diagnosecodesystemen (beliebig anzupassen und erweiterbar).
\index{Konsultation!Diagnose}

\clearpage

\chapter{Aménager l'interface utilisateur}
	\label{customize}
	% *******************************************************************************
% * Copyright (c) 2007-2010 by Elexis
% * All rights reserved. This document and the accompanying materials
% * are made available under the terms of the Eclipse Public License v1.0
% * which accompanies this distribution, and is available at
% * http://www.eclipse.org/legal/epl-v10.html
% *
% * Contributors:
% *    G. Weirich - initial implementation
% *
% *  $Id: customize.tex 4896 2009-01-02 08:23:59Z rgw_ch $
% *******************************************************************************
% !Mode:: "TeX:UTF-8" (encoding info for WinEdt)

\section{Funktionsprinzip}
Hervorstechendstes Merkmal vom Elexis ist die grosse Flexibilität. Wenn Sie ein anderes Praxisprogramm gewöhnt sind, wird Ihnen die Bedienung von Elexis vielleicht etwas ungewöhnlich vorkommen. Wir möchten deshalb hier zunächst einige grundsätzliche Konzepte erläutern.

\index{Bedienungskonzepte}
 \subsection{Schreibtisch / Perspektive}
 \index{Perspektive}\index{Ansicht}\index{View}
Stellen Sie sich Ihren Arbeitstisch vor. Vermutlich werden Sie sich im Lauf der Zeit angewöhnt haben,
bestimmte Dinge an einen bestimmten Ort auf Ihrem Schreibtisch zu legen, also
Arbeitsfunktionen einem Ort zuzuordnen, wo sie sie jeweils (idealerweise) leicht wiederfinden. Ihre Anordnung ist nicht unbedingt dieselbe wie bei jemand anderem, der dasselbe Schreibtischmodell besitzt.

Das Programmfenster von Elexis ist so ein Schreibtisch (s. fig. \ref{fig:tour1}.Es ist in keiner Weise festgelegt, welche Funktion wo zu finden ist, ja es ist nicht einmal festgelegt, welche Elemente überhaupt auf dem Schreibtisch erscheinen, und welche vielleicht irgendwo in einer Schublade verstaut sind und nur bei Bedarf hervorgeholt werden müssen. Ähnlich wie bei einem echten Schreibtisch kann es auch zu einem heillosen Durcheinander kommen, in dem man überhaupt nichts mehr findet. Ein gutes Ablageprinzip wird sich erst im Verlauf der Arbeit einspielen.

%\usepackage{graphics} is needed for \includegraphics
\begin{figure}[htp]
\begin{center}
  \includegraphics[width=0.9\textwidth]{images/tour1}
  \caption{Standard-Perspektive}
  \label{fig:tour1}
\end{center}
\end{figure}

Eine Anordnung von Arbeitsflächen nennen wir eine 'Perspektive' (perspective). Die einzelnen Unterfenster bzw. Funktionseinheiten (oben 'Patienten' und 'Patienten-Detail'), aus denen sich die Perspektive zusammensetzt bezeichnet man als 'Ansicht' (View).

\subsubsection{Perspektive und Views}

Oben (fig. \ref{fig:tour1} sehen sie als Beispiel eine Perspektive, die für einen kleinen Bildschirm geeignet ist, es zeigt einen Screenshot auf einem 15-Zoll-TFT-Monitor. Die Ansichten \glqq
Patienten\grqq{}(links) und \glqq Patient Detail\grqq{}(rechts) liegen obenauf,
andere Ansichten sind dahinter angeordnet, so dass nur ein Karteireiter oben
zu sehen ist.

Auf einem grösseren Bildschirm würden Sie vermutlich eine andere Anordnung
bevorzugen: fig. \ref{fig:tour2} zeigt einen Screenshot auf einem
17-Zoll-TFT-Monitor mit mehreren Views gleichzeitig.

%\usepackage{graphics} is needed for \includegraphics
\begin{figure}[htp]
\begin{center}
  \includegraphics[width=0.9\textwidth]{images/tour2}
  \caption{Komplexere Perspektive}
  \label{fig:tour2}
\end{center}
\end{figure}

\subsubsection{Ansichten / Views}
Jede Ansicht entspricht einer bestimmten Funktionalität. Im abgebildeten Fenster sehen sie die Ansicht einer Patientenliste (links) und die Details des 'aktiven' Patienten (rechts). Es gibt weitere Ansichten wie KG-Eintrag des aktuellen Patienten, eine Liste aller KG-Einträge, die Fixmedikation, die Rezepte,die Arbeitsunfähigkeitszeugnisse,
die Agenda etc. Jede Ansicht ist eine definierte \glqq Sicht\grqq auf die vorhandenen Daten, daher der Name \glqq Ansicht\grqq. Sie lassen sich über die Reiter aktivieren. Die Reiter selber lassen sich beliebig anordnen, aktivieren oder deaktivieren.

Egal wie Sie die Views angeordnet haben, jede View lässt sich zur besseren
Übersicht jederzeit auf Vollbildgrösse bringen, indem man auf den Reiter
doppelklickt (s. Abb. \ref{fig:tour3}).

\begin{figure}[htp]
\begin{center}
  \includegraphics[width=0.9\textwidth]{images/tour3}
  \caption{View maximieren}
  \label{fig:tour3}
\end{center}
\end{figure}

\subsubsection{Views und Perspektiven anpassen}


In der Standard-Startperspektive ist links eine \glqq Startleiste\grqq{} zu
sehen. Diese führt Sie zu vordefinierten Perspektiven - es erscheinen in diesen
jeweils die passenden Ansichten. Die Werkzeugleiste führt, wie bei anderen
Programmen üblich, zu verschiedenen Funktionen. Jede Ansicht hat einen
Karteireiter, über den sie in den Vordergrund gebracht oder maximiert werden kann.
\par
\clubpenalty=5000
\medskip
Die Programmfenster-Inhalte und Zusammenstellungen der Ansichten lassen sich leicht Ihren indivicuellen Wünschen anpassen:\\

\begin{itemize}
  \item Sie können nicht benötigte Ansichten entfernen damit sie mehr Platz für
  die verbleibenden Ansichten haben
	\item können Views in der Horizontalen und Vertikalen vergrössern oder verkleinern
	\item können Views an beliebige andere Stellen des Bildschirms schieben (indem Sie sie
	an den Reitern mit gedrückter linker Maustaste \glqq festhalten\grqq{})
\end{itemize}

Jede Zusammenstellung kann als Perspektive gespeichert werden - und ist als
solche auf einfache Art wieder aufrufbar.

\subsection{Perspektiven einrichten und speichern}
\label{tour:customize}
Sie können nicht nur eine Perspektive erstellen, sondern beliebig viele. Ihre MPA braucht möglicherweise eine andere Perspektive als Sie selber, z.B. wünscht sie sich die Agenda gross. Oder Sie selber verwenden unterschiedliche Perspektiven, z.B. eine für Konsultationen und eine andere für die Buchhaltung oder wenn Sie einen Bericht schreiben. Perspektiven lassen sich mit Elexis in wenigen Schritten zusammenstellen:\\

\bigskip

\textbf{Schritt 1: Benötigte Ansicht(en) öffnen}\\

Wählen Sie im Menu \textsc{Fenster - Ansicht - Andere}. Es öffnet sich ein Dialog wie in Abb. \ref{fig:cust1}.\\

\begin{figure}[htbp]
   \begin{minipage}{0.4\textwidth}
       \centering
       \includegraphics[width=0.8\textwidth]{images/customize1}
       \caption{View suchen}
       \label{fig:cust1}
     \end{minipage}\hfill
     \begin{minipage}{0.5\textwidth}
        Diese Dialogbox zeigt Ihnen sämtliche Views, die in Ihrer Elexis-Installation vorhanden sind (Welche und wieviele das sind, hängt von den installierten Plugins ab). Wenn Sie in der obersten Zeile den Anfang der gesuchten View zu tippen beginnen, wird die Liste automatisch gefiltert.

        Wenn Sie den Namen der gesuchten View nicht wissen, können Sie hier natürlich auch einfach alle Views durchblättern.
    \end{minipage}
\end{figure}

\bigskip
\pagebreak[3]
\textbf{Schritt 2: Die Ansichten an die gewünschte Stelle schieben und auf die gewünschte Grösse bringen:}\\

   \includegraphics[width=0.8\textwidth]{images/agendagewaehlt}

   \includegraphics[width=0.8\textwidth]{images/agendaanpassen}

\bigskip
\textbf{Schritt 3:Perspektive speichern}\\
\index{Perspektive!speichern}
Wenn Sie möchten, dass Ihre so erstellte Perspektive auch bei späteren Programmstarts oder auf anderen Computern zur Verfügung steht, haben Sie mehrere Möglichkeiten:
\begin{itemize}
\item Wählen Sie im Menu \textsc{Fenster - Perspektive - als Startperspektive speichern}. Damit legen Sie fest, dass die eben zusammengestellte Perspektive fortan beim Anmelden erscheint.
\item \textsc{Fenster - Perspektive - speichern als...}. Damit können Sie die aktuelle Perspektive unter einem frei wählbaren Namen abspeichern. Sie können sie zu einem späteren Zeitpunkt jederzeit mit \textsc{Fenster - Perspektive - andere...} mit diesem Namen wieder zurückholen.
\item \textsc{Fenster - Perspektive - speichern}. In diesem Fall wird die aktuelle Perspektive unter ihrem bestehenden namen gespeichert.
\item Und, last but not least, Wenn Sie unter \textsc{Datei - Einstellungen - Anwender} die Perspektive unter einem bestimmten Namen speichern, dann können Sie sie auch von einer anderen Arbeitsstation aus unter diesem Namen einlesen\footnote{Dies geht allerdings natürlich nur, wenn auch auf der aufrufenden Arbeitsstation alle Views, die in dieser Perspektive definiert sind, auch vorhanden sind}. Diese Option eignet sich, um einheitliche Arbeitsumgebungen zuammenzustellen.
\end{itemize}

\textbf{Oder: Perspektive zurücksetzen}\\
\index{Perspektive!zurücksetzen}
Falls Ihnen die gemachten Änderungen doch nicht zusagen, oder wenn Sie beispielsweise versehntlich eine View geschlossen haben, können Sie ganz einfach zur gespeicherten Vesion der aktuellen Perspektive zurückkehren: Wählen Sie im Menu \textsc{Fenster - Perspektive - Wiederherstellen.} Dies geht natürlich nur, solange Sie Ihre Änderungen nicht gespeichert haben (Schritt 3).




%%%%%%%%%%%%%%%%%%%%%%%%%%%%%%%%%%%%%%%%%%%%%%%%%%%%%%%%%%%%%%%%%%%%%%%%%%%%%%%
\part{Référence systématique}
\chapter{Concepts}
	% *******************************************************************************
% * Copyright (c) 2007-2008 by Elexis
% * All rights reserved. This document and the accompanying materials
% * are made available under the terms of the Eclipse Public License v1.0
% * which accompanies this distribution, and is available at
% * http://www.eclipse.org/legal/epl-v10.html
% *
% * Contributors:
% *    G. Weirich
% *
% *  $Id: konzepte.tex 5231 2009-04-02 13:16:56Z rgw_ch $
% *******************************************************************************
% !Mode:: "TeX:UTF-8" (encoding info for WinEdt)

\section{Contacts}
\label{kontakt}
\index{contact!définition}
Dans Elexis, chaque personne ou entreprise qui est dans une relation avec le cabinet, est
tout d'abord un\glqq contact\grqq{}. La saisie ou la modification des contacts se fait dans la perspective de contact.
\begin{flushleft}
    \includegraphics{images/contactperspective}
\end{flushleft}


Ils existent les types de contact suivants :
\begin{itemize}
  \item personne
	\begin{itemize}
  		\item mandant
  		\item utilisateur
  		\item patient
  		\item autres
    \end{itemize}
    \item{oganisation}
    \begin{itemize}
      \item{laboratoire}
      \item {autres}
    \end{itemize}
\end{itemize}


\section{Utilisateurs et Mandants}
\index{utilisateur!définition}\index{mandant!défintion}
Quelqu'un qui a le droit de facturer ses prestations à charge de l'assurance obligatoire de soins (ce qui est en Suisse seulement possible si on a son propre numéro de concordat RCC ), est un \textit{mandant}. Chaque processus dans Elexis (consultation, laboratoire, prescription etc.) se passe toujours sous la responsabilité et sur le compte précisément d'un Mandant. \index{mandant}

\medskip

JQuelqu'un qui peut manier le programme , est un \textit{utilisateur}. Un utilisateur travaille toujours sur ordre d'un Mandant spécifique.

Ainsi il existe à chaque moment dans Elexis un Mandant actuel et un utilisateur actuel.
\index{utilisateur}Mandant und Anwender können auch identisch sein (Wenn der
Mandant selbst am PC arbeitet).
Le Mandant et l'utilisateur peut aussi être identique (si le Mandant lui-même travaille au PC). Un utilisateur peut aussi modifier l'attribution à un Mandant (si une assistante médicale dans un cabinet de groupe travaille par exemple pour des Mandants différentes).
\index{groupes}\index{droits}
\index{droits}\index{utilisateur!droits}
Des utilisateurs ont certains droits individuellement réglables, avec lesquels on peut contrôler très finement à qui on permet quelles actions dans Elexis. Des utilisateurs peuvent aussi être rassemblés dans des groupes qui définissent certains droits communs (p. ex. groupes  \glqq assistantes médicales\grqq{} ou \glqq médecins\grqq{}). Le groupe \glqq Admin\grqq{}: est un groupe spécial : Celui qui fait partie de ce groupe, a automatiquement  \textit{tous} les droits.

\medskip

\textbf{Important}: Même si cela peut d'abord vous apparaître illogique : Aussi le chef ne devrait pas travailler habituellement comme Admin \index{administrateur}.
La raison se trouve dans le fait que l'Admin-Account permet aussi des suppressions irréversibles et d'autres modifications très désagréables. Dans l'état fiévreux du quotidien on risque facilement de cliquer une fois sur le faux bouton !
Par conséquent : Travaillez dans le quotidien avec un Account qui donne précisément les droits dont vous avez besoin au quotidien. Etablissez un deuxième Account pour vous,celui qu'est assigné au groupe Admin, et ne vous annoncez sous cet Account que lorsqu'il est vraiment nécessaire.

Le concept des groupes et droits est expliqué plus précisément à partir de la page\pageref{sec:gruppen} .

\section{Consultations, cas, garants et répondant des coûts}
\index{consultation!définition}\index{cas!définition}\index{facturation} Chaque contact retenu dans les Elexis entre le personnel du cabinet et un patient est une  \textit{consultation}. Lorsqu'une consultation est comptabilisé la facturation sera faite en faveur du Mandant, pour lequel l'utilisateur connecté a travaillé.
\label{definition:fall}
Chaque consultation est aussi assignée à un \textit{cas}. Ein Fall ist hier eher eine versicherungstechnische, als eine medizinische Einheit: Un cas est ici plutôt une entité assécurologique qu'une entité médicale : Le cas rassemble toutes les consultations qui sont comptabilisées avec le même système de facturation (voir\ref{settings:abrechnungssystem}à la page \pageref{settings:abrechnungssystem}). Cela peut parfois être identique avec la notion de cas médical (un accident qui est facturé à un assureur spécifique avec un numéro de cas spécifique), ou ne peut pas avoir de lien avec un cas médical  (p. ex. en général en Suisse un cas de maladie sera subsumé à la \glqq maladie\grqq{} qui rassemble toutes les consultations LAMAL).

Un cas ne peut être attribué à qu'un seul patient et un système, mais peut toutefois comprendre des consultations de plusieurs Mandants. (Une facture distincte est alors fournie pour chaque Mandant).

\section{Sticker}
\index{stickers}\index{étiquettes}\index{marquage}
\label{Etiketten}
Des patients et d'autres contenus de base de données peuvent être marqués avec des 'stickers' (étiquettes). Un 'sticker' est une caractéristique en principe arbitraire qui est liée avec un objet correspondant de la base de données. Par exemple un patient pourraient être marqué avec le 'sticker' 'modèle de médecin de famille', 'MRSA' ou autre. Un tel 'sticker' est affichée lors de l'appel de l'objet correspondant.

 Des 'stickers' sont définis sous \textsc{Fichier-Paramètres-Sticker} (page Fig. \ref{fig:etiketten1}). ELa quantité des stickers peut être définie au choix. Pour créer un nouveau sticker, écrivez le texte pour le sticker dans le champ en haut et cliquez sur 'nouveau sticker'. Le 'sticker' que vous venez de créer apparaît alors avec des valeurs standards sur la liste des stickers. Sélectionnez le sticker et suggérez une image (format 16x16 tout au plus à 32x32 pixels), une couleur de texte et une couleur d'arrière-plan.
L'importance de la  'valeur' vous est expliquée ci- dessous.


\begin{figure}
    \includegraphics{images/etikette1}
    \caption{Sticker erstellen}
    \label{fig:etiketten1}
\end{figure}

Des stickers crées de cette façon peuvent être attribués à un patient qui se trouve sur la liste des patients par un clic sur la touche droite de la souris ce qui permet de choisir un 'Sticker' sur un menu déroulant. On peut attribuer à chaque patient entre zéro et x 'stickers'. On trouvera dans la liste des patients la saisie correspondante avec les 'stickers' et images, couleur du texte et couleur d'arrière plan.  (Fig. \ref{fig:etiketten2}).
\begin{figure}
    \includegraphics{images/etikette3}
    \caption{Anzeige der Sticker}
    \label{fig:etiketten2}
\end{figure}

Ici, on voit alors aussi le sens de l'attribut 'valeur' d'un sticker:
Lorsqu'un patient a reçu plusieurs stickers assignés, la liste des patients indique toujours
celui avec  'la valeur' la plus élevée. Les chiffres que vous utilisez là concrètement n'ont pas d'importance car la valeur absolue ne joue pas de rôle, par contre les relations entre les 'valeurs'.


\medskip

Si vous ouvrez une consultation pour y faire vos notes, vous y voyez toutes les stickers assignées au patient.
\begin{figure}
    \includegraphics{images/etikette2}
    \caption{Konsultation mit Stickern}
    \label{fig:etiketten3}
\end{figure}


\clearpage

\section{Décompte des prestations}
\label{concept:leistung}
\index{Codes de prestation}\index{faire les comptes}
\begin{wrapfigure}{l}{7.5cm}
    \includegraphics[width=7.5cm]{images/leistungen1}
    \caption{Leistungen}
    \label{fig:leistungen}
\end{wrapfigure}
\index{prestation!comptabiliser}
Les codes de prestations qui peuvent être comptabilisé, sont fournis d'une part par les Plugins (p. ex. par Elexis-tarif médical-Suisse), d'autre part par des blocs de prestations définis par vous-même (voir ci-dessous).
Vous trouvez tous les systèmes de codage de prestations existants sous la fenêtre
\glqq prestations\grqq{} (Fig. \ref{fig:leistungen}): Vous voyez au bord inférieur de la fenêtre un onglet pour chaque système de codage installé. 
Cette fenêtre apparaît lorsque vous cliquez depuis la fenêtre de consultation sur
 \glqq{}saisie prestations\grqq{}. La structure est la même pour chaque système de codage de prestations :

La fenêtre partielle supérieure montre les codes les plus fréquemment appliqués par vous dans ce système de codage de prestations, chose qui vous permet un accès rapide sans chercher longtemps. La liste est mise à jour régulièrement et plus vous utilisez un certain code, plus haut dans la liste il apparaîtra lors de la prochaine ouverture de la fenêtre.


\medskip
Dans la partie moyenne, les codes jusqu'ici utilisés les plus fréquemment pour ce patient apparaissent triés selon
le même principe. La fenêtre partielle la plus basse met à disposition le système de codage de prestation entier
avec toute sa systématique.


\bigskip

Pour introduire un code de prestation, vous pouvez le tirer d'une des trois sections de la fenêtre de 'prestations' dans la fenêtre de 'saisie prestations' ou le choisir par double-clic.
Certains Plugins peuvent contenir un  \glqq Optifier\grqq (Optimizer/Verifier) qui reconnaît des erreurs et/ou peut appliquer des corrections. Ainsi refuse par exemple le Tarmed- Plugin d'une part la facturation double du code \textit{00.0010 Consultation 5 premières minutes } avec un message d'erreur  (Verifier), et d'autre part il ajoute automatiquement le code 00.0010 lorsque vous introduisez le code 00.0030 \textit{00.0030 Consultation 5 dernières minutes} car 00.0030 se combine toujours avec 00.0010 (Optimizer).
Les\textit{articles}, qui sont remis directement au patient peuvent aussi être comptabilisés directement depuis cette fenêtre et leur stock est automatiquement ajusté.

\subsection{Blocs de prestations et prestations propres}
\begin{wrapfigure}{r}{6cm}
\includegraphics{images/block1}
\caption{Leistungsblöcke}
\label{fig:bloecke}
\end{wrapfigure}
\index{prestations propres}
\index{blocs de prestations}
Comme autre allègement du travail Elexis permet aussi de résumer plusieurs codes de prestation dans des blocs de prestation qui seront comptabilisé entièrement ou partiellement, et ceci même si ces blocs proviennent des systèmes de codage de prestation tout à fait différents. A part des blocs de prestations de provenance des systèmes de codage de prestations pré installés, de tels blocs peuvent contenir aussi des éléments à comptabiliser définis par vous même.
Vous voyez dans la Fig.\ref{fig:bloecke} quelques exemples :\textit{cons15} est un exemple de bloc qui est comptabilisé
généralement en bloc. Pour ce faire vous pouvez, tirer le bloc avec la souris dans la fenêtre de saisie de prestation de la consultation ou, si vous travaillez plutôt avec le clavier, vous tapez le nom du bloc, suivi de la clé de libération des macros dans le texte de consultation
(conformément aux normes c'est le \#). L'entrée de cons15\# dans le texte de consultation comptabiliserait ainsi dans notre exemple une consultation de 15-minutes selon tarif Tarmed.
\textit{Les Vaccinations} seraient un exemple de bloc, qui serait plutôt pensé comme sommaire d'éléments semblables
(avec le but d'un gain de temps pour trouver plus rapidement l'élément cherché) qui seront de toute façon comptabilisé
séparément. Dans ce cas, on tire simplement les différents éléments du bloc dans la fenêtre de saisie de prestation .


Pour créer un nouveau bloc, on introduit (à libre choix mais toutefois unique) un nom pour ce bloc et clique ensuite sur \textit{créer nouveau ...}. Les prestations s'ajoutent au bloc dans la fenêtre des codes(voir  Fig. ref{fig:bloecke2}).
\begin{figure}[htp]
\includegraphics[width=0.9\textwidth]{images/block2}
\caption{Leistungsblock definieren}
\label{fig:bloecke2}
\end{figure}
Vous pouvez ajouter par drag\&drop soit des prestations prédéfinies de provenance d'un des systèmes de code installés, soit définir vos propres prestations. Ici vous devez aussi indiquer frais et prix en centimes/cents ainsi que le temps budgétisé pour la prestation en minutes.


\section{Articles et stocks}
\index{article}\index{médicament}\index{stock}
 Tout ce qui peut être acheté, stocké, livré ou prescrit est un  \textit{article}.
 Les articles sont organisés dans des classes par exemples classes au \textit{médicaments},
 ou à des  \textit{LiMA} ou au\textit{matériel bureautique}.
 Elexis peut adopter chaque article qui lui est connu comme \textit{article en stock}.

 Un article en stock est un article dont l'existence est contrôlée et qui peut être commandé si nécessaire de façon semi-automatique. Des plus amples informations concernant les articles et leur stockage se trouvent dans la description de la View (à la page \pageref{view:artikel} et suivantes.)

\section{Importation des données externes }
\index{importation}
En principe Elexis est en mesure d'importer des données de n'importe quelle source .
Toutefois, le format de ces données doit naturellement être connu ou standardisé d'une certaine manière. Par conséquent l'importation de données est en générale effectué par les 'Importer-Plugins'.
Il y existent des 'Importers' pour des données d'annuaire téléphonique, pour les bases de données d'autres programmes pour le cabinet médical, pour des laboratoires externes, pour des appareils de laboratoire et pour d'autres appareils médicaux qui sont capables de transférer leurs données sur un ordinateur, pour des LiMA, des médicaments,
le Tarmed et d'autres bases de données etc.
Vous trouvez une liste de Plugins disponibles dans le menu 'Plugins' sur  http://www.elexis.ch. Des 'Importers' supplémentaires peuvent être programmés assez facilement, en cas de besoin vous pouvez nous contacter pour demander un devis.

\medskip

Les 'Importers' se trouvent généralement dans le menu local des 'Views' qui affichent les données correspondantes (p. ex. Tarmed-Importer ou Importer du laboratoire). Une classe d' 'Importers' qui ne sont pas attribués à certaine 'View' se trouvent aussi dans le menu d' 'importation de données' sous 'Fichier' dans la barre de menu. Ici, un dialogue s'ouvre comme dans la fig. \ref{fig:importdlg}.
\begin{figure}
  % Requires \usepackage{graphicx}
  \includegraphics{images/importdlg}\\
  \caption{Import-Dialog}\label{fig:importdlg}
\end{figure}
\index{Import!contacts}
Dans les onglets en bas de la fenêtre vous trouvez tous les 'Importers' de base installés. Lesquels s'y trouvent dépend des Plugins installés. Seulement le 'Contact-Importer' qui est choisi dans l'illustration existe toujours. Cet 'Importer' peut importer des contacts des fichiers externes, pour autant que ceux-ci soient préparés de façon standardisée en forme de tableaux. Choisissez sous 'le type de fichiers' s'il s'agit d'un tableau Microsoft\texttrademark Excel\texttrademark(xls), d'un tableau Caracter Separated Values (csv) ou d'un tableau de Santésuisse contenant les assureurs et leurs codes EAN.
S'il s'agit de xls, le fichier doit contenir un tableau 0 avec les colonnes suivantes :


\begin{tabular}[h]{|l|l|}
\hline Titre de colonne & Légende\\
\hline
\hline  ID & une identification (en principe au choix mais univoque dans le fichier)\\
\hline Istpersonne & 1 si la saisie concerne une personne, 0 pour tout les autres cas (organisations etc)\\
\hline Istpatient & 1 si la saisie concerne un patient, 0 pour tout les autres cas\\
\hline titre & titre, personne de référence etc.\\
\hline désignation1 & En cas d'une personne son nom de famille\\
\hline désignation2 & En cas d'une personne son prénom\\
\hline supplément & \\
\hline date de naissance & En format dd.mm.yyyy ou yyyy-mm-dd\\
\hline genre & m ou f ou un mot qui commence avec m ou f\\
\hline e-Mail & adresse E-Mail\\
\hline website & une adresse WWW\\
\hline téléphone 1 & Numéro de téléphone primaire\\
\hline téléphone 2 & Numéro de téléphone supplémentaire\\
\hline mobil & Numéro de téléphone portable\\
\hline rue & Rue et numéro de maison\\
\hline code postal & code postal écrit comme 1224 ou comme CH-1224 (doit être formaté comme fichier texte)\\
\hline localité & \\
\hline adresse & Adresse comme elle apparaîtra sur une étiquette d'adresse. Nouvelle ligne par $\backslash$n\\
\hline EAN & Code EAN comme EAN13\\
\hline
\end{tabular}

\medskip

Pour que le format puisse être reconnu la première ligne du tableau doit contenir les titres de colonne précisément dans la forme de présentation ci-dessus. Chacune des colonnes citées doit exister mais peut toutefois être vide. Le fichier doit donc être codé comme iso-8859-1 (c'est une norme sous Windows ; avec la version MAC de Excel, le codage d'exportation devrait en conséquence éventuellement être adapté).


\medskip

Si vous avez fixé le type de fichier, cliquez vous sur le bouton 'choisir fichier' et cherchez le fichier à importer. Ne placez un crochet à 'préserver ID' seulement 
 \textbf{seulement},lorsque \begin{itemize}
\item chaque paquet de données dans le champ ID a une ID 
\item cette ID est univoque, et ne peut donc entrer en collision avec aucun autre contact dans Elexis
\item il est indispensable de maintenir cette ID.
\end{itemize}
Si vous ne placez pas le crochet, chose qui est recommandée dans la plupart des cas, alors Elexis fournira lors de l'importation une identité univoque pour chaque contact (comme si on introduisait manuellement ce contact).\\
Cliquez alors OK, pour commencer l'importation.

 \section{Plusieurs instances simultanément}
 \index{simultanément}
 Vous pouvez démarrer Elexis sans problèmes plusieurs fois simultanément pour afficher dans les fenêtres des perspectives différentes ou différents patients.
Certains éléments peuvent aussi être échangés par Cut\&Paste entre les instances courantes. Exemples d'application :

 \begin{itemize}
   \item Vous travaillez sur une entrée de patient et vous recevez un appel téléphonique  concernant un autre patient. Au lieu de quitter votre travail vous cliquez sur l'Elexis qui est ouvert en arrière plan et cherchez le dossier de ce patient.
   \item A son poste de travail votre assistante médicale voudrait avoir en même temps l'agenda et les données des patients à portée de vue. Si vous lui payez un deuxième écran (au lieu d'un deuxième PC), vous attachez les deux écrans au même PC à une carte graphique DualHead et mettez sur chaque écran une instance propre de Elexis.
   \item Pendant que l'Elexis est occupé à faire la facturation qui prend du temps, vous ne voudriez pas glander. Sans problème, vous démarrez une deuxième instance d'Elexis et continuez à travailler. (Vous pourriez naturellement aussi aller boire un café ou faire une promenade).
   \item Vous écrivez une lettre, mais vous aimerez y mettre certaines parties d'une autre lettre.
Ouvrez dans une instance d'Elexis l'ancienne lettre, copiez là certains passages et collez les dans l'autre lettre que vous êtes en train d'écrire dans l'autre instance d'Elexis.

 \end{itemize}

\section{Plugins}
\index{Plugin!définition}
Ce concept est discuté en détail à la page \pageref{expl:plugins}. Ici mentionnons pour le moment seulement : Elexis est extensible dans tout les sens. Il n'y a pas seulement un certain nombre de \glqq modules\grqq{} mais en fait, à tout moment, des nouvelles fonctions peuvent être programmé dont lors du lancement de la version actuelle on n'avait pas encore connaissance. Cela se fait sous forme de \glqq Plugins\grqq{}. Les 'plugins' peuvent être programmés par exemple pour, de la statistique, la comptabilité, l'importation de données de laboratoire, l'accessibilité des
appareils, l'exportation des données du dossier médical, des nouveaux systèmes de comptabilisation des prestations, des nouveaux systèmes de classifications des diagnostics etc.
Donc un 'Plugin' est dans Elexis simplement un programme avec des capacités au choix qui a la propriété de pouvoir coopérer avec Elexis.
Il est impossible de présenter ni dans ce guide ni ailleurs une liste exhaustive de tous les 'Plugins' parce que personne ne peut savoir quels 'Plugins' ont été commandés par des usagers indépendants auprès des programmeurs indépendants.
Un listing de tout les 'Plugins' qui nous sont connus se trouve sur : http://www.elexis.ch




\chapter{Menu et barre d'outils} 					
	% *******************************************************************************
% * Copyright (c) 2007 by Elexis
% * All rights reserved. This document and the accompanying materials
% * are made available under the terms of the Eclipse Public License v1.0
% * which accompanies this distribution, and is available at
% * http://www.eclipse.org/legal/epl-v10.html
% *
% * Contributors:
% *    G. Weirich - initial implementation
% *
% *  $Id: menu.tex 4903 2009-01-03 11:44:22Z rgw_ch $
% *******************************************************************************
% !Mode:: "TeX:UTF-8" (encoding info for WinEdt)

% Dieses Dokument enthält die Dokumentation der Menübefehle

\section{Menu}
Le menu est - comme la plupart des éléments d'Elexis- pas fixe. Les Plugins peuvent ajouter des propres commandes de menu ou des sous-menus entiers. Ce qui suit ne décrit par conséquent que le contenu des menus qui existent dans l'installation de base de Elexis.

\begin{itemize}
  \item {\textsc{fichier -- utilisateur}: S'annoncer comme autre utilisateur. Une boîte de dialogue s'ouvre dans laquelle on introduit le nom de l'utilisateur et un mot de passe.
Lorsqu'on clique sur  \glqq abandonner\grqq{}on clôture seulement la session de sorte qu'il n'y ait plus d'utilisateur branché. Il est recommandé d'installer pour chaque utilisateur son propre compte puisque Elexis lie la plupart des actions à un nom d'utilisateur, et puisque les droits d'utilisateur dépendent également de l'utilisateur connecté.}
  \item {\textsc{fichier -- mandant}: Activer un autre mandant. Dans ce cas, l'utilisateur actuel reste le même, toutefois il travaille pour un autre mandant. Cela signifie entre autres que le décompte des prestations de même que la responsabilité médicale en définitive vont sur le compte de ce mandant. Il est ainsi essentiel que dans un cabinet de groupe le mandant correct soit toujours activé. Elexis indique dans l'entête le nom de l'utilisateur actuel et le nom du Mandant actuel respectivement.}
  \item {\textsc{fichier -- connexion}: Etablir et/ou. modifier la connexion à la base de données.
Ceci n'est important que lors de l'installation du programme et peut être lu sous la rubrique concernant l'installation.}
  \item {\textsc{fichier -- Options}: Configuration centrale. La description en détail se trouve sous 'Configuration'.  (voir page \pageref{settings} et suivantes).}
  \item {\textsc{fichier -- Importation de données }: Ici, des données étrangères de différent type peuvent être importées (données de contact, données d'autres logiciels de gestion du cabinet etc.). Les options disponibles dépendent de installation des Plugins d'importation} \footnote{Il existe p.ex. un 'Plugin' d'importation pour le logiciel \textit{Aeskulap}. Un autre Plugin existe pour \textit{PraxisStar}. Informations et achat par l'intermédiaire du support (ad) elexis.ch}
  \item {\textsc{fichier -- fermer}: Fermeture du programme}
  \item {Le menue \glqq Edition\grqq{} est prévu comme dans d'autres programmes pour le presse-papiers.}
  \item {\textsc{fenêtre -- fixer perspective }: Cela sert à protéger la perspective actuelle des modifications par erreur. Des perspectives essentiels ne peuvent pas être fermés tant que se trouve un crochet devant ce point de menu.}
  \item {\textsc{fenêtre -- perspective -- enregistrer perspective }: Par ceci, vous sauvegardez l'aménagement des affichages actuels sous le même nom de perspective qu'elle a eu avant.}
  \item {\textsc{fenêtre -- perspective -- enregistrer perspective sous \ldots}:
  Par ceci, vous sauvegardez l'aménagement des affichages actuels sous un nouveau nom de perspective.}
  \item {\textsc{fenêtre -- perspective -- annuler perspective }: Reconduit la perspective actuelle à l'aménagement des affichages qu'elle avait avant la dernière sauvegarde. Peut rétrograder toutes les dernières modifications.}
  \item {\textsc{fenêtre -- perspective -- sauvegarder comme perspective de démarrage }: Déclare la perspective actuelle comme perspective de démarrage pour l'utilisateur actuel. Ainsi cette perspective apparaît après le login de l'utilisateur actuel.}
  \item {\textsc{fenêtre -- perspective -- autre }: Une boîte de dialogue, avec laquelle vous pouvez faire apparaître touts les affichages/Views existants dans le système classifiés d'après les thèmes. Vous pouvez feuilleter la liste ou introduire le nom de l'affichage recherché dans la case prévue.}
  \item {\textsc{denêtre - affichage }: Dans ce menu, on énumère d'abord quelques affichages (Views) qui font partie des perspectives standard. Cliquez sur un titre pour ouvrir l'affichage en question.}
  \item {\textsc{fenêtre - affichage - autre }: Il apparaît une boite de dialogue dans laquelle vous pouvez accéder, groupées par thème, à toutes les 'Views' existantes dans le système. Vous pouvez feuilleter la liste ou taper le nom de la 'View' dans la case prévue pour cela.}
\end{itemize}

\section{Barre d'outils}
La barre d'outils qui se trouve au-dessous du menu est également configurable par des Plugins ou par vos réglages personnels. Elle met à disposition des fonctions pour accéder aux perspectives (voir page \ref{perspektiven})
et pour imprimer des étiquettes. Si vous passez simplement avec la souris sur un bouton et si vous attendez un petit moment, la fonction du bouton en question est indiquée comme texte. 
\chapter{'Views' du système central}
	% *******************************************************************************
% * Copyright (c) 2007 by Elexis
% * All rights reserved. This document and the accompanying materials
% * are made available under the terms of the Eclipse Public License v1.0
% * which accompanies this distribution, and is available at
% * http://www.eclipse.org/legal/epl-v10.html
% *
% *  $Id: einfuehrung.tex 4904 2009-01-03 17:58:33Z rgw_ch $
%
%*******************************************************************************
% !Mode:: "TeX:UTF-8" (encoding info for WinEdt)

\section{Introduction}
Dans Elexis les 'Views' (vue) sont les éléments centraux d'affichage et de contrôle.
Une 'View' affiche un certain genre de données de certaine manière et peut permettre un traitement définie de ces données. Les 'Views' peuvent être arrangées et sauvegardées selon vos besoins et habitudes de travail à ce qu'on appelle des perspectives. On peut aussi aménager à des différents postes de travail des différentes perspectives, puisqu'à l'accueil, dans le laboratoire et dans la chambre de consultation des travaux différents sont au premier plan.


Ainsi, contrairement à d'autres logiciels, chez Elexis l'interface utilisateur n'est pas définie par le fabricant, mais par l'utilisateur.

Dans ce chapitre, sont décrits les 'Views' qui sont comprises dans le système de base de Elexis.
Une telle énumération ne peut jamais être exhaustive, puisque des nouveaux Plugins (élaborés par nous ou d'autres) peuvent apporter à tout moment leurs propres 'Views'.
Celles-ci devraient alors être décrites dans la documentation du Plugin en question.


\subsection{Ouverture et fermeture d'une 'View'}
Tous les 'Views' existantes dans le système (aussi ceux qui sont apportés par les Plugins externes) sont accessibles par le menu fenêtre - affichage. Dans ce menu se trouvent parfois quelques 'Views' qui ont été arrangées pour la perspective actuelle, mais aussi toujours un point de menu  \glqq autres\ldots\grqq{}
respectivement \glqq Other\ldots\grqq{}. Ici on trouve une liste de tous les 'Views' groupées d'après des thèmes (cf Fig. \ref{fig:viewlist}).
%\usepackage{graphics} is needed for \includegraphics
\begin{figure}[htp]
\begin{center}
  \includegraphics{images/showviewdialog}
  \caption{Liste aller Views}
  \label{fig:viewlist}
\end{center}
\end{figure}

Vous pouvez soit feuilleter cette liste, ou vous pouvez introduire en haut dans le champ de texte le nom de la 'View' cherchée. Aussitôt que vous commencez taper le nom,la liste sera filtré immédiatement selon les entrées existantes aux lettres correspondantes.

Marquez alors la 'View' de votre choix et ouvrez la soit par un double-clic soit en cliquant sur
\glqq OK\grqq{}.

\par
Pour fermer une 'View' il suffit de cliquer sur le symbole X dans l'onglet du fichier de la 'View' en question.

\subsection{Ouverture et sauvegarde d'une perspective}
\label{perspektiven} \index{perspective}
Une perspective est, comme expliqué en haut, une composition de 'Views' dotée d'un nom.
Elexis contient quelques perspectives prédéfinies qui sont accessibles par le menu démarrer respectivement par la barre d'outils. (Cf. aussi Fig. \ref{fig:toolbar}).

Une perspective a une importance spécifique en tant que  \glqq perspective de démarrage\grqq{}:
Cette perspective se présente toujours automatiquement après le login de l'utilisateur  correspondant sur le lieu de travail en question, ainsi que lorsqu'il clique sur le symbole 'Home' (ce bouton se trouve tout à gauche sur la barre d'outils). Toutes les autres perspectives (au choix de point de vue quantitative) peuvent être sauvegardées et accédées de nouveau sous un nom librement éligible. Les perspectives sont spécifiques au poste de travail. (Une perspective prête sur un lieu de travail n'est ainsi pas automatiquement disponible sur d'autres postes de travail) \footnote{Ceci doit être comme ça puisqu'il n'existent pas sur tout les postes de travail les mêmes Plugins - il n'est probablement pas indispensable que votre Plugin de comptabilité se trouve aussi sur le PC du laboratoire}.

\begin{itemize}
\index{perspective!perspective de démarrage}
\item Pour stocker l'arrangement 'View' actuel comme 'perspective de démarrage', choisissez sous  \textsc{fenêtre - perspective - sauvegarder comme perspective de démarrage}.

\item Pour sauvegarder la perspective actuelle à nouveau (p. ex. avec un arrangement ou une taille modifié de la 'View') choisissez le menu  \textsc{fenêtre - perspective - sauvegarder perspective.}.

\item Pour sauvegarder l'arrangement 'View' actuel sous un nom de perspective spécifique choisissez le menu \textsc{fenêtre - perspective - sauvegarder perspective sous …\ldots}

\item Pour restaurer la perspective actuelle (au cas où les changements apportés à la perspective ne conviennent pas ou si vous avez fermé par erreur des 'Views') choisissez  \textsc{fenêtre - perspective - restaurer}.

\item Pour revenir à la perspective de démarrage cliquez sur le symbole de la maison qui se trouve dans la barre d'outils tout à gauche.
\item Pour afficher une perspective sauvegardée choisissez 
\textsc{fenêtre - perspective - autres } et choisissez la perspective en question sur la liste. 

\end{itemize}

	% *******************************************************************************
% * Copyright (c) 2007 by Elexis
% * All rights reserved. This document and the accompanying materials
% * are made available under the terms of the Eclipse Public License v1.0
% * which accompanies this distribution, and is available at
% * http://www.eclipse.org/legal/epl-v10.html
% *
% * Contributors:
% *    G. Weirich - initial implementation
% *
% *  $Id: stammdaten.tex 4911 2009-01-05 17:56:39Z rgw_ch $
% *******************************************************************************
% !Mode:: "TeX:UTF-8" (encoding info for WinEdt)

\section{Stammdaten-Views}

\subsection{Patienten}
\index{Patientenliste}
Die Patientenliste dient sowohl der Anzeige existierender Patienteneinträge, als
auch dem Erfassen neuer Einträge. Die Liste zeigt all diejenigen Kontakte an,
die als Patient markiert sind.
\begin{figure}[ht]
	\includegraphics{images/patlistview}
	\caption{Patientenliste}
	\label{fig:patlist}
\end{figure}

Die Eingabefelder oben (Name, Vorname, Geburtsdatum) dienen dem Begrenzen der
Liste gemäss den gewünschten Parametern.
\begin{itemize}
  \item Bei Name und Vorname gilt:
	\begin{itemize}
      \item Wenn Sie mindestens zwei Buchstaben eingeben, erscheinen in der
      Liste nur noch diejenigen Einträge, die mit diesen Buchstaben \textit{beginnen}.
      \item Wenn Sie das Zeichen \% und mindestens zwei weitere Buchstaben
      eingeben, dann erscheinen in der Liste diejenigen Einträge, die diese
      Zeichenfolge \textit{enthalten}.
    \end{itemize}
  \item Beim Geburtsdatum gilt:
	\begin{itemize}
      \item Wenn Sie mindestens 3 aufeinanderfolgende Ziffern eingeben, dann wird
      die Zahl als Jahreszahl interpretiert und es werden diejenigen Patienten
      ausgewählt, die das entsprechende \textit{Geburtsjahr} haben.
      \item Wenn Sie zwei Ziffern, gefolgt von einem Punkt und ggf. weitere 2
      Ziffern eingeben, dann werden diejenigen Patienten angezeigt, die den
      entsprechenden \textit{Geburtstag} und ggf. \textit{Geburtsmonat} haben.
      Beachten Sie bitte, dass Sie Tag und Monat zweistellig eingeben müssen,
      also z.B. 04.05. und nicht etwa 4.5.
     \end{itemize}
\end{itemize}
Wenn keine Einträge existieren, die den eingegebenen Filterbedingungen
entsprechen, dann wird in der Liste angezeigt: \glqq keine Daten\grqq.

\medskip
\index{Patient Nr} \index{Patientenliste!filtern}
Sie können per Voreinstellungen die angezeigten Filterfelder beeinflussen (S. \ref{userconfig} auf S. \pageref{userconfig}).

\subsubsection{Markierungen / Sticker}
\index{Patienten!markieren}\index{Patienten!hervorheben}
Sie können die Anzeige der Patienteneinträge anhand bestimmter Kriterien beeinflussen. Diese Technik nennen wir hier 'Sticker'. Jeder Patient kann null bis mehrere Sticker haben, die dann sowohl in der Patientenliste als auch im Konsultationseintrag angezeigt werden können. Genauere Angaben dazu finden Sie unter \ref{Etiketten} auf S. \pageref{Etiketten};

\subsubsection{Toolbar}
\begin{itemize}
  \item Mit der Taste \glqq Neu\grqq{} (s. Abb. \ref{fig:patlist}) können Sie
  einen neuen Patienten erfassen. Klick
  auf diesen Knopf öffnet die Patienteneingabe-Dialogbox. Diejenigen Felder, die
  Sie bereits eingegeben haben, sind vorgegeben, die anderen können Sie soweit
  eingeben, wie sie im Moment bekannt sind. Mit Klick auf \glqq OK\grqq{}wird
  der neue Patienteneintrag angelegt. Bei Klick auf \glqq Abbrechen\grqq{}werden
  die eingegebenen Daten verworfen und es wird kein neuer Eintrag erstellt.
  Falls ein neuer Eintrag erstellt werden soll, und bereits ein Eintrag mit
  gleichen Daten existiert, dann erfolgt eine Rückfrage.

  \item Mit der Taste \glqq Filter\grqq{} (s. Abb. \ref{fig:patlist}) öffnen und Schliessen Sie eine Filterbox, in der Sie bestimmte Kriterien eingeben können, nach denen die Liste gefiltert wird.(s. Abb. \ref{fig:patlistfilter}).
	\begin{figure}[ht]
        \begin{minipage}{0.5\textwidth}
        \centering
    	\includegraphics[width=0.8\textwidth]{images/patlistfilter}
    	\caption{Filterbox öffnen}
    	\label{fig:patlistfilter}
        \end{minipage}\hfill
        \begin{minipage}{0.5\textwidth}
        \centering
        \includegraphics[width=0.8\textwidth]{images/patlistfilter2}
        \caption{Filterausdruck}\label{fig:filterexpr}
        \end{minipage}
    \end{figure}
\end{itemize}

Sie können den Filter auf verschiedene Weise beeinflussen:
\begin{itemize}
\item Nach Klick auf 'Feld...' öffnet sich eine Dialogbox wie in Abb. \ref{fig:filterexpr}. Sie können hier beliebige Datenbankfelder abfragen. Hierbei bedeutet '=': Der Ausdruck muss genau so sein, inklusive Gross/Kleinschrift. 'LIKE': Der Ausdruck muss so anfangen, Gross/Klein ist egal. 'REGEXP: Der Ausdruck soll als regulärer Ausdruck interpretiert werden. Eine Erläuterung dieses Konzepts würde hier aber zu weit führen.
\item Nach Klick auf 'Sticker' öffnet sich eine Dialogbox, die alle im System definierten Sticker enthält. Sie können ein oder durch mehrfachen Aufruf auch mehrere auswählen.
\item Sie können ein Script aus der Script-View (s. \ref{Script} auf S. \pageref{Script} ins Filterfeld hineinziehen, welches beliebige Bedingungen errechnen kann.
\end{itemize}

Die Filterbedingungen werden strikt von oben nach unten abgearbeitet. Also der als zweites eingegebene Filterausdruck wird überhaupt nur dann ausgewertet, wenn der erste passiert wurde. Es ist deswegen sinnvoll, weniger rechenintensive Filter (z.B. Sticker) oben und rechenintensivere (z.B. Scripts) weiter unten einzusetzen.
Klicken Sie dann auf einen der Spaltenköpfe oder auf das 'x', um die Liste neu - diesmal beeinflusst durch den Filter- einzulesen.

\medskip

Um eine Filterbedingung wieder zu entfernen können Sie sie rechts anklicken  und 'entfernen' klicken. Um alle zu entfernen, klicken Sie auf 'leeren'. Um den Filter nur temporär ausziuschalten, ohne ohn zu leeren, schliessen Sie ihn durch Klick auf den Filter-Knopf.


\subsubsection{Kontextmenü}
Das Kontextmenü erscheint, wenn Sie auf einem Pa\-tien\-ten\-eintrag mit der rech\-ten
Maus\-taste klicken. Es enthält fol\-gende Ein\-träge:
\begin{itemize}
    \item Sticker... Sie können dem Patienten so einen Sticker (S. \ref{Etiketten}) zuweisen oder entfernen.
  \item Patient löschen (s. oben)\footnote{Sie benötigen dazu das Recht \textsc{Löschen/Kontakt} (Vgl. \ref{sec:gruppen})}
  \item KG exportieren. Falls ein Export-Plugin installiert ist, wird die KG des
  aktuell markierten Patienten über dieses Plugin exportiert. Falls mehrere
  Export-Plugins definiert sind, erscheint zunächst eine Dialogbox, mit der sie
  das ge\-wünschte Ziel bzw. Format auswählen können.\footnote{Sie benötigen dazu das Recht \textsc{Daten/Kontakt/exportieren}}
\end{itemize}


\subsection{Patient-Detail}
\index{Patient!Detailangaben}
Diese View (Abb. \ref{fig:patdetail} zeigt Details des momentan ausgewählten
Patienten resp. der momentan ausgewählten Patientin an
 %\usepackage{graphics} is needed for \includegraphics

\begin{figure}[t]
\centering
  \includegraphics[width=0.8\textwidth]{images/patdetail}
  \caption{Patient Detailansicht}
  \label{fig:patdetail}
  \hfill

\end{figure}

Sie sehen, dass die einzelnen Einträge grau erscheinen und nicht geändert werden können . Wenn Sie auf das Schloss-Symbol rechts oben klicken, können Sie die View entsperren (sofern Sie die entsprechenden Rechte besitzen). Dann können
alle Felder können durch einfaches Überschreiben geändert werden. Eine
Änderung wird in dem Moment gespeichert, in dem ein Feld wieder verlassen wird.
(Explizites Speichern ist in Elexis nie notwendig). Bei erneutem Klick auf das Schloss, oder bei Auswahl eines anderen Patienten, wird die Sicherung gegen versehentliches Überschreiben wieder eingeschaltet.

\medskip

Die Felder im oberen Block sind alle einzeilige Textfelder und können direkt geändert
werden, bis auf das Feld \glqq Konto\grqq{}, welches nicht direkt beschreibbar
ist. Dieses Feld stellt den Saldo aller Forderungen an diesen bzw. Zahlungen von
diesem Patienten dar. Wenn der aktuell eingeloggte Anwender Verrechnungs-Rechte
besitzt, kann er den blauen Text \glqq Konto\grqq{} anklicken, dann öffnet sich
ein Dialog, in dem einzelne Buchungen eingegeben werden können.

\textbf{Achtung}: Normalerweise erfolgen Buchungen automatisch durch Erstellen
von Rechnungen und Einlesen von ESR-Files. Manuelle Buchungen können zu
Inkonsistenzen in der Buchhaltung führen. Führen Sie also nur dann manuelle
Buchungen durch, wenn Sie sich über die Konsequenzen exakt bewusst sind.

Das Feld \glqq Anschrift\grqq{} zeigt die Postanschrift\footnote{Postanschrift ist das, was auf Briefumschlägen und Adressetiketten erscheint, muss also nicht unbedingt den eingegebenen Adressdaten entsprechen. Beispielsweise könnte hier noch c/o oder ein Postfach oder eine Kontaktperson stehen.} des Patienten an. Diese
kann durch Klick auf den blauen Text \glqq Anschrift \grqq{}geändert werden.

Die darunterstehenden Felder sind alle aufklappbar: Standardmässig ist nur der
Titel sichtbar, durch Klick darauf öffnet sich das Feld.
\begin{itemize}
  \item Das Feld \glqq Zusatzadressen\grqq{}dient dazu, Kontakte, die in
  irgendeiner Beziehung zum Patienten stehen, zu erfassen. Beispielsweise
  Angehörige, Ämter, weitere Ärzte etc. Klick auf \glqq Hinzu\grqq{} öffnet eine
  Kontaktauswahl-Box, aus der die gewünschte Person oder Organisation ausgewählt
  werden kann. Danach erscheint eine Eingabebox, in der die Beziehung des eben
  ausgewählten Kontakts zum Patienten beschrieben werden kann. \\
  Mit Rechtsklick auf einen Eintrag in diesem Feld öffnet sich ein Kontextmenü,
  mit dem man den vollständigen Eintrag anzeigen, oder den Eintrag entfernen kann.
  \item Die Felder \glqq Diagnose\grqq, \glqq Persönliche Anamnese\grqq{},
  \glqq Allergien\grqq{}, \glqq Risiken\grqq{} und \glqq Bemerkungen \grqq{}
  können direkt beschrieben werden und werden wie gewohnt sofort beim Verlassen
  gespeichert.
  \item Das Feld \glqq Fixmedikation\grqq{}entspricht der View Fixmedikation.

\end{itemize}
\subsection{Kontakte}
Diese View (Abb. \ref{fig:kontaktlist}) zeigt eine Liste aller in Elexis
vorhandenen Kontakte an. Ein Kontakt
ist jede Person oder jede Organisation, welche in irgendeiner Beziehung zu
unserer Praxis steht. Das sind beispielsweise Patienten, Kollegen, Spitäler,
Versicherungen, Labors, Lieferanten usw.
%\usepackage{graphics} is needed for \includegraphics
\begin{figure}[htp]
\begin{center}
  \includegraphics[width=0.9\textwidth]{images/kontaktlistview}
  \caption{Kontaktliste-View}
  \label{fig:kontaktlist}
\end{center}
\end{figure}
Mit Klick auf das Briefumschlag-Symbol rechts oben können Sie eine
Adressetikette für den betreffenden Kontakt ausdrucken.

\subsection{Kontakt-Detail}
Hier werden die Details zum aktuell ausgewählten Kontakt angezeigt und können
geändert werden (Abb. \ref{fig:kontaktdetail}).
%\usepackage{graphics} is needed for \includegraphics
\begin{figure}[htp]
\begin{center}
  \includegraphics[width=0.9\textwidth]{images/kontaktdetail}
  \caption{Kontakt Detailview}
  \label{fig:kontaktdetail}
\end{center}
\end{figure}
In den Checkboxen der obersten Zeile können Sie den Typ des betreffenden
Kontakts festlegen. Beachten Sie, dass ein Kontakt auch mehrere Typen haben kann
(Beispielsweise kann jemand Anwender und auch Patient sein). Hingegen kann
ein Kontakt natürlich nur entweder eine Organisation oder aber eine Person sein.
Achten Sie darauf, dies und bei Personen auch das Geschlecht (m oder w) korrekt
zu erfassen, da Textformatvorlagen diese Informationen auswerten, um die
korrekten Formulierungen auszuwählen.

\medskip

Das Feld 'Kürzel/ID' enthält bei Patienten die Patientennummer und sollte daher nicht geändert werden. Bei anderen Kontakten kann es sinnvoll sein, hier ein Kürzel zum schnellen auffinden einzusetzen. Beispielsweise könnten Ärzte mit der präfix 'az', gefolgt von der Spezialität gefolgt von den Initialen eingetragen werden, Krankenkassen mit der Präfix 'kk' etc. So wäre der Internist Dr. Erst Meier beispielsweise als azInnEM gespeichert, die Swica Schaffhausen als kkSwicaSH. Die EInträge müssen nicht eindeutig sein und man muss das auch nicht so zu machen; es hilft einem lediglich z.B. beim Briefe schreiben rascher den Adressaten zu finden etc.

\medskip

Das Feld 'Externe ID' dient dazu eine im Prinzip beliebig grosse Anzahl von extern vergebenen Identifikationen (XID) festzuhalten. Durch Klick auf das blaue 'Externe ID' öffnet sich ein Dialog in dem alle eingetragenen XID's dargestellt werden. Im Kontaktdetail angezeigt wird immer der 'beste' d.h. 'eindeutigste' der zur Verfügung stehenden Identifikatoren. Beispiele für XID's sind etwa EAN, BAG-Nummer, Sozialversicherungsnummer/AHV-Nummer etc.

\medskip

In der untersten Zeile steht die Postanschrift des betreffenden Kontakts. Dies
ist die Adresse, wie sie beispielsweise im Adressfeld von Briefen oder
Rechnungen oder auf Adressetiketten erscheinen soll. Mit Klick auf das blaue
Wort \glqq Anschrift\grqq{}öffnet sich die Anschrifteingabe-Dialogbox (Abb.
\ref{fig:anschrift}), wo Sie beliebigen Text eingeben können. (Klick auf den
Button \glqq Postanschrift\grqq{} erstellt eine Standard-Anschrift aus den
vorhandenen Adressangaben)


\begin{figure}[htp]
\begin{center}
  \includegraphics{images/anschrifteingabe}
  \caption{Anschrift-Eingabe}
  \label{fig:anschrift}
\end{center}
\end{figure}
\bigskip
\pagebreak[3]
\subsection{Artikel}
\label{view:artikel}
\index{Artikel}
Ein \glqq Artikel\grqq{}ist jedes Objekt, das auf Lager genommen und/oder
abgegeben werden kann. Es gibt einerseits vordefinierte Artikel (z.B. die Liste
aller zugelassenen Medikamente), andererseits auch Eigenartikel. Elexis kann den
Lagerbestand von Lagerartikeln verwalten und halbautomatisch Bestellungen zur Neige
gehender Artikel vornehmen.


\subsection{Artikelliste}
In Abb. \ref{fig:artikel} ist eine Artikelauswahl-Liste und die
Artikeldetaildarstellung nebeneinander zu sehen.

 %\usepackage{graphics} is needed for \includegraphics
\begin{figure}[htp]
\begin{center}
  \includegraphics[width=0.9\textwidth]{images/artikelview}
  \caption{Artikel-View}
  \label{fig:artikel}
\end{center}
\end{figure}
Die Liste links können Sie in gewohnter Weise filtern, indem Sie einige
Buchstaben des gewünschten Artikelnamens eingeben.
In der Detailansicht sehen Sie Einzelheiten zum gerade ausgewählten Artikel. (In
Perspektiven, wo die Liste allein dargestellt ist, können Sie mit der rechten
Maustaste und \glqq Bearbeiten\grqq{} zur Detailansicht gelangen.)

\index{Lager}
Ein Artikel wird dadurch zum Lagerartikel, dass Sie ihm einen Mindestbestand
grösser als Null zuweisen. Geben Sie ausserdem einen Höchstbestand höher als der
Mindestbestand ein und weisen Sie dem Feld \glqq Istbestand\grqq{} den korrekten
Wert zu. Elexis wird bei einer halbautomatischen Bestellung von jedem Artikel,
dessen Istbestand unter dem Mindestbestand ist, soviele Exemplare bestellen, um
auf den Höchstbestand zu kommen.

Bei manchen Artikeltypen wird üblicherweise nicht eine ganze Verpackungseinheit
auf einmal abgegeben, beispielsweise Ampullen. Hierfür sind die Felder \glqq
Stück pro Packung\grqq{}und \glqq Stück pro Abgabe\grqq{}vorgesehen. Angenommen
ein Artikel wird in Packungen zu 10 Stück eingekauft, aber einzeln abgegeben.
In diesem Fall können Sie bei Stück pro Abgabe eine 1 setzen, bei Stück pro
Packung eine 10. Wenn dieser Artikel dann einem Patienten verrechnet wird, dann
wird automatisch 1/10 des Verpackungs-Verkaufspreises berechnet und auch nur
1/10 einer Packung aus dem Lager ausgebucht.

Die Angabe \glqq Aktuell an Lager\grqq{} meint dann die Zahl der einzelnen
Artikel, während \glqq Aktuell Pck. an Lager\grqq{} für die Zahl der
unangebrochenen Packungen steht.


\subsection{Lager und Bestellung}
\index{Artikel!bestellen}
Wie oben beschrieben, kann Elexis Ihr Warenlager halbautomatisch bewirtschaften.
Wann immer Sie einem Patienten einen Artikel verrechnen, wird dieser Artikel
automatisch aus dem Lagerbestand ausgebucht. Sobald der Bestand eines
Lagerartikels unter den von Ihnen definierten Mindestbestand fällt, \glqq
weiss\grqq{} Elexis, dass dieser Artikel nachbestellt werden muss. Nebst dieser
automatischen Erkennung können Sie selbstverständlich Bestellungen auch manuell
erstellen und/oder ändern.

Diese Funktionen sind in der View \glqq Bestellung\grqq{} erreichbar (s. fig.
\ref{fig:bestellungen}).
 %\usepackage{graphics} is needed for \includegraphics
\begin{figure}[htp]
\begin{center}
  \includegraphics[width=0.9\textwidth]{images/bestell1}
  \caption{Bestellungen - View}
  \label{fig:bestellungen}
\end{center}
\end{figure}


Links finden Sie die schon bekannten Artikelauswahlfenster für alle
Artikelkategorien, für die Sie Plugins haben (Normalerweise Medikamente,
Medicals, MiGeL und Eigenartikel). Rechts ist das Feld Bestellung, welches
anfangs leer ist. Sie haben nun folgende Möglichkeiten:
\begin{itemize}
  \item Mit Klick auf das Zauberstab-Symbol werden automatisch diejenigen
  Artikel der Bestellung zugefügt, von welchen weniger als der Mindestbestand an
  Lager ist. Es werden jeweils soviele bestellt, dass der für diesen Artikel
  definierte Höchstbestand erreicht wird.
  \item Sie können aus einem der Fenster links Artikel in die Bestellung
  herüberziehen.
	\item Sie können auf einen der Artikel in der Bestelliste mit der rechten
	Maustase klicken, und den Artikel aus der Liste entfernen oder die Zahl ändern.
	
	\item Sie können die Bestellung erst mal abspeichern und später weiterbearbeiten.
	
	\item Sie können eine früüher gespeicherte Bestellung wieder laden.
	\item Sie können die Bestellung ausdrucken. Dafür ist eine System-Textvorlage (s. S.
	\pageref{textvorlagen}) namens \glqq Bestellung\grqq{} notwendig, welche an
	einer Stelle den Platzhalter [Bestellung] enthält (s. Abb. \ref{fig:bestell2}).
	\item Last but not least können Sie, falls Sie ein entsprechendes Plugin für
	Ihren Lieferanten haben, die Bestellung direkt via Internet oder Modem
	absenden. Ein entsprechendes Plugin für Galexis ist bereits verfügbar, weitere
	werden entwickelt.
\end{itemize}
\begin{figure}[hb]
  % Requires \usepackage{graphicx}
  \includegraphics{images/bestell2}\\
  \caption{Ausschnitt aus der Vorlage Bestellung}\label{fig:bestell2}
\end{figure}

%\subsection{codes}
%Codes




	% *******************************************************************************
% * Copyright (c) 2007 by Elexis
% * All rights reserved. This document and the accompanying materials
% * are made available under the terms of the Eclipse Public License v1.0
% * which accompanies this distribution, and is available at
% * http://www.eclipse.org/legal/epl-v10.html
% *
% * Contributors:
% *    G. Weirich - initial implementation
% *
% *  $Id: konsviews.tex 4902 2009-01-03 10:47:34Z rgw_ch $
% *******************************************************************************
% !Mode:: "TeX:UTF-8" (encoding info for WinEdt)


\section{'Views' en rapport avec les consultations}



\subsection{Cas}
Cette view (Fig. \ref{fig:faelle2} létabli une liste de tout les cas du patient actuellement sélectionné. \index{Liste des cas}
\begin{wrapfigure}{l}{6.8cm}
  \includegraphics{images/faelleview}
  \caption{Fälle - View}
  \label{fig:faelle2}
\end{wrapfigure}

Le symbole qui se trouve à gauche de la désignation du cas indique si toutes les données nécessaires ont été rassemblées pour que la facturation puisse être faite : S'il est vert, l'établissement de facture devrait être possible, s'il est rouge, manque encore une ou plusieurs données.


\textit{Quelles} indications minimales sont nécessaires, dépend du système de facturation. Ainsi, pour les cas qui sont facturés selon la LAMal,l'indication d'un destinataire de la facture, d'un 	assureur et du numéro d'assurance est nécessaire. Les cas qui sont comptabilisés selon la LAA nécessitent un numéro de cas. Pour des factures en privé un destinataire de la facture devra au moins être indiqué.

\medskip

\index{filtrer!consultations}
\label{filter:fall}
Un clic sur le symbole de filtre dans l'entête de la 'View' a pour conséquence qu'il n'y a plus que les consultations dans la liste de consultation (cf \ref{view:konsultationen}) qui font partie du cas choisi actuellement. Si on choisi un autre cas, la liste est de nouveau filtrée. Cliquez encore une fois sur le symbole de filtre pour éteindre le filtre.

Le clic droit sur un cas ouvre son menu de contexte. Celui-ci contient les points
suivants :

\begin{itemize}
  \item {Supprimer un cas}. Ceci n'est possible, que si vous avez les droits nécessaires, et si plus aucune consultation n'existe .
  \item {Modifier un cas }. Ceci ouvre une autre 'View', dans laquelle les détails pour le cas actuellement séléctionné peuvent être introduits.
  \item {Réouverture du cas}. Ceci permet de réactiver un cas déjà fermé. \footnote{Un cas est fermé si une date de fin a été introduite. On ne peut plus ajouter une consultation à un cas fermé.}
  \item {Créer une facture}. Par ceci, une facture peut être produite qui concerne toutes les consultations non comptabilisées du cas actuel et du mandant actuel. Il s'agit d'un \glqq raccourci\grqq{} du procédé normale de l'établissement de facture qui convient surtout pour l'établissement immédiat de différentes consultations ou prestations.
\end{itemize}


\clubpenalty=5000

\subsection{Cas et consultations}
 %\usepackage{graphics} is needed for \includegraphics
\begin{wrapfigure}{l}{7cm}
  \includegraphics[width=7cm]{images/fallkonsview}
  \caption{Fälle und Kons}
  \label{fig:fallkons}
\end{wrapfigure}
Cette View (Fig. \ref{fig:fallkons} montre une liste synoptique des cas et des consultations correspondantes (seulement titres sans textes). Si on choisi dans le secteur supérieur un cas en cliquant dessus, les consultations correspondantes de ce cas sont indiquées dans le secteur inférieur.
Si on clique sur une consultation elle sera affichée dans la 'View' - consultation.
(cf page \ref{konsview} s. \pageref{konsview})

Pour établir un nouveau cas introduisez un titre pour ce cas et cliquez sur \textit{nouveau cas}. Pour une nouvelle consultation choisissez le cas concerné et cliquez sur  \textit{nouvelle consultation}

\medskip

Remarque : Vous avez constaté que cette 'View' et la View 'Cas' traité antérieurement sont jusqu'à un certain point redondantes. C'est ainsi. Vous pouvez préférer des 'Views' séparés pour les 'Cas' et les 'Consultations' ou favoriser une seule 'View' qui contient les deux. En général, vous n'appliquerez pas les deux concepts en même temps, mais celui qui vous convient mieux - Elexis vous laisse le choix.


\clearpage

\subsection{Historique des Consultations}
\label{view:konsultationen}
\begin{wrapfigure}{L}{7.5cm}
  \includegraphics{images/konslisteview}
  \caption{Konsultationsliste}
  \label{fig:konslisteview}
\end{wrapfigure}

Ceci est une énumération de toutes les consultations précédentes du patient actuellement sélectionné, indépendamment du cas respectif.
\index{liste des consultations} Pour chaque consultation le texte est affiché sans formatages. (cf Fig.\ref{fig:konslisteview}).\\
En cliquant sur le titre (bleu) d'une consultation vous choisissez cette consultation dans la 'View Consultation' (cf page \pageref{konsview}).

En cliquant sur le symbole du filtre à droite en haut vous ouvrez la fenêtre du dialogue du filtre (cf Fig. \ref{fig:konsfilter}).
\index{Filtre consultations} C'est ici que vous pouvez introduire les critères selon lesquels les consultations devraient être filtrées avant d'être affichées (que les consultations qui correspondent aux conditions du filtre soient affichées).
Dans le champ supérieur vous pouvez indiquer si seulement des consultations d'un certain cas ou si tous les cas doivent être affichés. Dans le champ  inférieur, vous pouvez suggérer les critères de recherche qui doivent exister dans le texte de la consultation. Plusieurs termes de recherche peuvent ainsi être liés avec AND, OR, NOT,AND NOT et OR NOT.

Par exemple si on introduit \glqq Lorem AND NOT ipsum\grqq{} on ne trouve que les consultations dont le texte contient  \glqq Lorem\grqq, mais pas \glqq ipsum\grqq{}.

Tout en bas vous pouvez enfin encore indiquer si l'écriture majuscule/minuscule doit être considérée, ou si des critères de recherche doivent être considérés comme des termes fixes. Une explication précise de ce thème irait ici trop loin ; à ce sujet vous pouvez trouver beaucoup de littérature en utilisant les mots de recherche \glqq Regular
Expression\grqq{}ou \glqq Pattern Matching\grqq{}. Cette technique permet de décrire le critère de recherche avec différents caractères de remplacement. Ainsi permet p. ex.\glqq M[ae][iy]e?r\grqq{} de chercher tous les Meiers, Mayrs etc. donc toutes les formes d'écritures.

\medskip

Remarque : 
Filtrer l'historique par cette procédure peut durer quelques secondes, puisque le texte de chaque consultation doit être fouillé complètement. Si on veut filtrer seulement d'après des cas ou des problèmes, le filtre par cas correspondant  (cf \pageref{filter:fall}) ou le filtre de liste de problème  (cf \pageref{filter:problemliste})est en général plus efficace.

\begin{figure}[ht]
\begin{center}
  \includegraphics{images/filterdialog}
  \caption{Filterdialog}
  \label{fig:konsfilter}
\end{center}
\end{figure}


\subsection{Consultation}
 \label{konsview}
Aperçu détaillé d'une saisie de consultation(cf Fig. \ref{fig:konsdetail}).
\begin{figure}[ht]
  \includegraphics{images/konsview}
  \caption{Konsultation: Detail}
  \label{fig:konsdetail}
\end{figure}

Vous trouvez dans la zone de texte les possibilités supplémentaires suivantes :
\begin{description}
\item[Makros]
Ecrivez un texte quelconque, marquez-le avec la touche gauche de la souris, cliquez ensuite avec la touche droite de la souris et choisissez  'comme macro…' . Donnez un nom arbitraire à la macro. Si vous tapez à l'avenir le nom de la macro suivi d'un \# le texte prédéfini de la macro est introduit dans le texte.

\item[Introduire des prestations]
Si vous tapez le nom d'un bloc de prestation suivi d'un \#, ce bloc est comptabilisé comme si vous l'aviez tiré avec la souris dans au champ de facturation.


\item[Commandes du texte ]
Il est possible d'introduire quelques commandes simples du texte:
Un mot au début d'une ligne qui est suivi d'un deux-points se présente en caractères gras, la même chose est le cas pour mot entre deux  *. Un mot entre deux  / est écrit en italique.
\end{description}


\subsection{Certificat d'incapacité de travail}
Cette 'View' sert à fixer une incapacité de travail . (Fig. \ref{fig:auf})
\index{Certificat} \index{Certificat d'incapacité de travail}.
\begin{figure}
  \includegraphics{images/aufview}
  \caption{AUF-View}
  \label{fig:auf}
\end{figure}
Une incapacité de travail se réfère toujours à un cas spécifique. Si aucun cas est marqué, vous serez d'abord invité d'en marquer un.

Si vous cliquez sur le symbole  \glqq nouveau\grqq (bouton vert avec un plus blanc), il apparaît une fenêtre dans laquelle vous pouvez fixer le début et la fin de l'arrêt de travail de même que le pourcentage de l'incapacité.
En cliquant sur le symbole de \glqq l'imprimante \grqq, une 'View'-Texte s'ouvre où vous pouvez effectuer encore manuellement des adaptations du texte du certificat avant de l'imprimer ou de le faxer.

\subsection{Ordonnances}
Dans cette 'View' les ordonnances seront enregistrées.  \index{ordonnance} Cliquez sur le symbole\glqq nouveau\grqq
(bouton vert avec un plus vert) pour créer une nouvelle ordonnance avec la date actuelle. Tirez les articles (médicaments) par 'Drag and Drop' d'une liste d'article ou de la 'View de médication à long terme' dans cette ordonnance. En cliquant sur le symbole \glqq imprimante\grqq vous ouvrez une 'texte-View' dans laquelle vous pouvez encore faire des modifications manuelles avant d'envoyer l'ordonnance définitivement vers l'imprimante ou un appareil de télécopie ou vers un connecteur d'exportation. Pour tout cela un modèle avec le nom  \glqq Ordonnance\grqq
doit avoir été crée qui contient un espace réservé [lignes de prescription] dans lequel les articles choisis sont insérés.


\subsection{Détails du cas}
\label{falldetail}
\index{Détails du cas}
 Cette 'View' (Fig. \ref{fig:falldetail}) sert à ajuster les détails d'un cas (une boîte de dialogue avec la même View est ouverte, s'il faut ouvrir un nouveau cas( cf \ref{definition:fall} page \pageref{definition:fall})).
\begin{figure}[ht]
  % Requires \usepackage{graphicx}
  \includegraphics[width=0.8\textwidth]{images/falldetail}\\
  \caption{Fall-Detail}\label{fig:falldetail}
\end{figure}
Indiquez dans la boîte de choix en haut quel système de facturation doit être appliqué pour ce cas (cf aussi \ref{settings:abrechnungssystem} page \pageref{settings:abrechnungssystem}). En dessous il y a un espace où vous pouvez choisir librement une désignation pour le cas. Celui-ci ne sert qu'à vos propres informations, afin que
 vous puissiez mieux distinguer les différents cas du même patient.
La prochaine ligne, 'la raison pour l'assurance' est une indication qui apparaîtra sur les factures qui concernent le cas (si jamais le modèle de facturation contient un champ spécifique pour cela).

La\textbf{date de départ } est généralement la date de la première consultation, ou en cas d'accident, la date de l'accident. La  \textbf{date finale } désigne la date quand le cas est terminé. Un cas qui a une date de fin, est marqué comme 'cas terminé' dans la liste des cas  (Fig. \ref{fig:faelle2}) . Le cas terminé ne permet plus de ajouter une consultation de plus. 
Généralement, un cas ne devrait être terminé que s'il s'agit d'un accident qui est terminé, ou si le patient change d'assureur et si les données de facturation changent pour cette raison.

La prochaine ligne, destinataires de la facture, est impérative, afin qu'une facture puisse effectivement être fournie. Cela doit être un contact déjà existant (p. ex. le patient lui-même).

\medskip
Toutes les autres lignes seront selon le choix du système de facturation différentes. Souvent il y existera aussi une ligne 'répondant des coûts' \footnote{Remarque importante : Pour le \textbf{système Tarmed } (Suisse): Si le destinataire de la facture et le répondant des coûts est identique, une facture Tiers-Payant est fourni, autrement une facture Tiers-garant . Veillez ainsi à ce que ces deux lignes soient correctes (Pour des cas LAA les assureurs d'accident doivent être les destinataires de la facture et en même temps répondants des coûts tandis qu'avec les cas LAMAL le patient est destinataires de la facture dans les cantons Tiers garants et la caisse de maladie est répondant des coûts.)}.

\clearpage

\subsection{Diagnostics}
\label{view:diagnosen} \index{Diagnostics}
Cette 'View' (Fig. \ref{fig:diagnosen}) sert à choisir des diagnostics et de les assigner aux consultations respectives. \begin{wrapfigure}{l}{7cm}
    \includegraphics[width=6.5cm]{images/diagnosenview}
    \label{fig:diagnosen}
    \caption{Diagnosen-Auswahl}
\end{wrapfigure}
Vous voyez vers le bas une série d'onglets qui correspondent aux Plugins de code de diagnostic installés. (comme standard : Code tessinois, CIM-10, et CISP-2). Pour choisir un diagnostic vous choisissez d'abord entre ces onglets le système de codification correspondant et ensuite le code.
Le choix peut avoir lieu par 'Drag and Drop' ou par double-clic .
Vous voyez pour chaque système de codification, une fenêtre divisée en trois : Dans le secteur supérieur se trouvent vos diagnostics les plus fréquemment utilisés (c.-à-d. de l'utilisateur actuellement connecté) ; dans la partie moyenne se trouvent les diagnostics qui ont jusqu'ici le plus fréquemment été utilisés pour le patient en question et dans la partie inférieure se trouve la systématique entière du système de codification choisi.



\medskip

De cette façon vous avez toujours accès aux codes des diagnostics les plus fréquemment utilisés et vous n'auriez que rarement à fouiller la systématique entière.


	% *******************************************************************************
% * Copyright (c) 2007 by Elexis
% * All rights reserved. This document and the accompanying materials
% * are made available under the terms of the Eclipse Public License v1.0
% * which accompanies this distribution, and is available at
% * http://www.eclipse.org/legal/epl-v10.html
% *
% * Contributors:
% *    G. Weirich - initial implementation
% *
% *  $Id: laborview.tex 4902 2009-01-03 10:47:34Z rgw_ch $
% *******************************************************************************
% !Mode:: "TeX:UTF-8" (encoding info for WinEdt)

\section{Laboranzeige-View}
\index{Labor!Eingabe}
\index{Labor!Anzeige}
Bei Elexis werden sowohl interne als auch externe Laborbefunde, sowohl
automatisch eingelesene als auch manuell eigegebene Befunde in derselben Sicht
angezeigt.

Die  Anzeige eines Befundes wird bestimmt durch
\begin{itemize}
  \item Ein Laboritem, zu dem dieser Befund gehört
  \item Ein Datum, an dem dieser Befund erhoben wurde
  \item Einen Patienten, zu dem dieser Befund gehört
\end{itemize}

Das Laboritem definiert, wie und wo der Laborbefund angezeigt werden soll, und
zu welchem Typ von Laborwerten er gehört. Das Erstellen von Labritems ist in der Regel nur bei der Installation des
Programms notwendig, bzw. dann, wenn Sie neue Laborparameter in Ihre
Standardbesimmungen aufnehmen möchten. Das genaue Vorgehen ist unter Konfiguration (S. \pageref{config:labor} genauer beschrieben.

 %\usepackage{graphics} is needed for \includegraphics
\begin{figure}[htp]
\begin{center}
  \includegraphics{images/labview}
  \caption{Labor-Anzeige}
  \label{fig:labview}
\end{center}
\end{figure}

\subsection{Manuelle Eingabe}
Um Laborwerte manuell einzutragen, gehen Sie so vor:
\begin{itemize}
    \item Wenn für das gewünschte Datum noch keine Spalte exstiert, klicken Sie auf das grüne Pluszeichen rechts oben, um ein Datum anzugeben.
    \item klicken Sie auf die Zeile und Spalte, wo Sie einen Laborwert eingeben möchten. Tippen Sie den Wert ein und verlassen Sie das Feld mit der Eingabetaste, oder der Pfeil-nach-unten-Taste.
\end{itemize}
Wenn ein Laborparameter numerisch ist, und der eingegebene Wert ausserhalb des Referenzbereichs ist, wird der Wert in rot angezeigt. Sie können diese Anzeige auch manuell ein- und ausschalten, indem Sie den Wert mit der rechten Maustaste anklicken und das Häkchen vor \glqq pathologisch\grqq{} setzen oder löschen.

\subsection{Automatisches Einlesen}
\index{Labor!automatisches Einlesen}
Elexis kann Laborwerte selbstverständlich auch automatisch einlesen. Hierfür dient das View-Menu rechts oben:\\
\begin{wrapfigure}{r}{7cm}
    \includegraphics{images/labor6}
\end{wrapfigure}
Klicken Sie auf \textit{import} und wählen Sie in der dann erscheinenden Dialogbox die Quelle für die einzulesenden Laborwerte aus. Was für Quellen hier angeboten werden, hängt von den vorhandenen Laborimport-Plugins ab. In Frage kommen Laborgeräte und verschiedene externe Labors. Eine aktuelle Liste aller vorhandenen Laborimport-Plugins finden Sie auf http://www.elexis.ch

\subsection{Laborblatt drucken}
Um ein Laborblatt auszudrucken, klicken Sie auf das Drucker-Symbol rechts oben. Dies erstellt eine Tabelle innerhalb einer System-Textvorlage \glqq Laborblatt\grqq{}, welche einen Platzhalter [Laborwerte] enthalten muss (s. auch \ref{textvorlagen}).

\section{Labor Neu}
Diese View dient dazu, alle Laborwerte anzuzeigen, welche noch nicht als \glqq gesehen\grqq{} markiert worden sind, und erlaubt es auch gleich, sie als gesehen zu markieren (S. Abb. \ref{fig:labneu}).

\begin{figure}
  % Requires \usepackage{graphicx}
  \includegraphics{images/labneu1}\\
  \caption{Anzeige neuer Laborwerte}\label{fig:labneu}
\end{figure}

Werte ausserhalb des Referenzbereichs werden rot dargestellt. Man kann gesehene Werte mit einem Häkchen in der Checkbox links markieren. Nach einiger Zeit werden diese dann aus der Liste entfernt (Solange sie noch nicht entfernt sind, kann man die Markierung mit einem erneuten Klick wieder löschen). Werte die älter als 96 Stunden sind, werden automatisch aus der Liste entfernt.

Wenn man einen Eintrag markiert wird das Laborblatt des betreffenden Patienten aktiviert (sofern eine Labor-View geöffnet ist). Man kann entweder einen einzelnen Wert oder alle Resultate des aktuell markierten Patienten, oder alle angezeigten Resultate als gelesen markieren.

Um die Werte einzeln oder gesamthaft als gesehen zu markieren, ist das Recht \textit{Daten/Patient/Labor abhaken} erforderlich (s. \ref{sec:gruppen}, S. \pageref{sec:gruppen}ff.)

	% *******************************************************************************
% * Copyright (c) 2007 by Elexis
% * All rights reserved. This document and the accompanying materials
% * are made available under the terms of the Eclipse Public License v1.0
% * which accompanies this distribution, and is available at
% * http://www.eclipse.org/legal/epl-v10.html
% *
% *  $Id: abrechnung.tex 4911 2009-01-05 17:56:39Z rgw_ch $
%
%*******************************************************************************
% !Mode:: "TeX:UTF-8" (encoding info for WinEdt)

\section{'Views' en relation avec la facturation}
\subsection{Consultations selon date}
\begin{wrapfigure}{l}{7.3cm}
\includegraphics[width=7cm]{images/heute}
\caption{Konsultation nach Datum}
\label{fig:heute}
\end{wrapfigure}

Cette View (Fig. \ref{fig:heute}) sert à afficher les consultations d'une certaine période (normalement celles du jour actuel). Elle donne un aperçu de facturation et du temps calculé pour chaque consultation en particulier et aussi du total. En cliquant sur la 'check-box', vous pouvez laisser calculer des consultations ouvertes \footnote{il s'agit de celles qui n'ont pas encore été facturées} ou clôturées ou les deux. Vous pouvez indiquer dans les champs de date le début et la fin de la période en question. Après chaque modification vous devez utiliser le bouton 'actualiser liste' pour que la liste sera calculée à nouveau.

\medskip

Dans la section inférieure de la 'View' vous apercevez le nombre total des consultations au cours de la période choisie, ainsi que (défini par le système de codage de prestation) le temps et le montant comptabilisé. Dans le champ dessous vous voyez les mêmes indications pour la consultation actuellement marquée en bleu.

Vous pouvez donc utiliser cette 'View' aussi pour pouvoir passer revu le soir
toutes les consultations de la journée et pour introduire les prestations ou pour les corriger.


\medskip

En outre, cette 'View' permet des fonctions statistiques simples :
%\begin{itemize}
Si vous cliquez sur le bouton 'filtre', une boîte de filtre s'ouvre dans la partie supérieure de la 'View'. Depuis une fenêtre de facturation vous pouvez tirer les positions que vous voulez prendre en compte vers cette boite. Lors de la prochaine actualisation de la liste, la 'View' ne calculera que des consultations dans lesquelles apparaît au moins un des codes de positions souhaités. Les codes et les sommes totales seront alors énumérés séparément (lors de l'impression, voir ci-dessous).

\medskip

Dans 'view-menu' vous trouvez l'option 'imprimer liste'. En cliquant dessus, une fenêtre s'ouvre contenant un tableau qui énumère et fait imprimer les consultations indiquées.

\medskip

Pour des statistiques plus détaillées vous pouvez choisir aussi dans la'view-menu' l'option 'statistiques'. Ceci fourni un fichier en format CSV\footnote{Character Separated Values : un format standard pour des fichiers sous forme de tableau} qui peut être lu et statistiquement conditionné par des programmes comme OpenOfficde.org calc ou Microsoft\texttrademark Excel\texttrademark Ce fichier contient toutes les positions comptabilisées avec fréquence, coûts et chiffre d'affaire.


\subsection{Consultations à facturer}
\index{facturation} Cette View (cf Fig. \ref{fig:konsv})sert à choisir les consultations, dont une facture doit être établie.
Ceci concerne seulement les consultations du mandant actuel.

\begin{figure}[hb]
\includegraphics{images/konsv}
\caption{Konsultation zur Verrechnung auswählen}
\label {fig:konsv}
\end{figure}
Pour cela il y a des possibilités suivantes :
\begin{itemize}
  \item Choix automatique (Icône de baguette magique) : Les consultations à facturer sont choisies automatiquement d'après certaines règles et transférées dans la liste de choix. Cela sera expliqué ci-dessous plus précisément (facturation automatique).
  \item Tirer le nom du patient vers la liste de choix : Toutes les consultations concernant tous les cas (LAMAL , LAA etc) du patient choisi sont marquées pour être facturées.
  \item Tirer des cas (LAMA ; LAA etc) depuis la liste vers la liste de choix : Toutes les consultations des cas choisis sont marquées pour être facturées.
  \item Tirer des consultations depuis la liste vers la liste de choix : Seulement les consultations choisies sont marquées pour être facturées.
\end{itemize}
Avec toutes les méthodes mentionnées vous pouvez librement modifier votre choix postérieurement. Vous pouvez ajouter d'autres éléments, ou vous pouvez éliminer des éléments (après clic droit sur un élément dans la liste de choix), ou vous pouvez supprimer même le choix entier. A ce moment là il n'y a pas encore eu de modification des données .

Si vous avez fini de choisir vous pouvez cliquer sur  \glqq établir les factures\grqq pour que les factures pour tous les éléments présents dans le choix soient établies. Toutes les consultations qui font partie d'un cas sont toujours résumées. Plusieurs factures sont ainsi fournies si plusieurs cas d'un patient  se trouvent dans la liste de choix.

\subsubsection{Facturation automatique}
\index{Facturation automatique}\label{auto}
Par cette méthode la facturation des consultations suit des critères spécifiquement déterminés auparavant  (cf Fig. \ref{fig:rnautomatik}).
\begin{figure}
  % Requires \usepackage{graphicx}
  \includegraphics[width=1.0\textwidth]{images/rechnungsautomatik}\\
  \caption{Halbautomatischer Rechnungsvorschlag}\label{fig:rnautomatik}
\end{figure}

\begin{itemize}
\item Proposer tout les cas qui sont retenu pour la facturation : Si vous cliquez cette 'check-box' les consultations des cas seront choisis pour lesquels vous avez spécifié une date de facturation dans les détails du cas. (voir Fig. \ref{fig:falldetail}).
\item Proposer toutes les séries de traitement qui ont commencé avant le … : Choisit toutes les consultations non facturées d'un cas (LAMA, LAA etc) jusqu'aujourd'hui à condition qu'au moins une consultation avait eu lieu avant la date limite.
\item Proposer toutes les séries de traitement qui ont été terminées avant le … : Choisit toutes les consultations non facturées d'un cas (LAMA, LAA etc) à condition que la dernière consultation avait eu lieu avant la date limite.
\item Proposer toutes les séries de traitement dont la somme est plus grande que … : Choisit toutes les consultations non facturées d'un cas (LAMAL, LAA etc) à condition que la somme totale de la facture dépasse le montant choisi.
\item Proposer la facturation de tous les traitements du trimestre passé : Facturation du dernier \textit{trimestre} selon le calendrier avec les dates limites suivantes : 31.3 ; 30.6 ; 30.9 et 31.12.
\end{itemize}
S'applique à toutes les options : N'est exploitée seulement lorsque le crochet est mis dans la 'check-box'. Il ne suffit donc pas de seulement introduire une valeur. Les différentes options s'appliquent de façon additive : Finalement toute consultation est choisie pour la facturation à la quelle s'applique au moins un des critères actifs.

\medskip

\clearpage

\subsection{Factures}
\begin{figure}[ht]
  % Requires \usepackage{graphicx}
  \includegraphics[width=1.0\textwidth]{images/rechnungsview}\\
  \caption{Rechnungen-View}\label{fig:rechnungen}
\end{figure}

Dans cette 'View' (Fig. \ref{fig:rechnungen})vous voyez les factures établies. Une facture a toujours un 'état' spécifique :
\begin{description}
    \item [Ouvert] immédiatement après la facturation. 
    \item [Ouvert et imprimé ] La facture avait été édité au moins une fois (soit par impression soit par une autre méthode d'exportation). A partir de ce moment le délai de payement commence à s'écouler. (Toutefois Elexis ne peut pas constater si par exemple une facture n'a pas été correctement imprimé ou si elle n'avait pas été envoyée. Pour cette raison se trouve ici une source d'erreurs potentielle.)
    \item[Rappel] Le rappel a été établi mais pas encore imprimé.
    \item[Rappel imprimé ] Le rappel a été imprimé.
    \item [2ième Rappel établi, 2ième Rappel imprimé, 3ième Rappel établi, 3ième Rappel imprimé ]: en analogie
    \item[Payement partiel ] Il y a au moins un payement mais qui ne couvre pas la totalité de la facture.
    \item[payée] La facture a été entièrement payée par un ou plusieurs paiements.
    \item [payé trop] ça peut aussi arriver pour une fois ;-)
    \item [Perte partielle ] Une partie de la facturation est définitivement mise dans les pertes (Contrairement à la situation du  \glqq payement partiel \grqq{} vous ne comptez plus avec un payement de plus)
    \item [Perte totale] La facture entière fait partie des pertes. 
    \item [En poursuite ] exactement ça
    \item [Annulé ] Une fois une facture établie, elle ne peut plus être supprimée. Cela doit être ainsi car sinon il serait possible que quelqu'un réclame une facture qui n'existe plus ou que quelqu'un veut des renseignements concernant une facture inexistante. Lorsqu'une facture est invalide pour une raison quelconque (erreurs, exonération du montant etc.) elle doit être annulée. L'annulation d'une facture a dans tous les aspects pratiques le même effet qu'une suppression à part du fait que le numéro de facture reste attribué et que la facture peut être examinée plus tard encore.
    \item [faux] Si un module de facturation constate qu'une facture contient des fautes (par ex. le module Trust-X pourrait réclamer qu'il y manquent certaines numéros EAN) la facture concernée reçoit le signe 'faux' et peut donc être corrigée.
    \item [à imprimer] Toutes les factures encore ouvertes mais pas encore imprimées de même que les rappels pas encore imprimés se trouvent dans cet état.
    \item [encours des crédits ] L'ensemble de tout des factures 'ouvertes et imprimées' , des 'rappels imprimés', des '2ième rappel imprimé' des '3ième rappel imprimé' des 'payements partiels' et des factures 'en poursuite'. Il s'agit donc de toutes les factures dont vous attendez encore un payement.
    \item [Stop des rappels ] exactement ça
\end{description}

La liste de facturation peut être sélectionné selon des différents critères. Pour mettre à jour la liste avec les données modifiées veuillez cliquer sur le bouton 'actualiser liste'. Pour afficher les factures d'un certain état veuillez choisir l'état en question sur le menu déroulant en haut à gauche sous 'état' (cf Fig. \ref{fig:rechnungen}). Pour afficher que la facture d'un patient spécifique cliquez sur \glqq patient\grqq{}. La boite de dialogue s'affiche pour vous laisser choisir les contacts. Introduisez la nom du patient et sélectionnez-le ensuite sur la liste. Cliquez sur o.k ou sur annuler pour retourner à l'affichage de la liste de tout les patients.

Pour ne choisir qu'un numéro de facture spécifique veuillez entrer le numéro dans le champ 'No de facture' et appuyez sur la touche 'Entrée' ou cliquez sur 'actualiser liste'.  Pour chercher une facture avec un montant spécifique (par ex. pour classer un payement de provenance non connue veuillez introduire le montant et poussez la touche d'entrée.

\medskip
Si vous cliquez sur le symbole 'filtre' vous recevez des options supplémentaires concernant l'affichage. 

\begin{wrapfigure}{l}{7cm}
\includegraphics[width=7cm]{images/rechnungsfilter}
\end{wrapfigure}
Les champs 'Montant : de  - à' servent à filtrer une somme spécifique. Vous pouvez aussi remplir qu'un des deux champs de sorte que l'autre devienne un limitation ouverte.
 Les champs \glqq date de facturation de\grqq{} et \glqq date de facturation jusque\grqq{} servent à chercher des factures uniquement émises entre les deux dates mentionnées. Par contre les champs  \glqq dates d'état : de - à\grqq{} permettent à filtrer des factures dont la dernière modification d'état se trouve entre les deux dates introduites. Aussi ici vous pouvez remplir un seul champ et laisser l'autre ouvert. 
 Si vous sortez de ce dialogue en cliquant sur OK, la liste sera actualisés selon vos nouveaux critères.

\medskip

Tant que le bouton 'filtre' reste enclenché toutes les 'actualisations de liste' seront liés avec le Filtre ET. Si vous filtrez par exemple la date 'd'état' jusqu'au 30.10.2007 et vous cliquez après dans le menu déroulant sous 'état' sur '2ième rappel imprimé' et vous appuyez sur le bouton 'actualiser liste', vous trouverez toutes les factures qui avaient été mises avant le 30.10.2007 dans l'état '2ième rappel imprimé'. 

En bas de cette fenêtre vous voyez la quantité de factures qui remplissent ces conditions et les sommes les concernant.

\subsubsection{'View-menu' de la liste des factures}
Le 'View-menu' (triangle à droite en haut, cf Fig. \ref{fig:rechnungen}) a des options suivantes :
\begin{description}
\item [Affichage complet / Affichage réduit ] Montre toutes les donnés en détail ou les réduit aux titres.
\item [Imprimer liste ] Imprime une liste de tout les patients respectivement factures qui sont marquées dans l'affichage actuel. Pour cette action il faut qu'il y existe un modèle d'impression pour le système nommé 'liste' qui contient un champ (liste).
\end{description}
\subsubsection{Changer une facture}
Vous pouvez changer la facture si vous cliquez avec la touche droite de la souris sur une facture dans la liste :
\begin{description}
\item [Facturer] Facturer une facture séparée. (v. ci-dessous)
\item [Comptabiliser paiement ] Vous pouvez introduire ici les paiements manuellement . Par exemple des paiements comptants ou les paiements par acompte. (Normalement la comptabilisation par fichier ESR se fait automatiquement).
\item [Ajouter émolument ] Ajouter manuellement par ex. les frais de rappel.
\item [Changer l'état ] On peut changer l'état de la facture manuellement. Elexis reconnaît la majorité des changements de l'état d'une facture. Ainsi change l'état des factures payées automatiquement sur  \glqq payée\grqq{} lorsqu'un fichier ESR de la banque est comptabilisé. Certains changements de l'état ne peuvent se faire que manuellement. Par exemple : Elexis ne peut par distinguer automatiquement entre \glqq paiement partiel\grqq{} et \glqq perte partielle\grqq{} puisque ceci doit être lié à une décision du créancier. La même chose vaut pour les factures \glqq En poursuite\grqq{} et la \glqq perte totale\grqq{}.
    A part de ça il faudra toujours être prudent avec des changements manuelles de l'état d'une facture car il n'y se font pas de corrections de comptabilisation.
\item [Augmenter le niveau de rappel] Ceci augmente le niveau de rappel d'une étape jusqu'au maximum du 3ième rappel.
\item [Annuler] Ceci permet d'annuler une facture. Il y existe la possibilité de débloquer certains traitements (lorsque la facture était erronée et doit être refaite) ou de les laisser bloqués (si le traitement ne doit définitivement pas être facturé).
\end{description}

\subsubsection{Facturation}
Par le bouton  \glqq facturation\grqq{} toutes les factures sélectionnées seront facturées. (Pour sélectionner une facture cliquez avec la touche gauche de la souris sur la facture. Pour sélectionner plusieurs factures sur la liste cliquez sur ces factures en tenant enfoncée en même temps la touche CTRL (ou MAC). Pour marquer toute une rangée de factures cliquez d'abord sur la première facture et après, en tenant la touche SHIFT, sur la dernière facture de la rangée.) Elexis ne facturera donc \textit{pas} toute la liste mais seulement les factures sélectionnées de la liste !


Les cibles possibles de la facturation dépend des différents 'Plugins de facturation' installés.
La cible peut par exemple être une imprimante qui imprime des factures selon TARMED, mais elle peut aussi être un fichier XML ou directement un centre de confiance. Des informations plus précises vous pouvez trouver dans les chapitres respectives  (Tarmed: page \pageref{arzttarife}).
\bigskip
En cliquant sur le symbole de la baguette magique vous mettez en route l'automatisme de l'établissement des rappels. Celle-ci choisit les factures à rappeler selon les critères fixés dans le champ en bas à droite, augmente le niveau de rappel, ajoute des émoluments comme prédéterminé et réunit ces factures en groupes  \glqq à imprimer\grqq{}.

\subsection{Compte du patient}
\index{Compte}Cette 'View' permet de voir tous les mouvements de compte du patient. Les factures se comptabilisent par des chiffres négatifs, les paiements ou annulations par des chiffres positifs de sorte que vous puissiez apercevoir facilement et à travers plusieurs facturations et paiements où vous en êtes avec ce client du point de vu financier.

\subsection{Liste des comptes des patients}
Cette liste permet de voir tous les mouvements de compte en même temps.

\subsection{Prestations}
Cette 'View' fonctionne de façon semblable comme la 'View-Diagnostics'  (page \ref{view:diagnosen} à la page \pageref{view:diagnosen}): Dépendant des Plugins pour le codage des prestations installés on trouve pour chaque système de codage un onglet . Pour des précisions veuillez consulter \ref{concept:leistung} à la page \pageref{concept:leistung}.




	% *******************************************************************************
% * Copyright (c) 2007 by Elexis
% * All rights reserved. This document and the accompanying materials
% * are made available under the terms of the Eclipse Public License v1.0
% * which accompanies this distribution, and is available at
% * http://www.eclipse.org/legal/epl-v10.html
% *
% * Contributors:
% *    G. Weirich - initial implementation
% *
% *  $Id: rest.tex 2933 2007-07-29 10:05:35Z rgw_ch $
% *******************************************************************************
% !Mode:: "TeX:UTF-8" (encoding info for WinEdt)

\section{Diverse Views}

\subsection{Datenanzeige}
Dies ist eine View, die beliebige Felder der Elexis-Datenbank anzeigen
kann. Mehrere Exemplare dieser View können (mit unterschiedlichen oder denselben
Inhalten) in einer Perspektive eingebunden werden (S. Abb. \ref{figure1}).
\begin{figure}[hb]
\includegraphics{images/data1}
\caption{Zwei Fenster der 'Datenanzeige'}
\label {figure1}
\end{figure}
Durch Klick auf den \textbf{+} - Button können Sie ein weiteres Exemplar der
View öffnen, durch Druck auf den Editieren-Button können Sie die anzuzeigenden
Daten einstellen.

\begin{figure}[hb]
\includegraphics{images/data2}
\caption{Eingabedialog für den Datentyp}
\label{figure2}
\end{figure}
Es erscheint dann eine Dialogbox wie in Abb. \ref{figure2}.
Sie können hier jeden Datentyp einsetzen, der auch in Textvorlagen als
Platzhalter verwendet werden kann (Vgl. S. \pageref{Platzhalter}).
Wenn Sie die Checkbox 'Feld kann geändert werden' ankreuzen, dann können die
Daten (ausreichende Rechte vorausgesetzt) direkt durch Schreiben in dieses
Fenster geändert werden.
Die Anordnung und der Inhalt der Datenanzeige-Views werden beim Verlassen von
Elexis, oder bei Betätigen der Menüaktion 'Perspektive speichern' gespeichert.

\subsection{Fixmedikation}
Diese View zeigt die Fix- oder Dauermedikation des aktuell selektierten
Patienten an (S. Abb. \ref{fig:fixmedi})

\begin{figure}[htp]
\begin{center}
  \includegraphics{images/fixmediview}
  \caption{Fixmedikation}
  \label{fig:fixmedi}
\end{center}
\end{figure}
Sie können Medikamente aus dem Artikel-Fenster oder aus einem Rezept in diese
View ziehen, und Sie können auch Artikel aus der Fixmedikation in ein Rezept
ziehen. Mit Klick auf \glqq Hinzu\ldots\grqq{} öffnen Sie die Artikel-View. Mit
Klick auf \glqq Liste\ldots\grqq{} erstellen Sie eine Einnahmeliste für den
Patienten. Dazu muss eine Textvorlage namens \glqq Einnahmeliste\grqq{}
existieren, und diese muss an einer Stelle den Platzhalter [Medikamentenliste]
enthalten. Mit Klick auf \glqq Rezept\grqq{} erstellen Sie ein Rezept mit der
Dauermedikation. Hierzu muss eine Textvorlage namens \glqq Rezept\grqq
existieren, welche an einer Stelle den Platzhalter [Rezeptzeilen] enthält.

\subsection{Medikamenten-Verlauf}
Diese View zeigt alle Medikamente, die beim aktuellen Patienten je verschrieben oder abgegeben worden sind, mit Datum und Dosierung (falls angegeben). Durch Klick auf die entsprechenden Spaltenköpfe können Sie nac Abgabedatum oder Medinamen sortieren. Bei Medikamenten aus der Fixmedikation wird ausserdem, falls gegeben, das Stopdatum angezeigt.

\subsection{Kompendium online}
Wenn Sie eine aktive Internet-Verbindung haben, dann wird in dieser View das
Arzneimit\-tel-Kom\-pen\-dium der Schweiz angezeigt.

\subsection{Open Drug Database}
Diese View zeigt bei aktiver Internet-Verbindung die entsprechende Site an, die Sie z.. für die Suche nach Generika oder Interaktionen verwenden können.

\subsection{Pendenzen}
Erinnerungen, Reminders, Pendenzen: Diese View zeigt Ihnen Dinge an, an die Sie
denken mochten oder sollten (s. Abb. \ref{fig:pendenzen}).

\begin{wrapfigure}{l}{7.5cm}
  \includegraphics[width=7.2cm]{images/pendenzenview}
  \caption{Pendenzen-View}
  \label{fig:pendenzen}
\end{wrapfigure}

Eine Pendenz hat ein Fälligkeitsdatum und einen Status (geplant, fällig,
überfällig, erledigt, bleibt unerledigt).

Es gibt folgende Typen von Pendenzen:
\begin{itemize}
  \item Aufträge für eine bestimmte Person oder Aufträge an alle.
  \item Erinnerungen, die immer angezeigt werden, sobald ihr Fälligkeitsdatum
  erreicht oder überschritten ist.
  \item Erinnerungen, die nur dann angezeigt werden, wenn sie fällig sind
  \textit{und} wenn ein bestimmter Patient ausgewählt ist.
  \item Pendenzen, die nicht nur angezeigt werden, sondern die auch direkt eine
  bestimmte Aktion auslösen können (z.B. einen Serienbrief schreiben).
\end{itemize}

Wenn ein Patient fällige Pendenzen hat, dann wird in der Patientenliste das
Pendenzen-Symbol angezeigt (s. Abb. \ref{fig:pendenzen})

Um eine neue Pendenz zu erstellen, klicken Sie auf das Symbol \glqq Neue
Pendenz\grqq{}(roter Stern). Es erscheint dann eine Dialogbox, in der Sie Text,
typ, verantwortliche Person, Fälligkeitsdatum und anfänglichen Status der
Pendenz eingeben können.

Doppelklick auf eine Pendenz öffnet diese zum Bearbeiten. Es erscheint dieselbe
Dialogbox.


	
	
\chapter{Plugins}
	% *******************************************************************************
% * Copyright (c) 2007 by Elexis
% * All rights reserved. This document and the accompanying materials
% * are made available under the terms of the Eclipse Public License v1.0
% * which accompanies this distribution, and is available at
% * http://www.eclipse.org/legal/epl-v10.html
% *
% * Contributors:
% *    G. Weirich - initial implementation
% *
% *  $Id: einleitung.tex 4904 2009-01-03 17:58:33Z rgw_ch $
% *******************************************************************************
% !Mode:: "TeX:UTF-8" (encoding info for WinEdt)

\section{Introduction}
Pour établir des lettres , ordonnances, certificats etc. Elexis utilise de façon standardisée un logiciel valable : OpenOffice
Ceci ne doit pas forcément être la seule solution car le traitement de texte est appliqué par Elexis en forme de Plugin. On pourrait donc aussi laisser créer un Plugin pour Microsoft\texttrademark{}Office\texttrademark{}  ou n'importe quel autre logiciel de traitement de texte. Nous nous limitons ici par contre à l'OpenOffice qui est le logiciel de référence pour Elexis.
L'origine de OpenOffice se trouve dans StarOffice qui avait été développé dans les années 80 et qui représente aujourd'hui une Office-Suite en analogie à Microsoft Office avec la différence qu'il s'agit d'un produit open source disponible pour plusieurs systèmes d'exploitation.
Dans l'installateur de la version Windows de Elexis une version adaptée de OpenOffice.org est intégrée. Sous Linux on peut se servir de la version OpenOffice qui est normalement intégrée dans Linux. Pour Macintosh l'intégration ne fonctionne malheureusement pas encore. Faites attention de n'installer q'une seule version de OpenOffice sur votre système car sinon ceci pourrait provoquer des conflits entre les versions.


\medskip

Après l'installation de OpenOffice et de Elexis il faut que les deux programmes 'fassent connaissance'. Pour ceci il faut configurer le Plugin de texte dans Elexis de sorte qu'il ait accès sur OpenOffice.(Le Plugin
\glqq NOA-Text\footnote{Nous utilisons le module 'Nice Office Access' de \href{http://www.ubion.org}{www.ubion.org}
 pour intégrer OpenOffice}\grqq{}doit être installé, chose qui est normalement d'emblée le cas.

Séléctionnez dans le menu  \textsc{Fichier - Options} Vous trouvez dans la liste à gauche : \textsc{Traitement de texte}
Il y apparaît un boite de dialogue comme en \ref{fig:text1}. Choisissez là  \glqq
%\usepackage{graphics} is needed for \includegraphics
\begin{figure}[htp]
\begin{center}
  \includegraphics{images/text1}
  \caption{Konfiguration des Textplugins}
  \label{fig:text1}
\end{center}
\end{figure}

NOA-Text\grqq{}. Après vous choisissez dans la liste à gauche OpenOffice.org (cf Fig. \ref{fig:text2}
En cliquant sur \glqq définir\grqq{} vous définissez le chemin d'accès à votre installation OpenOffice.
%\usepackage{graphics} is needed for \includegraphics
\begin{figure}[htp]
\begin{center}
  \includegraphics{images/text2}
  \caption{Konfiguration der OpenOffice-Installation}
  \label{fig:text2}
\end{center}
\end{figure}
Vous cliquez ensuite dans la boite de dialogue qui s'ouvre sur OpenOffice.org et cliquez sur 'terminer'.
Dans la boite de dialogue encore ouverte vous cliquez sur 'ok'. OpenOffice devrait être à disposition dès le prochain démarrage de Elexis. Lors de la première utilisation vous devez encore accepter les conditions de licence de OpenOffice.org.)




	% *******************************************************************************
% * Copyright (c) 2007 by Elexis
% * All rights reserved. This document and the accompanying materials
% * are made available under the terms of the Eclipse Public License v1.0
% * which accompanies this distribution, and is available at
% * http://www.eclipse.org/legal/epl-v10.html
% *
% *  $Id: agenda.tex 4904 2009-01-03 17:58:33Z rgw_ch $
% *******************************************************************************
% !Mode:: "TeX:UTF-8" (encoding info for WinEdt)

\section{Agenda de Elexis}\label{Agenda}
\index{rendez-vous} Il s'agit d'une agenda multiposte pour plusieurs mandants. Ce Plugin fait parti de la distribution standard. Ce qui suit explique la configuration et l'utilisation de l'agenda.
\subsection{Configuration}


Choisissez dans le menu  \textbf{Fichier -Options}. Si le Plugin Agenda est installé, vous trouverez là une rubrique  \textit{Agenda}:



%\includegraphics[width=3in,bb=0 0 382 420]{images/settings1}
% settings1.jpg: 499x548 pixel, 94dpi, 13.49x14.81 cm, bb=0 0 382 420

\includegraphics{images/settings1}

Dans la partie supérieure  \textit{Zone d'utilisateur} \index{Agenda!Zone d'utilisateur} vous pouvez définir combien et quelles agendas peuvent être gérées parallèlement. Il peut s'agir par exemple d'une agenda pour chaque médecin d'un
\index{cabinet de groupe} cabinet de groupe, ou des agendas pour des différentes ressources comme par exemple le médecin, ECG, Laboratoire, Ergométrie etc.
La quantité et le titre des 'zones d'utilisateur' dépend entièrement des besoins spécifiques de votre cabinet médical.

En dessous vous trouvez \textit{type de rendez-vous} \index{Agenda!type de rendez-vous}. Dans cette rubrique vous définissez quels types de rendez-vous sont à gérer par l'agenda dans votre cabinet. Un 'type de RDV' peut être toute sorte d'inscription qui se fera dans l'agenda. Par exemple aussi des  \textit{colloques avec l'équipe},  \textit{Acupuncture},  \textit{Check-Up},  \textit{Formation}  etc. Les 'types de RDV' seront affichés plus tard de façon individuelle et peuvent suivre des horaires différents.  Les deux premières inscriptions , 'libre' et 'réservé', doivent être introduits avec cette signification et dans cette séquence mais peuvent aussi être nommé différemment (par ex. \textit{vide}  et \textit{bloqué}). Les autres lignes vous pouvez nommer de façon arbitraire et il peut y avoir autant que vous voulez.

Le champ tout en bas, \textit{état du rendez-vous}\index{Agenda!état du rendez-vous}, est également très dépendant de la réalité spécifique de votre cabinet médicale. Comme dans le cadre des 'types de RDV' les deux premières inscriptions sont fixes dans leur signification mais peuvent changer de nom, tandis que les autres inscriptions sont tout à fait libres. On pourrait introduire ici par ex.   \textit{annulé}, \textit{attend résultats labo}, \textit{attend médecin}  etc.

La prochaine page de réglage de l'agenda concerne les icônes \index{Agenda!icônes} par lesquels les différents types de rendez-vous peuvent être affichés. Vous arrivez aux icônes dans la liste gauche sous rubrique 'utilisateurs' - 'Agenda-icons'.

\includegraphics[width=3in]{images/settings2}

(Si en cliquant sur 'Agenda-icons' les 'types de RDV' que vous venez d'introduire ne s'affichent pas, il faut fermer Elexis et redémarrer pour qu'ils soient lus correctement.).
Cliquez sur le bouton  \textit{modifier} et choisissez une image dans le format  .*gif, *png oder *.ico .

La partie suivante concerne les couleurs d'affichage pour les 'types de RDV' et 'état du RDV' :

\includegraphics[width=3in]{images/settings3}

Choisissez sous 'utilisateur -couleurs' la couleur qui vous convient pour les différents champs des types de rendez-vous et d'état du RDV. Après un double-clic sur un champ vous pouvez choisir sa couleur.
\includegraphics[width=3in]{images/settings4.png}


La ligne supérieure concerne les 'types de RDV'. Les couleurs affichées ici seront affichées dans le dialogue où on introduit les rendez-vous.
La ligne inférieure concerne 'l'état du RVD'. Les couleurs affichées ici seront affiché dans l'affichage normale de l'agenda.


La partie suivante du réglage de l'agenda concerne l'organisation de la journée \index{organisation de la journée}:

\includegraphics[width=3in]{images/settings5.png}
% settings5.png: 733x406 pixel, 96dpi, 19.39x10.74 cm, bb=0 0 550 304

Ici on peut régler pour chaque jour de la semaine les périodes qui seront de façon standard à disposition pour la planification. Ceci peut naturellement aussi être changé ultérieurement pour chaque jour mais ici il s'agit des préréglages approchés.

Choisissez en haut la 'zone d'utilisateur' souhaitée (par ex. un médecin du cabinet de groupe) et introduisez ici le début et la fin des périodes qui ne sont pas à disposition pour la planification. Ces plages de temps seront ensuite occupé par le 'type de RDV' \textit{bloqué} bloqué. Vous pouvez introduire des périodes de ce genre ad libitum pour chaque jour de la semaine.  \index{jour de la semaine}.

La dernière partie du réglage de l'agenda concerne le réglage du temps à programmer pour chaque 'type de RDV' :

\includegraphics[width=3in]{images/settings6.png}
% settings6.png: 537x394 pixel, 96dpi, 14.21x10.42 cm, bb=0 0 403 295

Ici vous voyez pour chaque 'zone d'utilisateur' et chaque 'type de RDV' une possibilité de fixer le temps à programmer . Vous pouvez changer chaque champ en cliquant dessus et en écrivant par-dessus. L'agenda consacrera de façon standardisé le temps fixé pour ce type de RDV mais celui pourra être adapté manuellement si nécessaire. Si vous introduisez à un endroit 0, le type de RDV ne sera pas disponible pour cette 'zone d'utilisateur'. La ligne supérieure est le temps standard qui est toujours appliqué si le système ne trouve pas une autre durée spécifique.
En outre vous pouvez faire quelques réglages pour imprimer des cartes de rendez-vous. Ces réglages vous pouvez trouver sous \textit{impression}\index{Agenda!impression}.

\includegraphics[width=3in]{images/settings-agenda-druck1.png}

Le modèle standard pour l'impression des cartes pour rendez-vous s'appelle  \textit{carte RDV}. Vous pouvez choisir un autre modèle système quelconque. Les heures du rendez-vous seront intégrés dans la variable
\textit{[rendez-vous]}.

Lors de l'impression de la carte RDV une fenêtre s'ouvre qui montre un aperçu de la carte RDV. Vous pouvez imprimer la carte RDV depuis le traitement de texte.
Si vous voulez que la carte RDV soit imprimée directement sur l'imprimante, marquez
\textit{imprimer directement}. Vous pouvez ensuite choisir l'imprimante et de façon optionnelle le bac de l'imprimante en question. Si vous ne choisissez pas de bac , le bac mémorisé dans le modèle système ou le bac standard de l'imprimante séléctionnée sera utilisé.

\includegraphics[width=3in]{images/settings-agenda-druck1.png}

Vous venez de finir la configuration de l'agenda. Cliquez sur la touche \textit{OK} et fermez Elexis. A partir du prochain démarrage du logiciel, les nouveaux réglages seront à disposition.

Les prochaines pages ont pour but de vous montrer l'utilisation de l'agenda.

\subsection{Utilisation de l'agenda}

La 'View-Agenda'  (Fig. \ref{fig:agenda1}) n'est normalement pas affichée.  Pour la visualiser choisissez dans le menu
 \textbf{fenêtre-view-autres}, tapez dans le champ de filtre en haut  \textit{agenda}, choisissez l'agenda et cliquez  \textit{OK}. Tirez ensuite la fenêtre de l'agenda \index{agenda-fenêtre} dans la position souhaitée de la perspective comme c'était décrit sous \textit{premiers pas} \ref{tour:customize} à la page \pageref{tour:customize}.
\begin{wrapfigure}[23]{L}{3in}
\includegraphics[width=3in]{images/use2.png}
\caption{Agenda-view standard}\label{fig:agenda1}
\end{wrapfigure}
Dans la partie à droite vous pouvez régler la date. Si vous cliquez sur le bouton 'aujourd'hui'vous arrivez au jour actuel.
Si vous cliquez sur les flèches vous pouvez avancer ou rétrocéder un mois et si vous cliquez sur les flèches doubles, vous pouvez avancer ou rétrocéder une année. Pour choisir une date spécifique cliquez directement dessus. Si vous cliquez sur le triangle en haut à droite, vous ouvrez le 'view-menu' dans lequel vous pouvez choisir la 'zone d'utilisateur' que vous voulez afficher et les limites des journées réglables ici individuellement.

Dans le secteur principale vous pouvez voir les inscriptions de l'agenda avec les couleurs et icônes que vous avez défini pour les différents types de rendez-vous. Les périodes libres sont en couleur verte. Prenez en considération que dans cette agenda la durée des périodes n'est pas proportionnel à leur espace visualisé. Au début il faut s'habituer un peu mais ceci s'est avéré très utile car par la suite on peut afficher toute la journée dans un espace relativement petit.

\medskip

Dans l'espace en bas à droite vous voyez des informations supplémentaires qui concernent le rendez-vous marqué actuellement.

\bigskip

Si vous double-cliquez sur une plage libre, vous pouvez introduire un nouveau rendez-vous et si vous double-cliquez sur un rendez-vous déjà donné vous pouvez le modifier. Dans les deux cas la boîte de dialogue s'ouvre Fig. \ref{fig:termineingabe}.

\begin{figure}[ht]
\includegraphics[width=5in]{images/use4.png}
% use4.png: 625x523 pixel, 96dpi, 16.53x13.84 cm, bb=0 0 469 392
\caption{Termineingabe-Dialog}\label{fig:termineingabe}
\end{figure}
La boîte de dialogue est assez complexe et contient des multiples plages :

\begin{itemize}
 \item En haut à gauche se trouve un calendrier qui vous permet de choisir aussi une autre journée.
\item En haut au milieu se trouvent les endroits où on introduit l'heure du début et l'heure de fin de la consultation de même que la durée de la consultation.
\item  En dessous vous trouvez la liste des rendez-vous où vous pouvez facilement voir quel
rendez-vous on avait déjà donné. (on peut fixer un ou plusieurs rendez-vous parallèlement).

\item En haut à droit vous trouvez la checkbox  \textit{verouillé}, ce qui bloque toute modification ultérieure du rendez-vous.
\item  En dessous vous trouvez le bouton pour  \textit{placer un rendez-vous}. Vous pouvez après avoir placé un rendez-vous choisir une autre date ou autre heure pour introduire un rendez-vous supplémentaire. (Si vous voulez introduire qu'un seul rendez-vous, vous pouvez cliquer directement sur 'OK'.
\item Au milieu vous trouvez la \textit{barre de la journée}, qui démontre l'organisation de la journée actuellement affichée. Les couleurs correspondent au types de consultations que vous avez défini pour les différents types de rendez-vous dans la configuration. Le curseur gris symbolise la période actuellement choisi. A l'aide de la souris vous pouvez placer ce curseur où vous voulez.
\item En dessous de la barre de la journée vous trouvez l'affichage horaire dont la trame peut être adaptée à vos besoins en cliquant dessus. Pour choisir l'heure pour une consultation vous déplacez le curseur sur la barre de la journée.


\item "	En dessous on trouve les coordonnées du patient séléctionné de même que le type et l'état du rendez-vous en question . Si vous ne voulez pas introduire un nom de patient mais un texte libre vous pouvez l'introduire dans le champ  \textit{identité}.
\end{itemize}

En cliquant sur ok l rendez-vous est introduit et le dialogue se ferme.


Si vous cliquez avec la touche droite sur un rendez-vous un menu contextuel s'ouvre dans lequel vous pouvez changer plusieurs détails concernant ce rendez-vous.

\includegraphics[width=4in]{images/use5.png}
% use5.png: 412x416 pixel, 96dpi, 10.90x11.01 cm, bb=0 0 309 312

Le plus important dans ce menu contextuel semble être \index{les changements de l'état du rendez-vous}: Puisqu'un tel changement se reproduit sur tout les ordinateurs branchés sur le réseau, ceci permet de constater sur n'importe quel poste de travail si par exemple le patient est \textit{arrivé}.

Si vous voulez changer pour une seule journée les périodes réservées vous pouvez le faire en choisissant dans le 'View-menu' (triangle à droite en haut)  \textit{les limitations de la journée}. Le champ de dialogue suivant se montre :

\includegraphics[width=3in]{images/use3.png}
% use3.png: 438x244 pixel, 96dpi, 11.59x6.46 cm, bb=0 0 328 183

Ici vous pouvez fixer les périodes réservées (bloquées) pour la journée actuelle comme décrit dans le chapitre configuration.

\subsubsection{Plusieurs agendas en même temps}
Vous pouvez sans problème laisser afficher plusieurs fenêtres d'agenda simultanément par exemple pour des différentes 'zones d'utilisateurs' ou des différentes journées.
\includegraphics[width=3in]{images/agendamulti.png}
% agendamulti.png: 308x393 pixel, 96dpi, 8.15x10.40 cm, bb=0 0 231 295

\subsubsection{Fenêtre agrandie}

Votre assistante médicale aimerait peut être avoir sur son écran une agenda qui donne plus d'informations en même temps. Utilisez pour ceci la 'View' :  \textit{Agenda - grande}:

\includegraphics[width=5in]{images/agenda2.png}
% agenda2.png: 605x500 pixel, 96dpi, 16.01x13.23 cm, bb=0 0 454 375

Comme vous voyez tout les informations importantes peuvent être affichées de façon synoptique. Les fonctions expliquées de l'agenda restent par contre les mêmes. Si vous voulez vous pouvez naturellement aussi utiliser les deux 'views' de l'agenda simultanément.

\subsubsection{Imprimer des cartes de rendez-vous}

Dans le 'View-menu' (triangle à droite en haut) de l'agenda vous pouvez sélectionner \textit{imprimer carte de rendez-vous} pour le patient spécifique. Le style correspondant s'appelle \textit{carte de rendez-vous}.
Des informations plus détaillées pour la configuration vous trouvez ci-dessus.

Lors de l'impression d'une carte de rendez-vous apparaît une fenêtre avec la carte de rendez-vous préparée. Vous pouvez imprimer cette carte depuis le logiciel de traitement de texte et vous pouvez fermer cette fenêtre après avoir cliqué sur \textit{OK} ou \textit{Annuler}.


	% *******************************************************************************
% * Copyright (c) 2007-2008 by Elexis
% * All rights reserved. This document and the accompanying materials
% * are made available under the terms of the Eclipse Public License v1.0
% * which accompanies this distribution, and is available at
% * http://www.eclipse.org/legal/epl-v10.html
% *
% * Contributors:
% *    G. Weirich - initial implementation
% *
% *  $Id: konsviews.tex 3329 2007-11-07 17:44:06Z rgw_ch $
% *******************************************************************************

% !Mode:: "TeX:UTF-8" (encoding info for WinEdt)
\section{Résultats dans Elexis}
\label{befunde}
\index{Résultats}\index{série de résultats Quick/TP}
Intégration des séries de résultats datés et classés selon le texte (par ex. poids, glycémie, Quick/TP, résultats radiologiques etc.) .
\subsection{Configuration}

\begin{figure}[htbp]
   \begin{minipage}{0.35\textwidth}
       \centering
       \includegraphics[width=0.9\textwidth]{images/befunde1}
       \caption{Befund}
       \label{fig:befundesettings}
     \end{minipage}\hfill
     \begin{minipage}{0.65\textwidth}
     Si ce Plugin est installé, vous trouvez dans le menu 'Fichier-Options' une rubrique  \textit{résultats}. Cette rubrique est probablement encore vide  (Fig. \ref{fig:befundesettings}).\\

     Pour ajouter un nouveau paramètre de résultats cliquez sur \textit{ajouter}. Vous serez ensuite demandé d'introduire le nom du paramètre. Nous avons choisi 'radiographie'. Il apparaîtra un onglet avec le nom du paramètre. Vous devez encore créer des champs pour l'introduction des données.

    \end{minipage}
\end{figure}
\begin{figure}[htbp]
   \begin{minipage}{0.35\textwidth}
       \centering
        % befunde2.png: 538x515 pixel, 96dpi, 14.23x13.62 cm, bb=0 0 403 386
       \includegraphics[width=0.9\textwidth]{images/befunde2}
       \caption{Parameter 2}
       \label{fig:befundesettings}
     \end{minipage}\hfill
     \begin{minipage}{0.65\textwidth}
        Cliquez après chaque ligne sur   \textit{Apply}  réspectivement  \textit{appliquer}:
        Si un champ doit contenir plusieurs lignes cliquez sur le 'checkbox' correspondante. Fig. \ref{fig:befunde4}vous montre une variante avec plus que deux lignes:

    \end{minipage}
\end{figure}
\begin{figure}[htbp]
   \begin{minipage}{0.35\textwidth}
       \centering
    \includegraphics[width=0.9\textwidth]{images/befunde7.png}
    % befunde7.png: 580x520 pixel, 96dpi, 15.34x13.76 cm, bb=0 0 435 390
    \caption{Mehrspaltig}\label{fig:befunde4}
       \label{fig:befunde4}
     \end{minipage}\hfill
     \begin{minipage}{0.65\textwidth}
Certaines valeurs peuvent aussi être calculées au lieu d'être introduites directement. Vous pouvez introduire pour cela simplement une expression en forme de :  \textit{Résultat = Formule }, où par Fx vous pouvez vous référer à d'autres lignes de la même page. L'exemple à gauche montre comment calculer le BMI (indice de masse corporelle) avec les données introduites pour le poids et la taille. Le résultat est normalement affiché avec une exactitude à 9 chiffres, raison pour laquelle on l'arrondit à une décimale.
    \end{minipage}
\end{figure}

\clearpage

\subsection{Application}
Ouvrez le View des résultats .
\begin{flushleft}
\includegraphics[width=3in]{images/befunde4.png}
% befunde4.png: 276x394 pixel, 96dpi, 7.30x10.42 cm, bb=0 0 207 295
\end{flushleft}

Vous y voyez les paramètres configurés pour les résultats:
\begin{flushleft}
\includegraphics[width=4in]{images/befunde5.png}
% befunde5.png: 621x636 pixel, 96dpi, 16.43x16.83 cm, bb=0 0 466 477
\end{flushleft}
Pour y introduire des nouveaux résultats veuillez cliquer sur le plus vert à droite en haut. 
\begin{flushleft}
\includegraphics[width=3in]{images/befunde6.png}
% befunde6.png: 438x290 pixel, 96dpi, 11.59x7.67 cm, bb=0 0 328 217
\end{flushleft}
Vous pouvez voir maintenant les champs que vous avez introduits lors de la configuration pour y introduire vos résultats. Par un clique sur OK les données introduites seront intégrées. Avec un double-clique sur la ligne vous pouvez ouvrir les champs pour appliquer des corrections.

Si une valeur doit être calculée vous devez cliquer sur le titre en bleu pour que le calcul se fasse :\
\begin{center}
\includegraphics{images/befunde8}
\end{center}

\subsection{Variables dans un texte}
\index{variables} \index{variables dans un texte}
Les résultats peuvent aussi être introduits en forme de variables dans un document de texte. Pour cela vous pouvez appliquer la syntaxe comme c'est décrit sous (\ref{datenfelder_extern}, page \pageref{datenfelder_extern}). Le nom clé du plugin des résultats est : \textsc{Befunde-Data}.

Pour introduire dans un document par exemple un tableau avec l'historique de l'évolution du poids du patient actuel, vous introduisez les variables suivantes dans le texte :
\begin{verbatim}
    [Befunde-Data:Patient:all:Gewicht]  (pour introduire un tableau avec tout les mesures du poids)
    [Befunde-Data:Patient:last:Gewicht] (pour introduire seulement la dernière valeur du poids)
\end{verbatim}

	% !Mode:: "TeX:UTF-8" (encoding info for WinEdt)
\section{Elexis-Arzttarife-Schweiz}
\label{arzttarife}
Da Elexis ein universelles Praxisprogramm ist, ist die Abrechnung nach Tarmed nur eines von beliebig vielen
möglichen Abrechnungssystemen. Konsequenterweise ist deshalb sowohl die Leistungserfassung, als auch die
Rechnungserstellung nicht im Kernsystem enthalten, sondern in Plugins ausgelagert.
Da Elexis aber ein Schweizer Programm ist, ist dieses Plugin natürlich Teil der
Standarddistribution.
\subsection{Einstellungen}
\begin{figure}
  % Requires \usepackage{graphicx}
  \center
  \includegraphics[width=0.9\textwidth]{images/arztrechnung1}\\
  \caption{Abrechnungssysteme}\label{fig:tarmed1}
\end{figure}

Sobald das Tarmed-Plugin installiert ist (was standardmässig immer der Fall ist), können bei den Abrechnungssystemen (s. Abb. \ref{fig:tarmed1}) Tarmedleistungen und Tarmeddrucker ausgewählt werden. Standardmässig werden beim ersten Erstellen eines Falls die Abrechnungssysteme KVG, UVG, IV, VVG und MV eingerichtet; weitere können Sie manuell hinzufügen bzw. werden von Plugins erstellt (z.B. Covercard). Weitere Hinweise zu den Abrechnungssystemen finden Sie im Anhang unter \ref{settings:abrechnungssystem} auf S. \pageref{settings:abrechnungssystem}. \textbf{Wichtig:} Achten Sie darauf, die für Ihren Kanton gültigen Taxpunktwerte einzutragen bevor Sie die ersten Leistungen verrechnen. Denken Sie bei einer Änderung des Taxpunktwerts bitte auch daran, dies hier nachzutragen, bevor Sie Leistungen verrechnen, für die der neue Taxpunktwert gilt. Nachträgliche Änderungen schon gedruckter Rechnungen sind sehr mühsam. Vergessen Sie auch nicht, unter 'Labortarif' den aktuell gültigen TP-Wert einzusetzen.

Ein Taxpunktwert gilt immer ab einem bestimmten Datum und für so lange, bis ein neuer Taxpunktwert eingegeben
wird. Ein einmal eingesetzter Wert kann nicht mehr geändert oder gelöscht werden (da sonst früher damit berechnete
Leistungen ungültig würden). Man kann aber jederzeit einen neuen Taxpunktwert ab einem bestimmten Datum hinzufügen,
 indem man den entsprechenden Knopf klickt.
Wenn Sie den Abschnitt Tarmed öffnen, erscheint der Punkt  Rechnungseinstellungen (Abb. \ref{fig:tarmed2}). Dort können Sie für jeden Mandanten einzeln einstellen, wie die Rechnungsdetails aussehen sollen.

\begin{figure}
  % Requires \usepackage{graphicx}
  \center
  \includegraphics[width=0.9\textwidth]{images/arztrechnung2}\\
  \caption{Tarmed-Einstellungen}\label{fig:tarmed2}
\end{figure}


Wählen Sie in der oberen Combobox einen Mandanten aus. Klicken Sie dann auf das Wort Leistungserbringer.
Es erscheint eine Liste mit allem, was Tarmed von Ihnen wissen will:

\includegraphics[width=3in]{images/tarmed3}
% tarmed3.png: 291x360 pixel, 96dpi, 7.70x9.52 cm, bb=0 0 218 270

\begin{itemize}



\item Anrede - nunja, das war ja noch einfach
\item Titel - ebenso
\item Kanton: Der Kanton, in dem Sie Ihre unter dieser Mandantenbezeichnung erfassten Leistungen erbringen. Falls Sie in mehreren Kantonen tätig sind, sollten Sie für jeden Kanton einen eigenen Mandanten anlegen.
\item EAN - Tarmed sagt: Sie sind ein Artikel. Hier müssen Sie Ihre Europäische
Artikelnummer eingeben. Dies muss zwingend eine 13-Stellige Ziffernfolge sein.
\item NIF - Die IV sagt: Sie sind ein NIF-Träger (was auch immer das sein soll). Hier müssen Sie Ihre NIF eingeben.
\item KSK - Santésuisse sagt: Sie sind ein Konkordatsnummernträger - Hier
müssen Sie Ihre Konkordats- bzw. ZSR- Nummer eintragen, und zwar ohne
Trennzeichen . Es muss immer ein Buchstabe gefolgt von 6 Zahlen sein.
\item Zwischenbemerkung: Elexis ist hier bewusst ausbaufähig designt. Also falls irgendwelchen Bürokraten noch
ein weiteres Nummernsystem einfallen sollte, mit dem wir uns auch noch
klassifizieren müssen - kein Problem. Elexis kann Sie unter beliebig viele Codesysteme eintragen. Aber weiter:
\item TarmedESR5OrEsr9 - Das ESR-System (5-oder 9-stellige Teilnehmernummer).
Steht in Ihrer ESR-Vereinbarung. Meistens wird esr9 stimmen.
\item TarmedESRPlus - esr16or27 ist richtig, wenn Sie den Betrag in der
ESR-Zeile eincodieren möchten/können (dies ist der Normalfall), esr16or27plus müssen Sie angeben,
wenn der Kunde den Betrag manuell eingeben müssen soll.
\item TarmedSpezialität - Ihre Spezialität, unter der Sie mit diesem Mandanten abrechnen
\end{itemize}
Klicken Sie dann je nach Ihren Verhältnissen auf das Wort  \textit{Bankverbindung}  oder  \textit{Postkonto}

\includegraphics[width=4in]{images/tarmed4.png}
% tarmed4.png: 438x250 pixel, 96dpi, 11.59x6.61 cm, bb=0 0 328 187


Bei Bankverbindung wählen Sie anschliessend noch durch Klick auf \glqq
Finanzinstitut\grqq{}  Ihre (hoffentlich schon als Kontakt erfasste)
 Bank aus. Danach müssen noch zwei Details zum ESR-Vertrag ergänzt werden:
TarmedERSParticipantNumber - Die ESR-Teilnehmernummer Ihrer Bank (dort zu erfragen)
TarmedESRIdentity - Ihre BESR-Kundennummer, die Sie ebenfalls von der Bank erfragen müssen

\textbf{Bitte beachten:} Sie werden keine gültige Tarmed-Rechnung ausdrucken
oder ans Trust-Center übermitteln können, bevor Sie nicht alle diese Daten
korrekt eingetragen haben. Dafür kann Elexis nichts, das sind Anforderungen von
Tarmed.

\subsubsection{Drucker-Einstellungen}

Für den Ausdruck von Rechnungen verwendet Elexis den Drucker, der in der Vorlage
gespeichert ist. Um diesen zu ändern, muss die jeweilge Vorlage geöffnet werden,
und dann unter \textit{Datei/Druckereinstellung} der Drucker gewählt werden.
Danach sollte die Vorlage auf dem neuen Drucker ausgegeben werden. Zusätzlich
zum Drucker kann der zu verwendende Schacht konfiguriert werden.

\subsection{Die Rechnung}

Wie oben bereits angedeutet, kann eine Tarmed-Rechnung sehr unterschiedlich sein:

\begin{itemize}
 \item Eine XML-Datei, geeignet zur Übermittlung an ein TrustCenter
\item Eine Datei geeignet zur Übermittlung an die Ärztekasse
\item Ein Tarmed-Rechnungsformular auf Papier, geeignet für Tiers-Payant-Systeme
\item  Eine Seite mit Einzahlungsschein und ein separater Rückerstattungsbeleg für Tiers-Garant-Systeme
\end{itemize}

Welche dieser Methoden die Richtige ist, hängt von Ihrem Kanton, Ihren vertraglichen Regelungen und dem spezifischen Fall ab,
für den die Rechnung ist. So werden UVG-Fälle jeweils im Tiers Payant abgewickelt, während KVG-Fälle in den meisten (aber nicht allen) Kantonen Tiers Garant laufen, was von Krankenkassenseite aber teilweise durch einzelne Tiers-Payant-Verträge
wieder verkompliziert wird. Die bottomline ist: Elexis kann Ihnen da nicht helfen, wird aber die korrekte Rechnungsform erstellen, wenn Sie unter \textsc{Fall Detail}
die korrekten Angaben gemacht haben.

\subsection{Views dieses Plugins}
Dieses Plugin arbeitet mit der existierenden Rechnungen-View des Kernsystems zusammen. Es bringt nur eine eigene View zum Verbuchen der Zahlungseingänge mit:

\subsubsection{ESR}
\begin{figure}[hb]
  % Requires \usepackage{graphicx}
  \includegraphics{images/esr1}\\
  \caption{View zum einlesen von ESR Dateien}\label{fig:esr}
\end{figure}

In Abb. \ref{fig:esr} sehen Sie die View zum Einlesen von ESR-Dateien. Ihre Bank wird Ihnen, wenn Sie die entsprechende Vereinbarung unterzeichnet haben, ESR-Dateien zum Abholen per Internet bereitstellen. Diese ESR-Dateien enthalten die Zahlungseingänge zu Ihren Rechnungen. Elexis kann die ESR-Dateien einlesen und die bezahlten Beträge automatisch den entsprechenden Rechnungen gutschreiben.


	
\chapter{Traitement de texte}
	% *******************************************************************************
% * Copyright (c) 2007 by Elexis
% * All rights reserved. This document and the accompanying materials
% * are made available under the terms of the Eclipse Public License v1.0
% * which accompanies this distribution, and is available at
% * http://www.eclipse.org/legal/epl-v10.html
% *
% * Contributors:
% *    G. Weirich - initial implementation
% *
% *  $Id: einleitung.tex 4904 2009-01-03 17:58:33Z rgw_ch $
% *******************************************************************************
% !Mode:: "TeX:UTF-8" (encoding info for WinEdt)

\section{Introduction}
Pour établir des lettres , ordonnances, certificats etc. Elexis utilise de façon standardisée un logiciel valable : OpenOffice
Ceci ne doit pas forcément être la seule solution car le traitement de texte est appliqué par Elexis en forme de Plugin. On pourrait donc aussi laisser créer un Plugin pour Microsoft\texttrademark{}Office\texttrademark{}  ou n'importe quel autre logiciel de traitement de texte. Nous nous limitons ici par contre à l'OpenOffice qui est le logiciel de référence pour Elexis.
L'origine de OpenOffice se trouve dans StarOffice qui avait été développé dans les années 80 et qui représente aujourd'hui une Office-Suite en analogie à Microsoft Office avec la différence qu'il s'agit d'un produit open source disponible pour plusieurs systèmes d'exploitation.
Dans l'installateur de la version Windows de Elexis une version adaptée de OpenOffice.org est intégrée. Sous Linux on peut se servir de la version OpenOffice qui est normalement intégrée dans Linux. Pour Macintosh l'intégration ne fonctionne malheureusement pas encore. Faites attention de n'installer q'une seule version de OpenOffice sur votre système car sinon ceci pourrait provoquer des conflits entre les versions.


\medskip

Après l'installation de OpenOffice et de Elexis il faut que les deux programmes 'fassent connaissance'. Pour ceci il faut configurer le Plugin de texte dans Elexis de sorte qu'il ait accès sur OpenOffice.(Le Plugin
\glqq NOA-Text\footnote{Nous utilisons le module 'Nice Office Access' de \href{http://www.ubion.org}{www.ubion.org}
 pour intégrer OpenOffice}\grqq{}doit être installé, chose qui est normalement d'emblée le cas.

Séléctionnez dans le menu  \textsc{Fichier - Options} Vous trouvez dans la liste à gauche : \textsc{Traitement de texte}
Il y apparaît un boite de dialogue comme en \ref{fig:text1}. Choisissez là  \glqq
%\usepackage{graphics} is needed for \includegraphics
\begin{figure}[htp]
\begin{center}
  \includegraphics{images/text1}
  \caption{Konfiguration des Textplugins}
  \label{fig:text1}
\end{center}
\end{figure}

NOA-Text\grqq{}. Après vous choisissez dans la liste à gauche OpenOffice.org (cf Fig. \ref{fig:text2}
En cliquant sur \glqq définir\grqq{} vous définissez le chemin d'accès à votre installation OpenOffice.
%\usepackage{graphics} is needed for \includegraphics
\begin{figure}[htp]
\begin{center}
  \includegraphics{images/text2}
  \caption{Konfiguration der OpenOffice-Installation}
  \label{fig:text2}
\end{center}
\end{figure}
Vous cliquez ensuite dans la boite de dialogue qui s'ouvre sur OpenOffice.org et cliquez sur 'terminer'.
Dans la boite de dialogue encore ouverte vous cliquez sur 'ok'. OpenOffice devrait être à disposition dès le prochain démarrage de Elexis. Lors de la première utilisation vous devez encore accepter les conditions de licence de OpenOffice.org.)




	% *******************************************************************************
% * Copyright (c) 2007 by Elexis
% * All rights reserved. This document and the accompanying materials
% * are made available under the terms of the Eclipse Public License v1.0
% * which accompanies this distribution, and is available at
% * http://www.eclipse.org/legal/epl-v10.html
% *
% * Contributors:
% *    G. Weirich - initial implementation
% *
% *  $Id: vorlagen.tex 2821 2007-07-16 14:51:35Z rgw_ch $
%  *******************************************************************************
% !Mode:: "TeX:UTF-8" (encoding info for WinEdt)


 \section{Vorlagen}
 \label{textvorlagen}
 \index{Textformatvorlagen}\index{Briefvorlagen}
In Elexis erstellte Dokumente basieren immer auf bestimmten Vorlagen. Eine
Vorlage enthält einerseits das Aussehen des Dokuments (Briefkopf etc.),
andererseits bestimmte Platzhalter, in die beim Erstellen des Dokuments dann die
entsprechenden Daten eingefügt werden.

Eine Vorlage ist einfach ein mit OpenOffice erstelltes Dokument mit dem
gewünschten Aussehen. Platzhalter werden als gewöhnlicher, in eckige Klammern
gesetzter Text der Form [Datentyp.Feld] eingetragen, wie z.B.
[Patient.Vorname]. Eine Auflistung der möglichen Platzhalter finden Sie auf
Seite \pageref{Platzhalter}

Es gibt zwei Typen von Vorlagen:
\begin{itemize}
  \item {Systemvorlagen} sind Vorlagen, die für bestimmte Programmfunktionen
  benötigt werden. So kann beispielsweise ein Rezept nur auf der Basis einer
  Systemvorlage namens \glqq Rezept\grqq{} gedruckt werden. Systemvorlagen
  müssen einen bestimmten Namen haben (eben z.B. \glqq Rezept\grqq{}), und
  meistens an einer Stelle einen speziellen Platzhalter, der angibt, wo der
  Inhalt eingefügt werden soll.
  \item {Benutzervorlagen} sind Vorlagen, die beliebig erstellt und benannt
  werden können, und die beliebige Felder (oder gar keine) enthalten können.
  Benutzervorlagen können beispielsweise für Konsiliarberichte, Zuweisungen
  etc. erstellt werden.
\end{itemize}


\subsection{Systemvorlagen}
\label{systemvorlagen}
Folgende Systemvorlagen werden im Basis-System genutzt (Plugins können ggf. auch
eigene Systemvorlagen definieren):
\begin{itemize}

  \item {Rezept} Ein Rezept, meist auf A5 oder A6 gedruckt. Die Gestaltung
  erfolgt nach eigene Geschmack bzw. gemäss gesetzlichen Vorgaben, falls
  vorhanden. An der Stelle, wo später die Medikamente eingefügt werden soll,
  muss ein Platzhalter [Rezeptzeilen] stehen.
  \item {AUF-Zeugnis} Ein Arbeitsunfähigkeitszeugnis, ebenfalls gemäss örtlichem
  Usus frei gestaltbar. Die Eckdaten können mit den Platzhaltern [AUF.von],
  [AUF.bis] und [AUF.Prozent] eingetragen werden.
  \item {Laborblatt} Dies dient zum Ausdrucken der im System vorhandenen
  Laborwerte. Das Blatt ist frei gestaltbar, an einer Stelle muss der
  Platzhalter [Laborwerte] stehen.
  \item {Einnahmeliste} Eine Medikamentenliste für den Patienten. Frei
  gestaltbar, an einer Stelle muss der Platzhalter [Medikamentenliste] stehen.
  \item {Bestellung} Eine Bestellung zur Übermittlung per Brief, Fax oder Mail.
  Frei gestaltbar, an einer Stelle muss der Platzhalter [Bestellung] stehen.
  \item {AgendaListe} Ein Ausdruck der Agenda eines Bereiches für einen Tage.
  Frei gestaltbar, an einer Stelle muss der Platzhalter [Termine] stehen
  \item {Abrechnungsliste} Eine Liste aller in einem bestimmten Zeitraum
  erfolgten Abrechnungen (s. \pageref{fig:konnd}). Frei gestaltbar, an einer
  Stelle muss der Platzhalter [Liste] stehen.
  \item {Terminkarte} Eine Liste der Termine eines Patienten. Frei gestaltbar,
  an einer Stelle muss der Platzhalter [Termine] stehen.
  \item {Tarmedrechnung\_EZ} (Alle Tarmedrechnung\_xx Vorlagen sind vom
  Arzttarife-Schweiz-Plugin beigesteuert) Rechnung mit Einzahlungsschein, meist
  der beim Patienten verbleibende Teil. Muss auf einem A4-Blatt sein. Die oberen
  zwei Drittel sind frei gestaltbar, an einer Stelle muss der Platzhalter
  [Leistungen] stehen- Das untere Drittel muss frei bleiben, dort wird der
  Einzahlungsschein gedruckt.
  \item {Tarmedrechnung\_M1} Erste Mahnung mit Einzahlungsschein. Gestaltung s. Tarmedrechnung\_EZ.
  \item {Tarmedrechnung\_M2}Zweite Mahnung.
  \item {Tarmedrechnung\_M3}Dritte Mahnung.
  \item {Tarmedrechnung\_S1}Erste Seite des Tarmed-Formulars. Die Gestaltung ist
  fix vorgegeben, nur die persönlichen Daten können geändert werden (Dabei darf
  das Layout nicht verschoben werden).
  \item {Tarmedrechnung\_S2} Folgeseite des Tarmed-Formulars. Die Gestaltung ist
  ebenfalls fix vorgegeben.
\end{itemize}


\subsection{Benutzervorlagen}
Benutzervorlagen können beliebig erstellt und benannt werden.

	% *******************************************************************************
% * Copyright (c) 2007 by Elexis
% * All rights reserved. This document and the accompanying materials
% * are made available under the terms of the Eclipse Public License v1.0
% * which accompanies this distribution, and is available at
% * http://www.eclipse.org/legal/epl-v10.html
% *
% * Contributors:
% *    G. Weirich - initial implementation
% *
% *  $Id$
% *******************************************************************************
% !Mode:: "TeX:UTF-8" (encoding info for WinEdt)
% Dieses Dokument enthält die Dokumentation der Platzhalter für Datenfelder

\section{Platzhalter für Datentypen}
\label{Platzhalter}
Diese Platzhalter können beispielsweise in Textdokumentvorlagen eingesetzt
werden, in eckige Klammern gesetzt, also z.B.[Patient.Name].
Sie können auch in der View 'Datenansicht' als Datenquellen eingesetzt werden.
Die folgende Liste erhebt keinen Anspruch auf Vollständigkeit; insbesondere
können durch Plugins zusätzliche Felder eingeführt werden.
\begin{description}
  \item [Anwender.Name] Name des aktuell eingeloggten Anwenders
  \item [Anwender.Vorname] Vorname des aktuell eingeloggten Anwenders
  \item [Anwender.Titel] Titel des aktuell eingeloggten Anwenders
  \item [Anwender.Kuerzel] Initialen des aktuell eingeloggten Anwenders
  \item [Anwender.Label] Login-Name des aktuell eingeloggten Anwenders
  \item [Mandant.Name,Vorname,Titel,Kuerzel,Label] dieselben Felder, wie bei Anwender, bezogen auf den aktuell
  aktiven Mandanten. 	
  \item [Mandant.EAN] Die EAN des aktuell aktiven Mandanten. Nur vorhanden, wenn
  das Plugin Arzttarife Schweiz geladen ist.
  \item [Mandant.KSK] Die KSK (bzw. ZSR)-Nummer des aktuell aktiven Mandanten.
  Nur vorhanden, wenn das Plugin Arzttarife Schweiz geladen ist.
  \item [Patient.Name,Vorname,Titel] Name etc.des aktuell selektierten Patienten
  \item [Patient.Geburtsdatum] Geburtsdatum des aktuell selektierten Patienten
  \item [Patient.PatientNr] Die interne Patientennummer des aktuell selektierten Patienten.
  \item [Patient.Diagnosen] Diagnosen wie auf dem Titelblatt genannt
  \item [Patient.Allergien] Allergien wie auf dem Titelblatt
  \item [Patient.Strasse, Patient.Plz, Patient.Ort] Adresse des aktuell
  selektierten Patienten.
  \item [Patient.PersAnamnese] Anamnese wie auf dem Titelblatt
  \item [Patient.Telefon1, Patient.Telefon2, Patient.Natel] Telefonnummern
  \item [Patient.Medikation] Aktuelle Fixmedikation des aktuell selektierten
  Patienten
  \item [AUF.von] Beginn der aktuell ausgewählten Arbeitsunfähigkeit
  \item [AUF.bis] Ende der aktuell ausgewählten Arbeitsunfähigkeit
  \item [AUF.Prozent] Prozentsatz der aktuell ausgewählten AUF
  \item [AUF.Grund] Grund der aktuellen AUF (Unfall Krankheit)
  \item [AUF.Zusatz] Allfälliger Zusatztext
  \item [Fall.ArbeitgeberName] Name des Arbeitgebers, wenn eingetragen
  \item [Fall.Kostenträger] Bezeichnung des Kostenträgers
  \item [Fall.Versicherungsnummer] Versicherungsnummer, wenn angegeben
  \item [Rechnung.RnNummer] Nummer der aktuellen Rechnung
  \item [Rechnung.RnDatum] Rechnungsdatum
  \item [Rechnung.RnDatumVon] Datum der ersten Konsultation dieser Rechnung
  \item [Rechnung.RnDatumBis] Datum der letzten Konsultation dieser Rechnung
  \item [Konsultation.Datum] Datum der aktuell ausgewählten Konsultation
  \item [Konsultation.Eintrag] Text der aktuell ausgewählten Konsultation
  \item [Konsultation.Diagnose] Diagnosen der aktuell ausgewählten Konsultation

\end{description}

\section{Geschlechtsspezifische Formulierungen}
Auch dies sind eine Art Platzhalter, welche aus alternativen Formulierungen bestehen:

\begin{verbatim}
    [Datenobjekt:mw:Formulierung Mann/Formulierung Frau]
    oder
    [Datenobjekt:wm:Formulierung Frau/Formulierung Mann]
    oder
    [Datenobjekt:mwn:Formulierung Mann/Formulierung Frau/Formulierung neutral]
\end{verbatim}

Wenn die das Datenobjekt eine  männliche Person beschreibt, wird die Formulierung Mann verwendet, wenn es eine weibliche Person bezeichnet, die Formulierung Frau, wenn es gar keine Person bezeichnet oder wenn das Geschlecht nicht eingetragen ist, die Formulierung neutral.

\medskip

Beispiele
\begin{itemize}
    \item Sehr [Adressat:mwn:geehrter Herr [Adressat.Name]/geehrte Frau [Adressat.Name]/geehrte Damen und Herren]
    \item Bitte um Aufgebot [Patient:wm:der obengenannten Patientin/des obengenannten Patienten]
\end{itemize}

\section{Daten aus externen Plugins}
\label{datenfelder_extern}\index{Datenfelder!extern}
Wenn vom jeweiligen Hersteller vorgesehen, können auch Daten externer Plugins in Platzhalter eingebunden werden. Dabei ist allerdings eine etwas andere Syntax zu beachten, als bei den 'gewöhnlichen' Platzhaltern. Dies kommt daher, dass sich Platzhalter sonst immer auf das aktuell selektierte Objekt eines bestimmten Typs beziehen, was ja bei Daten aus externen Plugins nicht möglich ist, da Elexis deren Daten nicht kennt. Bei externen Daten gibt es dagegen:
\begin{itemize}
\item Den Titel des Plugins, der die bereitstellt
\item Den Namen des Datenobjekts 
\item Eine Auswahl der Werte mit diesem Namen (Es könnten ja beispielsweise Serien von mehreren Daten sein. die genauen Optionen dieses Parameters hängen vom bereitstellenden Plugin ab.
\item Eine Bezeichnung der Daten, die bereitgestellt werden sollen
\item Möglicherweise Parameter, die für die Anwahl dieser Daten benötigt werden. Auch dieser Parameter hängt vom bereitstellenden Plugin ab.
\end{itemize}
Dementsprechend besteht ein Platzhalter für Daten aus Plugins aus vier bis fünf Teilen, die durch : getrennt sind.
\begin{verbatim}
    [pluginName:objektName:auswahl:daten]  oder
    [pluginName:objektName:auswahl:daten:parameter]
\end{verbatim}

\medskip

Als Beispiele sei auf die Beschreibung des Plugins 'Befunde' hingewiesen (\ref{befunde}, S. \pageref{befunde}).



\chapter{Fonctionnement multi-client}
	% *******************************************************************************
% * Copyright (c) 2007 by Elexis
% * All rights reserved. This document and the accompanying materials
% * are made available under the terms of the Eclipse Public License v1.0
% * which accompanies this distribution, and is available at
% * http://www.eclipse.org/legal/epl-v10.html
% *
% * Contributors:
% *    G. Weirich
% *
% *  $Id: multiuser.tex 6282 2010-04-19 19:24:51Z niklausgiger $
% *******************************************************************************
% !Mode:: "TeX:UTF-8" (encoding info for WinEdt)

\section{Einführung}
\index{Gemeinschaftspraxis}\index{Gruppenpraxis}\index{Gesundheitszentrum}
Elexis ist von Haus aus auf Mehrbenutzer- und Mehrmandantenbetrieb ausgelegt.
Es gibt dabei keine Beschränkung der Zahl der Anwender oder Mandanten. Es sind
auch keine besonderen Vorkehrungen notwendig, um vom Einzel- auf den
Mehrbenutzerbetrieb umzustellen. In diesem Kapitel müssen deshalb lediglich
einige Konzepte vorgestellt werden, welche im Mehrbenutzer- und
Mehrmandantenbetrieb nützlich sein können.

\section{Gruppen/Rollen und Rechte}
\label{sec:gruppen}
\index{Gruppen}\index{Rollen}
Sobald mehr als ein Anwender auf den gemeinsamen Datenbestand zugreift, stellt
sich die Frage, welche Daten jeder Anwender lesen, schreiben oder löschen können
soll. Im Allgemeinen gilt das Prinzip, dass jeder das kann, was er
zur Erledigung seiner Arbeit braucht, aber möglichst nicht mehr. Dies reduziert
die Möglichkeit von Fehlbedienungen und erleichtert im Problemfall die Suche
nach den Ursachen.

Bei Elexis kann jeder Anwender Mitglied eine oder mehrere \textit{Rollen}\footnote{Rollen wurden früher (und werden stellenweise noch jetzt) als 'Gruppen' bezeichnet. Wir wechseln mit dieser Version auf die gebräuchlichere Bezeichning 'Rolle'.} ausüben. Eine Rolle ist ein frei wählbarer Bezeichner, der keine weitere Funktion hat als
die, Anwender mit gemeinsamen Rechten zusammenzufassen. In einer grösseren
Arztpraxis könnte es beispielsweise die Rollen 'MPA', 'Labor', 'Arzt' und
'Buchhaltung' geben, in einer Einzelpraxis vielleicht nur die Rollen 'MPA' und
'Arzt'. Elexis bringt als Standardausstattung die Rollen 'Anwender' und 'Admin'
mit.
\index{Rechte}\index{Anwender!Rechte}\index{Zugriffsrechte}

Sobald Anwender und Rollen definiert sind, können die Rechte verteilt werden.
Dieser Schritt ist unter Standardbedingungen nicht unbedingt notwendig, da die
normalerweise existierenden Rechte oft schon für diese Zwecke richtig definiert
sind.

Die Verteilung von Zugriffsrechten erfolgt über das Menü \textsc{Datei -
Einstellungen - Gruppen und Rechte - Zugriffsteuerung} (S. \ref{fig:zugriff}).
%\usepackage{graphics} is needed for \includegraphics
\begin{figure}[htp]
\begin{center}
  \includegraphics{images/zugriff}
  \caption{Zugriffssteuerung}
  \label{fig:zugriff}
\end{center}
\end{figure}

Wie Sie erkennen können, sind die Rechte benannt und hierarchisch angeordnet -
Das Recht \glqq Vorlagen ändern\grqq{} ist offensichtlich dem Recht \glqq
Dokumente\grqq{} untergeordnet. Im unteren Abschnitt des Fensters sehen Sie alle
im System vorhandenen Gruppen und Anwender.
Die Regel ist nun so:
\begin{itemize}
  \item Wer ein Recht hat, hat implizit auch alle diesem Recht untergeordneten
  Rechte.
  \item Wer ein Recht hat, hat aber \textit{nicht} automatisch die diesem Recht
  übergeordneten Rechte.
  \item Jeder hat die Rechte aller Gruppen, denen er angehört, und zusätzlich
  diejenigen Rechte, die ihm individuell zugesprochen wurden.
  \item Wer zur Gruppe \glqq Admin\grqq gehört, hat sämtliche Rechte, auch wenn
  sie ihm nicht explizit erteilt wurden.
  \item Wer das Recht \glqq Zugriff - Rechte erteilen\grqq hat, kann sebst
  Zugriffsrechte verwalten, auch wenn er kein Administrator ist.
\end{itemize}

Man kann Zugriffrechte also an Gruppen/Rollen oder an einzelne Anwender erteilen. Jedes
Recht kann an keine oder beliebig viele Rollen und/oder Anwender erteilt
werden. Dies geschieht, indem man zunächst das Recht im oberen Feld anklickt,
und dann im untern Feld m it der linken Maustaste oder [STRG]+linke Maustaste
eine oder mehrere Gruppen oder Anwender auswählt.

\textbf{Wichtig:}Beherzigen Sie bitte folgende Grundregel: Niemand, auch der
Chef nicht, sollte für die tägliche Arbeit als
Admin angemeldet sein oder die Rolle 'Admin' haben. Die Gefahr, mal schnell einen Fehler zu machen, der
wichtige Daten löscht, ist zu gross. Erstellen Sie als Praxisoberhaupt für sich
selbst zwei verschiedene Benutzerkonten (accounts):
\begin{itemize}
  \item Einen normalen Benutzer (z.B. Dr. Test), der die Rolle oder
  Ärzte, aber jedenfalls nicht Admin hat, und der genau diejenigen Recht
   hat, die er für die Alltagsarbeti braucht.
  \item Einen Administratorbenutzer (z.B. Praxisadmin), der die Rolle Admin
  hat, und dessen Passwort Sie strikt für sich behalten, auch wenn Sie
  allen Mitarbeitern und Mitarbeiterinnen voll vertrauen. Logen Sie sich mit
  diesem account nur dann ein, wenn Sie wirklich mal Dinge tun müssen, die mit
  dem normalen Account nicht möglich sind.
\end{itemize}

\section{Definierbare Anwendereinstellungen}
In einer kleinen Praxis wird oft die Möglichkeit von Elexis geschätzt, den
Arbeitsplatz ganz individuell einzurichten. Die MPA kann ihren Bildschirm mit
anderen Farben und anderen Layouts gestalten, als die Ärztin.
In einem grösseren Betrieb, in dem vielleicht auch die Anwender zwischen
verschiedenen Arbeitsplätzen wechseln müssen, ist dagegen oft eine einheitliche
Gestaltung für alle, oder zumindest eine einheitliche Gestaltung je
Funktionsgruppe erwünscht. Elexis kommt diesem Wunsch mit konfigurierbaren
Farben und Layouts entgegen.

\subsection{Individuelle Programmeinstellungen}
In den Unterseiten von \textsc{Datei - Einstellungen - Anwender} können Sie
Farbdesigns, Tastenkürzel und Art der Schnellstartleiste definieren. Sie können
diese Einstellungen unter einem frei definierbaren Namen speichern. Geben Sie
dazu einen Namen ein und klicken Sie auf \glqq Einstellungen speichern
nach\grqq{}.

Von einer anderen Arbeitsstation oder einem anderen Benutzeraccount aus können
Sie dieselben Einstellungen unter diesem Lamen wieder laden. Geben Sie dazu den
Namen ein und klicken Sie auf \glqq Einstellungen laden von\ldots\grqq{}.

\subsection{Definierbare Perspektivenlayouts}
Die Fensteranordnung, also die Perspektive, mit der Sie normalerweise arbeiten,
lässt sich ja auch individuell einstellen. Diese EInstellung ist an den
Arbeitsplatz gebunden (da sie ja auch von der Art des angeschlossenen Monitors
abhängt). Sie können aber auch die Perspektivenanordung unter einem Namen
speichern, um sie auf einen anderen Arbeitsplatz zu replizieren. Geben Sie
einfach den gewünschten Namen ein  und klicken Sie auf \glqq
Arbeitsplatzeinstellungen speichern nach\ldots\grqq{}.

Um die Perspektive auf einen anderen Arbeitsplatz zu kopieren, geben Sie dort
unter \textsc{Datei - Einstellungen - Anwender} den vorher vergebenen Namen ein
und klicken auf \glqq Arbeitsplatzeinstellungen laden von\grqq{}.



	
\part{Annexes}
\appendix

\chapter{Configuration du système requise}
	% *******************************************************************************
% * Copyright (c) 2007 by Elexis
% * All rights reserved. This document and the accompanying materials
% * are made available under the terms of the Eclipse Public License v1.0
% * which accompanies this distribution, and is available at
% * http://www.eclipse.org/legal/epl-v10.html
% *
% *  $Id: voraussetzungen.tex 3094 2007-09-04 10:03:03Z rgw_ch $
%
%*******************************************************************************
% !Mode:: "TeX:UTF-8" (encoding info for WinEdt)

\section{Prérequis minimale au hardware}
\label{systemvoraussetzungen}
\begin{itemize}
 \item Un PC à peu près actuel (cadence min. de 1GHz, 512 MB RAM min , 1GB recommandé) disque dur de 1GB min.
\item  Carte graphique 1024 x 768 pixel min (1280 x 1024 recommandé) et un écran convenable (par ex. 17 pouces TFT ou plus).
\item Recommandé : Imprimante avec alimentation bac 1 pour du papier A5 (Ordonnances, Certificats) et bac 2 et eventuellement bac 3 pour du papier A4 (Lettres, Factures)
\item Recommandé : Imprimante d'étiquettes.
\item Recommandé : Drive externe pour sauvegarde .
\item Recommandé : Accès Internet protégé par une \textbf{Hardware}-Firewall (une protection par simple Personal Firewall est  \textbf{explicitement déconseillée}) (voir p. \pageref{sicherheit}).
\end{itemize}

\section{Systèmes d'exploitation supportés}
Elexis fonctionne en principe dans tout les systèmes d'exploitation pour lesquels une Environment Version 1.5 ou plus avancée du Java Runtime existe . En particulier ce sont les systèmes suivants :
\begin{itemize}
\item  Windows 2000, XP, Vista
\item  Macintosh OS à partir de 10.4 (Tiger)  \footnote{Sous MacOS-X une intégration OpenOffice n'est malheureusement pas possible.}
\item Linux (SuSE à partir de 9.3 ou Xubuntu/Kubuntu à partir de 6.06)\footnote{Attention : Sous Linux avec Gnome-Desktop (comme Ubuntu) l'intégration OpenOffice ne fonctionne pas, par contre sous KDE (Kubuntu) et Xfce(Xubuntu) ça fonctionne. }
\end{itemize}
Dans ces systèmes d'exploitation Elexis peut être installé directement avec une des versions complètes mises à disposition. L'installation sur d'autres systèmes nécessitera plus ou moins de travail 'manuel'. Veuillez prendre en considération que nos offres forfaitaires et contrats de maintenances ne couvrent que les systèmes qui remplissent les conditions mentionnés ci-dessus en ce qui concerne la Hardware et le système d'exploitation. 

\chapter{Transformer la version d'évaluation en version complète}
	% *******************************************************************************
% * Copyright (c) 2007 by Elexis
% * All rights reserved. This document and the accompanying materials
% * are made available under the terms of the Eclipse Public License v1.0
% * which accompanies this distribution, and is available at
% * http://www.eclipse.org/legal/epl-v10.html
% *
% * Contributors:
% *    G. Weirich
% *
% *  $Id: vollversion.tex 4911 2009-01-05 17:56:39Z rgw_ch $
% *******************************************************************************
% !Mode:: "TeX:UTF-8" (encoding info for WinEdt)

\label{vollversion}
\index{version complète}
Si vous avez apprécié la version d'évaluation de Elexis vous de devez en fait rien faire d'autre pour recevoir une version complète : La version d'évaluation \textbf{est} une version complète. La seule différence se trouve dans la base de données.

Pour transformer une version d'évaluation en version complète il faut procéder de façon suivante :
\begin{itemize}
  \item \glqq Installer  \grqq{}le moteur de base de données de la version complète. (\ref{dbengine})
  \item Effacer la base de données de la version d'évaluation (si elle était installée avant).
  \item Lier Elexis avec ce moteur de base de données.  (\ref{connect})
  \item Lier éventuellement avec une version OpenOffice préexistante  \ref{config:ooo}
  \item Introduire la version actuelle des données de base.
  \item Etablir la configuration de base pour votre cabinet médical.

\end{itemize}
Ces pas d'installation ne sont cependant pas si banaux. La mise à jour de l'ensemble des donnés et surtout l'établissement d'un configuration de base convenable pour votre cabinet médical peuvent être assez laborieux. A ce sujet Elexis ne peut pas se distinguer d'autres logiciels de gestion d'un cabinet médical car la complexité de la tâche reste la même pour tous.
Si vous pensez accomplir ce travail vous-même, il faudra réserver suffisamment de temps (une journée au minimum) et suivre ce chapitre étape par étape.
Si vous n'êtes pas sûr de pouvoir le faire nous vous conseillons d' acheter l'installation et la configuration de base y inclus une heure d'instruction pour votre personnel du cabinet.
Ce manuel est par la force des choses un peu penché sur la Suisse. Pour d'autre pays probablement pas toutes les informations sont correctes respectivement utiles.
\bigskip

Pour compléter nous attirons votre attention sur le fait que cette documentation peut contenir des fautes et pourrait être incomplète. Nous ne pouvons pas assumer la responsabilité si vous subissez des dégâts matériels ou immatériels suite à une configuration défectueuse ou une documentation incorrecte. Nous vous conseillons de minutieusement tester le tout et si possible aussi de faire des contrôles manuels (par exemple des factures) avant de travailler \glqq véritablement \grqq{} avec votre configuration. 

\section{Ce dont vous avez besoin}
Avant de commencer la configuration vous devez collectionner les programmes et données suivants :
\begin{itemize}
  \item Kit d'installation pour votre serveur de base de données
  \item Noms, Noms d'utilisateur et mots de passe pour tous qui devaient pouvoir utiliser Elexis
  \item Noms, Numéros du Concordat , Numéros EAN, Coordonnées bancaires ou pour le compte de chèque postal, numéros de participant BVR de tous les mandants.
  \item Relevé actualisé des valeurs du point de votre canton
  \item Une conception de vos en-têtes pour lettres, ordonnances, certificats.
  \item Une liste des examens de laboratoire effectués dans votre laboratoire du cabinet.
  \item Liste des médicaments, CIM-10, Tarmed , Liste des analyses, MiGel dans la mesure où vous en avez besoin.
  \item Les numéros EAN des assurances maladies et accidents
\end{itemize}

\section{Installation du moteur de base de données}
\label{dbengine}
L'installation de Elexis consiste en deux parties : Un \textit{Serveur} sur lequel sont installés les données et un ou plusieurs \textit{Clients}, qui ont accès aux données et qui permettent de les visualiser et de les modifier. Le serveur et les clients peuvent se trouver sur le même ou sur des différents ordinateurs.
Un\textit{\textit{Serveur}} au sens large est un ordinateur à part sur lequel fonctionnent un ou plusieurs logiciels serveur.

Elexis peut utiliser (en principe une quelconque) base de données qui se laisse utiliser selon le standard de l'industrie JBBC comme logiciel de serveur. L'installation automatique est configurée d'avance pour les systèmes de base de données suivants :
\begin{itemize}
\item MySQL (\href{http://www.mysql.com}{www.mysql.com}): Il s'agit de la base de données le plus répandue dans l'Internet. La majorité des applications basées sur une base de données qu'on trouve dans le Web, utilisent en arrière-plan un serveur MySQL. Un serveur MySQL coute environs Fr 750.-pour une utilisation à des fins commerciales. A des fins privés il est gratuit.

\item PostgreSQL (\href{http://www.postgresql.org}{www.postgresql.org}): Il s'agit d'un serveur de base de données OpenSource. Il maîtrise un jeu d'instructions plus large que MySQL mais est considéré comme un peu plus lent que celui-ci. Cependant cela ne devrait pas jouer un rôle pour Elexis car les test de rapidité se font normalement sous les conditions de plusieurs milliers d'accès par seconde, un situation qui pourrait se produire que dans des très rares cas dans un cabinet médical. PostgreSQL est gratuit pour toutes les formes d'utilisation.

\item HSQLDB: Il s'agit d'une base de données OpenSource qui est écrite en Java. Elle peut être utilisée soit en tant que serveur indépendant soit intégrée dans le logiciel. HSQL est un peu plus lent que les deux systèmes mentionnés précédemment mais pour des environnements petits comme ceux d'un cabinet médical éventuellement suffisant. Cependant il faut faire spécifiquement attention en ce qui concerne la sauvegarde des données car une panne d'ordinateur (ou même le fait d'éteindre l'ordinateur de façon improviste) peut rendre la base de données inutilisable. HSQL est gratuit.
\end{itemize}

\textbf{Attention}: Nous \textit{déconseillons explicitement} l'utilisation du serveur de base de donnée HSQL utilisé pour la version d'évaluation de Elexis pour le travail au cabinet médical (risque de perte des données !) Nous vous conseillons plutôt d'installer MySQL ou PostgreSQL.
\begin{itemize}
 \item Comme serveur nous vous conseillons de choisir de préférence un ordinateur sur lequel personne ne travaille directement. Qu'il y ait encore d'autres logiciels de serveur installés comme par ex. pour les mails, fax, imprimantes etc, ne joue aucun rôle. Attention! Si vous n'installez pas le serveur de base de donnée sur un ordinateur réservé à cette fonction mais sur un poste de travail il doit avoir au moins 1 GB de RAM et 2 GB seraient préférable.
 \item Installez là le serveur de base de donné de votre choix. (Nous préconisons mysql ou PostgreSQL)
 \item Créez dans la base de donnée un 'user-account' nommé : elexisuser
 \item Créez une base de donnée (vide) avec le nom elexis sur laquelle 'elexisuser' a un accès illimité.
 \item Décidez-vous absolument pour une stratégie de sauvegarde des données efficace et fiable. Plus d'information la concernant ci-après.
 \item La configuration ultérieure se fait depuis les 'clients'. Pour le travail de tout les jours le serveur ne nécessite pas forcément un écran et un clavier et peut se trouver à un endroit frais du cabinet ou même à la cave.
\end{itemize}

\textbf{Important!}
\textit{N'oubliez jamais la sauvegarde des données!}

Elexis enregistre toutes les données dans cette base des données. Une destruction de cette base de donnée n'est pas du tout impossible. Une interruption du courrant peut \glqq  choper le disque dur au tendon d'Achille\grqq{} un dommage mécanique peut détruire des secteurs importants du disque dur et les rendre illisibles, une faute d'un logiciel peut effacer les données et un virus peut se défouler sur vos données. Il y a des multiples stratégies de sauvegarde des données. Nous vous présenterons quelques unes en ce qui suit :

\begin{description}
\item[ La Réplication  ] Certaines banques de données (comme par exemple MySQL à partir de la version 4.0) peuvent copier leurs données de façon constante vers un serveur qui se trouve sur un autre ordinateur. Puisque seulement les données qui changent sont transmis (en arrière-plan) ceci demande moins de capacité que ce qu'on pourrait croire. Cette méthode s'appelle la  \textit{Réplication} . En fin de compte on a deux bases de données identiques. Si le serveur se casse on peut dans un délai de quelques minutes choisir le deuxième ordinateur comme serveur et continuer le travail pratiquement sans interruption.
\item [La Machine virtuelle ] Un concept apparenté : On laisse tourner le serveur de la base de donnée sur un machine virtuelle spécifiquement réservée pour cela (par ex. de VMWare) et on sauvegarde de façon régulière toute la machine virtuelle. En cas de panne du serveur on peut également dans quelques minutes starter la machine virtuelle de sauvegarde sur le même ou n'importe quel autre ordinateur dans le réseau et continuer à travailler.
\item [Sauvegarde des données fréquente ] On peut laisser faire toutes les quelques minutes une sauvegarde automatisée (par ex. avec mysqldump) et sauvegarder de cette façon des données en plusieurs générations sur des différents supports informatiques. Cette méthode utilise le moins de ressources de toutes les méthodes mentionnées ici et crée les fichiers de sauvegarde les plus petits. En cas de panne du serveur par contre la remise en route prend plus de temps : Il faut d'abord démarrer le serveur de base de donnée sur un ordinateur de réserve et y mettre les fichiers sauvegardés pour ensuite adapter selon la configuration tout les 'clients' au nouveau serveur.
\end{description}

\section{Effacer la base de donnée de la version d'évaluation}
Si Elexis trouve lors du démarrage une base de donnée de la version d'évaluation le lien sera toujours fait avec cette base de donnée indépendamment des réglages de connexion que vous avez défini. Pour cette raison vous devez d'abord fermer Elexis et effacer ou renommer la base de donnée de la version d'évaluation qui se trouve dans le répertoire  \glqq demoDB\grqq{}dans le répertoire du programme Elexis. Après avoir effacé ou renommé ce répertoire vous pouvez redémarrer Elexis qui devrait être maintenant apte à se connecter à votre nouvelle base de donnée que vous avez installé préalablement.

\section{Créer un lien avec la base de données}
\label{connect}
Démarrez Elexis et choisissez dans le menu  Fichier -> Connexion.

\includegraphics[width=0.9\textwidth]{images/verbindung11.png}
% verbindung11.png: 1024x768 pixel, 72dpi, 36.12x27.09 cm, bb=0 0 1024 768



Entrez le type de base de donnée (ici mysql), l'adresse du serveur (ici 192.168.0.2) ou son nom Internet (par ex. testserver.elexus.ch) de même que le nom de la base de donnée (ici elexistest) et cliquez sur
\textit{suite}.

\includegraphics[width=0.9\textwidth]{images/verbindung12.png}
% verbindung12.png: 1024x768 pixel, 72dpi, 36.12x27.09 cm, bb=0 0 1024 768
Introduisez dans la ligne supérieure le nom d'utilisateur pour la base de donnée (ici testuser) et dans la ligne inférieure le mot de passe (ici testelexis) et cliquez sur \textit{terminer}.

\includegraphics[width=0.9\textwidth]{images/verbindung13.png}
% verbindung13.png: 1024x768 pixel, 72dpi, 36.12x27.09 cm, bb=0 0 1024 768

Il y aura quelques messages d'erreur qui apparaîtront mais vous pouvez les ignorer en les fermant. Enfin il faut redémarrer Elexis.

\includegraphics[width=4in]{images/verbindung14.png}
% verbindung14.png: 1024x768 pixel, 72dpi, 36.12x27.09 cm, bb=0 0 1024 768

Vous pouvez maintenant faire une login dans le nouveau système Elexis avec le nom \textit{Administrator} et le mot de pass \textit{admin}.

\section{Lier avec Open-Office}
\label{config:ooo}
Le fichier d'installation contient déjà une version complète de OpenOffice. Si vous voulez travailler avec celle-ci vous ne devez faire rien de plus. 
Si vous avez par contre déjà installé une autre version de OpenOffice sur votre ordinateur, la version apportée par Elexis pourrait provoquer des conflits. Procédez en ce cas là de façon suivante :


\begin{itemize}
\item Effacez le répertoire ooo dans le répertoire de Elexis (ou renommez-le).
\item Redémarrez Elexis
\item Allez sous  \textsc{Fichier - Options - NOAText} cliquez sur \glqq définir\grqq{} et choisissez le répertoire de l'installation existante de OpenOffice et cliquez sur \textit{appliquer}.

\end{itemize}
\section{Entrer les données de base}
\subsection{Tarmed}
\index{Tarmed}
Vous pouvez télécharger une base de donnée Microsoft Access depuis www.tarmedsuisse.ch
Intégrez cette base de donnée dans votre PC avec système d'exploitation Windows comme DSN système. Choisissez dans Elexis \textit{WINDOW - PESPECTIVE - OTHER - PRESTATIONS}. Sous \textit{Codes}, vous trouverez un onglet  \textit{Tarmed}. Dans le 'View-menu' (petit triangle en haut à droite) vous choisissez \textit{import} et introduisez la base de donnée que vous venez d'intégrer comme DNS système. Dépendant de la vitesse de l'ordinateur l'importation de toute la base de donnée Tarmed prendra entre une à 5 minutes.

\subsection{CIM-10 (ICD-10)}
\index{CIM-10 (ICD-10)}
\label{config:icd10}
Vous pouvez télécharger le catalogue CIM-10 de l'OMS en forme informatisée de:

http://www.icd10.ch/index.asp

Vous avez besoin de la \textit{version informatisée ASCII de la CIM-10 systématique de l'OMS } et des \textit{métadonnées systématiques 2006 de la CIM-10 de l'OMS ASCII}. Décompressez tout les deux fichiers zip dans le même dossier. Vous pouvez ignorer l'avertissement que vous êtes en train d'écraser un fichier. Choisissez dans Elexis  WINDOW - PERSPECTIVE - OTHER - PRESTATIONS. Sous 'Codes' vous trouverez un onglet \textit{CIM-10}. Dans le 'View-menu' (petit triangle en haut à droite) vous choisissez  \textit{import}. et introduisez le dossier dans lequel vous avez décompressé les deux fichiers.

\subsection{Médicaments et Medicals}
\index{médicaments}\index{Medicals}\index{SL}
Ces deux groupes ont l'origine dans la même base de donnée. Vous nécessitez la liste transfer.dat que vous pouvez abonner par exemple chez www.e-mediat.ch. Choisissez dans Elexis 'WINDOW - PERSPECTIVE - OTHER - ARTICLES'. Sous 'Articles' vous trouverez les onglets \textit{Medicals} respectivement \textit{Médicaments}. Dans le 'View-menu' (petit triangle en haut  à droite) vous choisissez 'import' et introduisez le chemin d'accès pour le dossier dans lequel vous avez mis le fichier Transfer.dat.

\subsection{Liste des analyses}
\index{Liste des analyses}\index{LA}\index{LFA}
Cette liste est publiée par l'OFSP et actuellement pour des raisons incompréhensibles seulement en format pdf. Ceci nous force de faire en plus un pas de conversion qui risque certaines fautes de transcription. Sous Windows c'est le logiciel TextFromPDF qui est capable de faire la conversion, sous Linux par exemple xpdf. Veuillez constituer par ces logiciels une version plaintext de la liste des analyses. Ensuite vous procédez de nouveau dans Elexis 'WINDOW - PERSPECTIVE - OTHER - PRESTATIONS'. Sous \textit{Codes} vous trouverez l'onglet 'Analyses'. Dans le 'View-menu' (petit triangle en haut à droite) vous choisissez 'import' et introduisez le chemin d'accès pour le dossier dans lequel vous avez mis le fichier converti de la liste des analyses.
\subsection{LiMA}
\index{LiMA}
La liste LiMA n'est fourni par l'OFSP, une fois de plus, qu'en format pdf de sorte qu'on doive d'abord péniblement déboîter le ficher pour faire ensuite une conversion . Pour cela vous utilisez de nouveau sous Windows le logiciel TextFromPDF, sous Linux par exemple xpdf. Puisque la structure n'est pas si facilement automatiquement analysable comme c'était le cas avec la liste des analyses, vous devez faire en plus un pas supplémentaire. Le fichier texte doit être transformé en tableau au format .csv qui contient les colonnes : code, texte, unité, prix. Pour cette transformation vous pouvez utiliser par exemple OpenOffice Calc ou Microsoft Excel. Ensuite vous procédez de nouveau dans Elexis 'WINDOW - PERSPECTIVE - OTHER - ARTICLES'. Sous 'ARTICLES' vous trouverez l'onglet \textit {MiGel = LiMA}. Dans le 'View-menu' (petit triangle en haut à droite) vous choisissez \textit{import} et introduisez le chemin d'accès pour le dossier dans lequel vous avez mis le fichier converti de la liste LiMA.

\section{Configuration de base}
\index{Configuration de base}
\label{grundkonfiguration}
La configuration de base se fait par les pas suivants :
\begin{itemize}
  \item Installer les mandants et utilisateurs
  \item Définir les paramètres du laboratoire
  \item Créer des modèles de texte
  \item Installer le module de facturation
\end{itemize}
\subsection{Installer les mandants et les utilisateurs}
Ouvrez la perspective \textit{Contacts},

\includegraphics[width=4.5in]{images/grundkonfkonta.png}
% grundkonfkonta.png: 1024x585 pixel, 72dpi, 36.12x20.64 cm, bb=0 0 1024 585
\begin{itemize}
 \item Introduisez sous \textit{Bezeichnung1 = désignation1} le nom du nouveau mandant ou du nouvel utilisateur et cliquez sur \textit{créer nouveau}.
 \item Cliquez ensuite sur l'entrée que vous venez d'introduire dans la liste en haut de la page et complétez les données dans la partie inférieure. Comme toujours chez Elexis, il n'est pas indispensable de remplir toujours tous les champs. Ensuite vous \textit{déterminez} si le contact introduit sera  \textit{mandant} ou \textit{utilisateur} (Un mandant est toujours aussi un utilisateur et les deux sont toujours des \textit{personnes}
 \item Lorsque vous avez introduit tout les mandants et utilisateurs, vous allez sous 'Fichier - Options' et vous trouverez sous 'groupes, droits et accès' l'onglet \textit{mandant}
\end{itemize}

\includegraphics[width=4in]{images/grundkonfmand.png}
% grundkonfmand.png: 580x549 pixel, 72dpi, 20.46x19.37 cm, bb=0 0 580 549
\begin{itemize}
 \item Introduisez les données nécessaires pour les mandants déjà installés. Introduisez comme  \textit{sigle} le nom d'utilisateur et comme mot de passe le mot de passe attribué au mandant. Tout le reste dépend du type du mandant.

 \item Ensuite vous allez sous 'Sécurité' - utilisateurs
\end{itemize}

Introduisez là pour tout les \index{utilisateurs}utilisateurs définis les données respectives. N'oubliez pas de définir (sous utilisateur) pour chaque utilisateur un mandant de référence  (\textit{pour mandant}). Un mandant de référence (normalement lui-même) devrait être établi aussi pour des mandants déjà introduits (que vous trouvez aussi sous "utilisateurs" car un mandant est toujours aussi un utilisateur). Le mandant de référence définit pour qui l'utilisateur travaille normalement. Ceci peut être changé pendant le travail (sous  \textit{Fichier - Mandant}...),mais lors du login tout d'abord c'est le mandant de référence qui est activé.

\subsection{Introduire les paramètres du laboratoire}
Ouvrez d'abord de nouveau le \textit{View-Contacts}. Introduisez là votre laboratoire interne et aussi le laboratoire externe et marquez-les comme \textit{Laboratoire}
%TODO!!

\includegraphics[width=4in]{images/grundkonfmand1.png}
% grundkonfmand1.png: 752x585 pixel, 72dpi, 26.53x20.64 cm, bb=0 0 752 585

\subsection{Configurer le programme de texte}

Elexis travaille jusqu'alors que avec OpenOffice, raison pour laquelle nous expliquons ici que la configuration avec OpenOffic.

\begin{itemize}
 \item Si vous ne l'avez pas encore fait, installez OpenOffice (au minimum version 2.0)
 \item Choisissez dans Elexis sous \textit{Fichier - Options - Traitement de texte 1 : NOA-Text}
\end{itemize}

\includegraphics[width=3.5in]{images/grundkonfmand2.png}
% grundkonfmand2.png: 580x480 pixel, 72dpi, 20.46x16.93 cm, bb=0 0 580 480

\begin{itemize}
 \item Allez dans Elexis sous  \textit{Fichier - Options sous OpenOffice} et cherchez le chemin d'accès du sous répertoire 'program' de l'installation OpenOffice. Pour ceci vous cliquez sur 'Define = définir' et choisissez sous 'browse' le sous répertoire. Il se trouve sous Windows normalement sous : C./progammes/OpenOffice.org 2.0/programme respectivement à l'endroit où vous l'avez installé.
 \item Cliquez sur \textit{Apply=appliquer}, fermez la configuration et redémarrez Elexis. 
 \item Si vous ouvrez par ex. la 'Perspective - Lettres', la fenêtre de OpenOffice devrait s'ouvrir dans la fenêtre de Elexis. (Ceci durera lors de la première utilisation assez long temps ~30 secondes).
\end{itemize}

\subsubsection{\index{modèles!Druckvorlagen}Créer un modèle}
Pour quelques formulaires Elexis cherche des modèles prédéfinis qui ont un nom spécifique. Ces modèles définissent pour certains formulaires l'apparence spécifique pour leur fonction. Pour des données variables il faudra y introduire à des endroits spécifiques des espaces réservés.
Pour créer un formulaire vous procédez de la façon suivante :
Préparez votre formulaire tout normalement dans le programme de traitement de texte et sauvegardez-le comme tout document avec texte. Depuis Elexis vous choisissez la perspective \textit{Lettres} et ensuite 

\includegraphics[width=2.5in]{images/import.png}
% import.png: 297x226 pixel, 96dpi, 7.86x5.98 cm, bb=0 0 223 169

le View-menu à droite.

Choisissez là  \textit{Text importer} et cherchez le formulaire que vous venez de créer. Par cette action vous importez le document dans Elexis. Ensuite vous pouvez encore faire des adaptation du texte et après avoir fini vous choisissez de nouveau le View-menu à droite. Cette fois-ci vous choisissez \textit{sauvegarder comme modèle}.

\includegraphics[width=2.5in]{images/rezept1.png}
% rezept1.png: 219x147 pixel, 72dpi, 7.73x5.19 cm, bb=0 0 219 147

 Comme\textit{nom du modèle} vous devez introduire pour les modèles standard mentionnés en bas la nom correspondant pour des modèles personnels vous pouvez par contre utiliser des désignations quelconques. Sous  \textit{mandant} vous pouvez définir pour quel mandant ce modèle avait été crée ou si tout le monde l'utilisera.
 
Figurant ci-dessous vous trouvez une liste des modèles standard :


\begin{itemize}
\item Ordonnance : \index{modèles!ordonnance}Vous avez besoin pour cela un modèle nommé \textit{ordonnance}. Il pourrait se voir par exemple de cette façon :
 \end{itemize}
\includegraphics[width=4in]{images/rezept.png}
% rezept.png: 477x290 pixel, 72dpi, 16.83x10.23 cm, bb=0 0 477 290

A l'endroit où c'est écrit  [Rezeptzeilen] vous introduirez plus tard les médicaments que vous choisissez. Cette variable est donc indispensable. Tout les autres éléments du modèle \textit{ordonnance} sont facultatifs.

\begin{itemize}
 \item \textit{Certificat d'incapacité de travail }: Un modèle pour le certificat d'incapacité de travail.  Vous pouvez introduire comme variable[AUF.Grund=raison d'incapacité], [AUF.von=incapacité de ...], [AUF.bis=incapacité à ...], [AUF.Prozent=incapacité pourcentage] et tout les variables standards . Tous sont facultatives.


 \item \textit{Feuille labo}: Pour imprimer les résultats du laboratoire. Les résultats sont introduits dans la variable  Laborwerte  et cette variable est indispensable, d'autres peuvent être introduites selon besoin. 
 \item \textit{Liste}: Il s'agit de l'impression des différentes données en forme de listes. Quelque part doit se trouver la variable  Liste  à laquelle peuvent être jointes les données.


\end{itemize}

Des Plugins que vous utilisez peuvent éventuellement nécessiter certains modèles.

\subsection{Installer le module de facturation}
\label{conf:abrechnung}
Il est indispensable que l'installation de ce module soit fait avant de vouloir introduire des prestations. Le procédé dépend du module de facturation. Pour le module des tarifs pour les médecins en Suisse le procédé est décrit lors de la description du Plugin correspondant sous (S. \ref{arzttarife}, page \pageref{arzttarife} ff.)



\chapter{Réglages}
	% *****************************************************************************
% * Copyright (c) 2007 by Elexis
% * All rights reserved. This document and the accompanying materials
% * are made available under the terms of the Eclipse Public License v1.0
% * which accompanies this distribution, and is available at
% * http://www.eclipse.org/legal/epl-v10.html
% *
% * Contributors:
% *    G. Weirich - initial implementation
% *
% *  $Id: settings.tex 6282 2010-04-19 19:24:51Z niklausgiger $
% *******************************************************************************
%
% !Mode:: "TeX:UTF-8" (encoding info for WinEdt)

\label{settings}
Die Einstellungen sind alle im selben Dialog zusammengefasst, welcher unter
\textsc{Datei-Einstellungen} erreicht werden kann (Abb. \ref{fig:settingsmain}).
%\usepackage{graphics} is needed for \includegraphics
\begin{figure}[h]
\begin{center}
  \includegraphics[width=0.6\textwidth]{images/settingsmain}
  \caption{Einstellungs-Dialog}
  \label{fig:settingsmain}
\end{center}
\end{figure}


Wie üblich in Elexis ist der genaue Inhalt dieses Dialogs davon abhängig,
welche Plugins installiert sind. Mit den Reitern auf der linken Seite wählt man
einen Bereich aus, für den man Einstellungen ändern möchte. Wir gehen hier auf
diejenigen Seiten ein, die zur Grundausstattung von Elexis gehören. Grundsätzlich
sollten alle Einstellungen \glqq Ab Werk\grqq{}vernünftige Grundeinstellungen
aufweisen, so dass es zunächst nicht nötig ist, hier etwas zu ändern. Sie
brauchen daher dieses Kapitel auch nicht unbedingt weiterzulesen.

\section{Abrechnungssysteme}
\label{settings:abrechnungssystem}
\index{Abrechnungssystem}
\index{KVG}\index{UVG}\index{TarMed}
Auf dieser Seite legen Sie fest, welche Arten von Abrechnungssystemen in Ihrer Praxis verwendet werden.

Da Elexis ein universelles Programm ist, welches nicht nur Ärzte, sondern auch Angehörige anderer Gesundheitsberufe unterstützen kann, ist das Abrechnungssystem sehr offen und gehalten. Das bedeutet, dass anfänglich eine entsprechende Konfiguration notwendig ist.

Die Verrechnung von Leistungen hat drei grundsätzliche Elemente:
\begin{enumerate}
    \item Ein Codesystem, also vereinfacht gesagt ein Konzept, um einzelne Leistungen und deren Wert zu benennen. Beispiel: \glqq Tarmed\grqq{}, \glqq Akupunkturtarif\grqq{} usw.
   \item Ein Garantenkonzept, also wer bekommt die Rechnung, wer bezahlt letztlich usw. Beispiele: Tiers Garant, Privatrechnung etc.
   \item Ein Rechnungskonzept: Wie muss die Rechnung aussehen, auf welchem Weg wird sie versandt (elektronisch, Papier)
\end{enumerate}

Eine Kombination aus Codesystem,Garantenkonzept und Rechnungskonzept nennen wir hier \glqq Abrechnungssystem\grqq{}.

Man kann in Elexis beliebig viele Abrechnungssysteme definieren, welche parallel existieren und je nach Bedarf verwendet werden können.

Jedes Abrechnungssystem hat bestimmte Eigenschaften:\\
\begin{wrapfigure}{l}{5cm}
    \includegraphics[width=4.7cm]{images/abrechnungssystem1}
    \caption{Abrechnungssystem Detail}
    \label{fig:abr1}
\end{wrapfigure}

Der Name ist frei wählbar, Leistungscode-System und Standard-Rechnungsausgabe können aus den installierten Plugins gewählt werden (Für nicht vorhandene Abrechnungssysteme müssten entsprechende Plugins erstellt werden).
Der Multiplikator ist ein Faktor, der auf jede einzelne Position angewendet wird, wenn sie verrechnet wird. Man kann also verschiedene Abrechnungssysteme mit demselbem Leistungscode-System, aber verschiedenen Multiplikatoren (\glqq Taxpunkten\grqq{}) haben. Mit \textit{Hinzufügen} kann man einen Multiplikator definieren, welcher immer ab einem bestimmten Datum gilt.

Und schliesslich sind je nach Abrechnungssystem bestimmte Angaben notwendig, um die Leistungen zu verrechnen, beispielsweise Rechnungsempfänger, Kostenträger, Nummern etc.
Elexis macht hier keine Vorschriften, Sie können eingeben, was immer Sie wollen. Es sind die Datentypen Text, Kontakt und Datum möglich. (Bei UVG-Versicherungen beispielsweise auch \glqq Unfalldatum\grqq{}).\\


\section{Allgemein}
Auf dieser Seite werden allgemeine Einstellungen für den Programmablauf
definiert. Es sind dies:
\subsection{Einstellungen zum Log}
Das \glqq Log\grqq{} ist das Logbuch eines Programmes. Hier werden verschiedene
Information zum Programmablauf gespeichert, welche z.B. bei der Fehlersuche
nützlich sein können.
\begin{itemize}
  \item Logdatei: Der Ort, an den die Log-Informationen gespeichert werden. Dies sollte normalerweise eine Datei \glqq elexis.log\grqq{} in Ihrem
  Datenverzeichnis sein. Der Wert \glqq none\grqq{} ist nur sinnvoll, wenn Sie
  Elexis aus einer Entwicklungsumgebung heraus starten.
  \item Log-Stufe: Wieviele Meldungen ausgegeben werden sollen. Auf Stufe 1
  werden nur die allerschlimmsten Fehler, die einen Programmabbruch erzwingen,
  ausgegeben. Auf Stufe 5 werden sehr viele Meldungen, die nur in speziellenj
  Fällen sinnvoll sind, ins Log geschrieben. Wir empfehlen für den Normalbetrieb
  Stufe 2 oder 3.
  \item Alert-Stufe: Meldungen, die den entsprechenden Schweregrad haben, werden
  nicht nur ins Log geschrieben, sondern gleich am Bildschirm angezeigt.
  Achtung: Wenn Sie hier eine zu hohe Stufe angeben, werden Sie ständig durch
  aufpoppende Meldungsboxen irritiert werden. Wir empfehlen Stufe 1.
  \item Tabellenname für Trace: Trace bedeutet, dass alle Aktionen in einer
  speziellen Tabelle aufgezeichnet werden. Es lässt sich damit später
  nachvollziehen, von welcher Arbeitsstation aus zu welchem Zeitpunkt welche
  Aktion mit Elexis durchgeführt wurde. Dies erlaubt eine sehr genaue Kontrolle
  der Vorgänge, kostet aber natürlich Arbeitsgeschwindigkeit und Speicherplatz.
  Wir empfehlen im Normalfall die Einstellung \glqq none\grqq{}.
  \item Bevorzugte Sprache: Diese Einstellung definiert nicht, welche
  Sprachversion von Elexis ausgeführt wird (Das wird anhand der
  Betriebssystemeinstellungen und ggf. Startparameter entschieden), sondern
  vielmehr, welche Tarmed- und ICD-Versionen etc. importiert werden.
  \item Speicherdauer im Cache: Dies ist eine sehr technische Einstellung. Es
  geht darum, wie lange aus der Datenbank gelesene Objekte gültig bleiben
  sollen, bevor sie erneut gelesen werden. Wenn
  viele Arbeitsstationen im Netz sind, geben Sie hier besser kürzere Zeiten an
  (z.B. 5 Sekunden), wenn Sie von zuhause über eine langsame Internet-Verbindung
  auf Elexis zugreifen, eher eine längere Zeit (z.B. 300 Sekunden).
  \item Aktualisierungsintervall: Nach welcher Zeitspanne soll Elexis jeweils
  seine Views aktualisieren. Wenn beispielsweise die MPA einen Patienten der
  Agenda auf \glqq eingetroffen\grqq{} setzt, dann dauert es maximal soviele
  Sekunden, bis diese Statusänderung auf Ihrem Bildschirm sichtbar ist. Wenn Sie
  zu kurze Zeiten angeben, wird die Netzwerkbelastung unnötig hoch.
\end{itemize}
\section{Anwender}
In diesem Zweig der Einstellungen sind anwenderspezifische Einstellungen
untergebracht. Wenn Sie einheitliche Einstellungen möchten, können Sie auch einen
Einstellungssatz unter einem frei wählbaren Namen speichern und von einem
anderen Anwenderaccount oder Abreitsstation aus wieder unter diesem Namen laden.

Die Buttons \glqq Einstellungen laden von\ldots\grqq{} bzw. \glqq Einstellungen
speichern nach\ldots\grqq{} betreffen hierbei die anwenderspetifischen
Einstellungen(im Wesentlichen alles was im Zweig \textsc{Anwender} der
Einstellungen vorhanden ist), während die Buttons \glqq
Arbeitsplatzeinstellungen \ldots\grqq{} die auf der lokalen Station
gespeicherten Perspektivenlayouts betreffen.
\subsection{Anwender - Ansicht}
\label{userconfig}
Hier sind verschiedene Ansichtsoptionen zusammengefasst:
\begin{itemize}
\item Erweiterbare Felder: Hier geht es um die 'aufklappbaren' Felder in manchen Views, z.B. der Patient-Detail View die Felder Diagnosen oder Bemerkungen etc. Man kann festlegen, dass solche Felder immer erstmal geschlosse oder immer erstmal geöffnet sein sollen, oder dass sie sich ihren letzten Zustand merken sollen.
\item Anzuzeigende Felder in Patientenliste: Dies definiert die Filterfelder mit denen man in der Patientenliste Patienten suchen kann. Standardmässig werden Name, Vorname und Geburtsdatum als Filterkriterien angeboten, man kann aber auch PatientNr anzeigen lassen.

\item Zusatzfelder im Patient-Detail-Blatt: Hier können Sie beliebige zuätzliche Text-Angaben erfassen, die Sie für einen Patienten erfassen können wollen, und die nicht mit Bemerkungen oder Etiketten erfasst werden können. Geben Sie einfach pro Zeile einen Namen für einen abzuspeichernden Datentyp ein.

\end{itemize}

\subsection{Anwender - Schriftarten}
Hier können Sie die Standardschriftart und -grösse für alle View-Inhalte angeben. Einzelne Views und Plugins können immer noch andere Schriftarten einstellen, aber dies ist die Standardvorgabe.

\section{Datenaustausch}
Dies ist eine Sammelkategorie für Einstellungen von Plugins, die Datenaustausch von und nach Elexis anbieten. Ob und welche Einstellungsseiten hier zu finden sind, hängt von den installierten Transport-Plugins ab.
\section{Datenbank}
Anzeige von Einstellungsdetails der aktuellen Datenbankverbindung
\section{Druckereinstellungen}
Hier kann man für jede Papierart den dazugehörigen Drucker und -Schacht auswählen. Beim Labeldrucker kann man ausserdem einstellen, ob der Druckerauswahldiealog überhaupt jedesmal vor dem Drucken angezeigt werden soll (wenn man z.B. mehrere Labeldrucker hat).
\section{E-Mail}
Diese Einstellungen sind für das Versenden von E-Mails aus Elexis wichtig. Dies wird inbesondere beim automatischen Versenden von Fehlermeldungen verwendet. 

\section{Gruppen und Rechte}
Dies ist die zentrale Benutzerverwaltung. Auf diesen Einstellungsseiten können Anwender und Mandanten eingerichtet und die Zugriffsrechte verteilt werden. Das Konzept der Gruppen ist auf Seite \pageref{sec:gruppen} genauer erläutert.
Legen Sie zunächst unter \textsc{Gruppen und Rechte} fest, welche Anwendergruppen Sie benötigen.
Um einen neuen Anwender oder Mandanten einzurichten, müssen Sie diesen zunächst als \glqq Kontakt \grqq{} erfassen, und dort unter Kontakt-Details als Anwender bzw. Mandant kennzeichnen. Dann können Sie unter \textsc{Gruppen und Rechte - Mandanten} dem Mandanten einen Benutzernamen und ein Passwort zuordnen, und angeben, welchen Gruppen er zugehörig sein soll.
Unter \textsc{Gruppen und Rechte - Anwender} können Sie dasselbe für Anwender angeben, ausserdem noch, für welchen Mandant dieser Anwender normalerweise tätig ist.
Unter \textsc{Gruppen und Rechte - Zugriffsteuerung} können für jede Gruppe und jeden Anwender einzeln Rechte zugeordnet werden.  (S. \ref{sec:gruppen}).
\section{Laborwerte}
\label{config:labor}
Hier können die in der Praxis benötigten Laborparameter definiert werden. Dies kann manuell geschehen, oder, bei Laborimport-Plugins können Laboritems auch automatisch mit den vom Labor gelieferten Angaben erstellt werden.
Jedes Laboritems ist durch folgende Eckdaten gekennzeichnet:
\begin{itemize}
\item{Einen Namen}
\item{Ein Kürzel}
\item{Das Labor, von dem es stammt}
\item{Den Normbereich, gegeben durch die Methode, z.T. auch geschlechts- alters- zyklusabhängig}
\item{Eine Gruppe, unter der des aufgelistet wird (z.B. Hämatologie)}
\item{Eine Sequenznummer, die angibt, an welcher Stelle innerhalb der Gruppe es einsortiert wird.}
\item{einen Typ (numerisch, absolut, text, Formel)}
\end{itemize}

Jedes Laborresultat ist durch ein solches Item, ein Datum und einen Patienten eindeutig identifiziert.
Es kann deswegen durchaus mehrere Items für ein- und denselben Parameter geben. Beispielsweise kann es ein Item \textsc{Vitamin B12} von verschiedenen Labors geben, welche nicht zwingend denselben Normbereich haben müssen.
Sie können die Liste der Tabelle durch Klick auf die Spaltenköpfe umsortieren.

Mit \textsc{Neuer Laborparameter} können Sie manuell ein neues Item erstellen\footnote{Beim Import von Laborwerten aus externen Labors werden die benötogten Items je nach Import-Plugin idR. automatisch erstellt.} und die oben genannten Angaben eingeben. Hier ist es sehr wichtig, dass Sie sich im Voraus genau überlegen, welche Laborparameter Sie benötigen, und wie Sie diese gruppiert haben wollen.
%\usepackage{graphics} is needed for \includegraphics
\begin{figure}[htp]
\begin{center}
  \includegraphics{images/labor1}
  \caption{Neues Labor-Item erstellen}
  \label{fig:labor1}
\end{center}
\end{figure}

In Abb. \ref{fig:labor1} sehen Sie den Dialog zum Anlegen eines neuen bzw. Ändern eines existierenden Items. Ganz oben geben Sie da Labor ein, von dem es stammt (Das Labor muss bereits als Kontakt erfasst sein). Kürzel und Titel sind die Anzeige des Items. Als Typ wählen Sie \textit{Zahl}, \textit{Text} (Für Parameter, die sich nicht als Zahl darstellen lassen, z.B. Bakteriologiebefunde), \textit{Absolut} für Parameter, die nur positiv oder negativ sein können und \textit{Formel} für Resultate, die errechnet werden sollen (z.B. LDL-Cholesterin gem. Friedewald- Dies ist weiter unten (\ref{ref:formel}) genauer erklärt).

Unter \textit{ReferenzM} geben Sie den Referenzbereich für Männer, unter \textit{ReferenzF} für Frauen ein \footnote{Elexis benötigt diese Angaben, um numerische Laborresultate automatisch als pathologisch darstellen zu können. Es ist deshalb wichtig, dass Sie den Referenzbereich genauso, als von-bis eingeben.}, unter \textit{Einheit} entsprechend die Masseinheit für den Parameter.

Unter \textit{Gruppe} geben Sie an, wo dieser Parameter gruppiert sein soll. Bereits existierende Gruppen sind in der Combobox schon enthalten und können einfach ausgewählt werden. Um eine neue Gruppe zu erstellen, können Sie den Namen einfach eintippen. Der Gruppenname muss folgendes Format haben:
Ein- oder mehrere Buchstaben, ein Leerzeichen, dann ein beliebiger Text. Die räfix entscheidet über die Sortierung auf dem Laborblatt. So wird die Gruppe  \textit{A Hämatologie} oberhalb von \textit{B Elektrolyte} zu stehen kommen, und \textit{DA Leberwerte} vor \textit{DC Nierenparameter}. Die Bennenung und die Reihenfolge der Gruppen bleibt ganz Ihnen überlassen.

Unter \textit{Sequenz-Nr} schliesslich geben Sie ein, wo innerhalb der Gruppe dieser Parameter auf dem Laborblatt stehen soll. Dies muss eine Zahl sein. Hierbei kommt es nicht auf den Abstand der Zahlen der einzelnen Items an, sondern nur auf die Grösse relativ zueinander. Es ist empfehlenswert, die Zahlen nicht unmittelbar aufeinanderfolgend zu wählen, damit man später ev. leicht noch etwas dazwischenfügen kann.

\subsection{Berechnete Laborwerte (Typ Formel)}
\label{ref:formel}
\index{Formel}
\index{Laborwert!formel}
Ein Laborparameter vom Typ \textit{Formel} wird nicht eingetragen oder eingelesen, sondern mit einer -im Prinzip beliebigen- Formel berechnet. In der Regel wird man sich dabei auf andere Laborwerte beziehen, von welchen der zu errechnende Parameter abhängig ist. Als Beispiel wollen wir hier einen Parameter für LDL, berechnet nach der Friedewald-Formel, erstellen. Diese Fromel benötigt als Parameter die Werte für Gesamtcholesterin, Triglyceride und HDL-Cholesterin, sie lautet ja: \textit{Gesamtcholesterin-HDL-(TG/2.2)}. Für user Beispiel seien diese anderen Parameter so vorhanden:
\begin{itemize}
  \item Gesamtcholesterin in Gruppe \textit{G Fettstoffwechsel}, Sequenznummer 10
  \item HDL-Cholesterin in Gruppe \textit{G Fettstoffwechsel}, Sequenznummer 20
  \item Triglyceride in Gruppe \textit{G Fettstoffwechsel}, Sequenznummer 40
\end{itemize}

Wir erstellen jetzt einen neuen Parameter namens \textit{LDL (errechnet)} in Gruppe G und geben ihm z.B. die Sequenznummer 21 (hier sehen Sie, dass es gut war, dass wir bei den vorherigen Sequenznummern Lücken gelassen haben) (Abb. \ref{fig:labor2}).
%\usepackage{graphics} is needed for \includegraphics
\begin{figure}[htp]
\begin{center}
  \includegraphics{images/labor2}
  \caption{Berechneten Laborwert erstellen}
  \label{fig:labor2}
\end{center}
\end{figure}
 Dann Klicken Sie bitte auf die Typbezeichnung \textit{Formel}. Es öffnet sich ein Eingabedialog zum Eingeben der Formel:\\
 \includegraphics{images/labor3}\\
Um uns in der Formel auf andere Laborparameter zu beziehen, verwenden wir deren Gruppenindex (also das was vor dem Leerzeichen in der Gruppenbezeichnung steht) und Sequenznummer, getrennt durch einen Unterstrich. Um Gesamtcholesterin zu referenzieren, wählen wir also die Bezeichnung G\_20. Die Friedewald-Formel wird damit zu \textit{G\_10-G\_20-(G\_40/2.2)}. Dies würde dann allerdings auf 9 Stellen genau ausgegeben. Deshalb runden wir in unserem Beispiel mittels Math.round noch auf 2 Stellen.

 Wenn jetzt die Laborwerte, auf die sich die Formel bezieht, eingegeben werden, dann wird Elexis jeweils versuchen, die Formel auszuwerten. Wenn das nicht gelingt, weil beispielsweise noch nicht alle benötigten Parameter eingegeben sind, wird \textit{?formel?} eingesetzt (S. Abb.  \ref{fig:labor4}).
 %\usepackage{graphics} is needed for \includegraphics
\begin{figure}[htp]
\begin{center}
  \includegraphics{images/labor4}
  \caption{Laboreintrag, noch nicht komplett}
  \label{fig:labor4}
\end{center}
\end{figure}

Erst wenn alle benötigten Parameter vorhanden sind, wird der errechnete Wert eingesetzt (Abb. \ref{fig:labor5}).
%\usepackage{graphics} is needed for \includegraphics
\begin{figure}[htp]
\begin{center}
  \includegraphics{images/labor5}
  \caption{Laboreintrag, komplettiert}
  \label{fig:labor5}
\end{center}
\end{figure}

\section{Leistungscodes}
Dies ist wieder eine Sammelrubrik, die je nach vorhandenen Plugins für die Leistungsabrechnung unterschiedlich gefüllt sein kann. In der Schweiz ist hier standardmässig Labortarif und Tarmed vorhanden. Diese sind auf Seite \pageref{arzttarife} genauer erklärt.

\section{Textverarbeitung}
Auch die verwendete Textverarbeitung für Briefe etc. ist in Elexis ja durch Plugins frei definierbar. Welche Textverarbeitung verwendet werden soll, kann hier eingestellt werden. Diese Einstellung sollte normalerweise nicht mehr verändert werden, wenn erste Dokumente erstellt wurden, da diese sonst eventuell nicht mehr ohne weiteres lesbar wären. Wir empfehlen, unter Windows das Plugin NOAText und unter Linux Office-Wrapper zu verwenden.


\chapter{Sujets élargis}
    % *******************************************************************************
% * Copyright (c) 2007 by Elexis
% * All rights reserved. This document and the accompanying materials
% * are made available under the terms of the Eclipse Public License v1.0
% * which accompanies this distribution, and is available at
% * http://www.eclipse.org/legal/epl-v10.html
% *
% *  $Id: voraussetzungen.tex 3094 2007-09-04 10:03:03Z rgw_ch $
%
%*******************************************************************************
% !Mode:: "TeX:UTF-8" (encoding info for WinEdt)

\section{Scripting}
\index{Script}\label{Script}\index{BeanShell}
Elexis peut être élargi par des scripts. Des scripts sont des mini-programmes qui peuvent être executés immédiatement et qui peuvent accomplir quelques tâches simples.
On emploie comme interprète du script Beanshell (http://www.beanshell.org). Une description plus détaillée de ce concept dépassera le cadre du manuel, vous pouvez vous renseigner sur plus de détails à l'adresse suivante : http://www.rgw.ch/elexis/dox/elexis-scripting.pdf ou
http://www.elexis-forum.ch/viewtopic.php?t=107 ou vous trouverez quelques exemples.


\bigskip
Pour éditer, sauvegarder et démarrer des scripts il existe la 'View' Scripts dans Elexis.


\chapter{Réflexions concernant la sécurité des données}	
	% *******************************************************************************
% * Copyright (c) 2007 by Elexis
% * All rights reserved. This document and the accompanying materials
% * are made available under the terms of the Eclipse Public License v1.0
% * which accompanies this distribution, and is available at
% * http://www.eclipse.org/legal/epl-v10.html
% *
% *  $Id: sicherheit.tex 4905 2009-01-03 18:30:50Z rgw_ch $
%
%*******************************************************************************
% !Mode:: "TeX:UTF-8" (encoding info for WinEdt)

\label{sicherheit}
\index{sécurité des données}
Des donnés sensibles (comme les données des patients le sont toujours) doivent être archivées et sauvegardées avec une précaution particulière. Cet article décrit quelques concepts pour la sécurité des données.

Des données sensibles comme on les trouve dans les cabinet médicaux doivent :
\begin{itemize}
  \item{être protégées contre une perte}
  \item{être protégées contre falsification ou manipulation volontaire ou involontaire }
  \item {être protégées contre la prise de connaissance par des personnes non-autorisées}
\end{itemize}

Ces points seront traités dans le chapitre qui suit.

\section{Comment éviter la perte des données ? }\index{perte des données}
En principe il existe à tout moment et dans chaque système d'ordinateurs le risque de perte totale ou limitée des données. Ceci peut se produire à cause d'un défaut de hardware (un disque dur par exemple n'a qu'une durée de vie limitée de quelques années lors d'un fonctionnement permanent et il peut devenir soudainement illisible lorsque quelques secteurs importants sont détruits.) Ceci peut aussi se produire par des influences externes (par exemple une rafale de tension ou une panne de courant lors d'une écriture importante sur le disque). Mais aussi l'apparition subite des fautes non-détectées des logiciels participants peuvent mener à une perte des données.

Pour toutes ces raisons il faut réfléchir ::
\begin{itemize}
  \item {Pour quel laps de temps je pourrais à la limite reconstruire manuellement les données ou pour quel laps de temps une perte de données ne serait pas trop importante ?.}
  \item {Combien me coûtera la reconstruction manuelle de ces données ?}
  \item {Combien me coûteront des données irréparables ?}
\end{itemize}

Suite à ces réflexions on pourra estimer combien une solution de sauvegarde automatique des données \index{Backup}pourra coûter et ensuite on pourra fixer la fréquence utile de la sauvegarde.
Dans le cadre d'une utilisation fréquente d'une application au cabinet médical pour laquelle une perte de données ne sera non seulement pénible mais pourrait aussi entraîner des conséquences juridiques, une sauvegarde qui aura lieu toutes les heures ou plusieurs fois par jour pourrait être utile. Dans tout les cas au minimum une sauvegarde journalière est très recommandée.

Le fonctionnement de cette sauvegarde dépendra de la banque de données utilisée. Si vous ne connaissez pas le processus dans votre banque de données ou si vous ne pouvez pas le mettre en route vous-même, nous vous conseillons vivement d'engager quelqu'un de compétent pour accomplir cette maintenance - la renonciation d'une sauvegarde régulière peut entraîner des graves conséquences.

\section{Comment éviter la falsification des données ? }

Le dossier électronique est dans ce contexte un désavantage en comparaison avec un dossier sur papier : Des manipulations faites sur une note autographe ne sont généralement pas difficile à détécter. Par contre on ne voit pas dans le dossier électronique s'il est encore dans l'état originale. Elexis essaie d'y remédier par le concept de la gestion des versions de documents :
Un changement d'une note dans le dossier électronique du patient n'écrase jamais l'ancien enregistrement mais cré une nouvelle version de cet enregistrement qui sera marquée par la date, l'heure et l'utilisateur connecté. Si nécessaire les anciennes versions peuvent être visionnées et /ou restaurées très facilement. Un utilisateur standard n'a pas de possibilité d'effacer une note de façon durable. Par contre pour des raisons pratiques l'administrateur seul possède cette possibilité. De cette façon des fautes importantes peuvent être effacées ou de temps en temps il pourra "nettoyer" la base de données pour la rendre plus habile. On peut se protéger contre une éventuelle reproche d'avoir falsifié des documents en faisant avant un tel "nettoyage" une copie de la base de données sur un support de donnés seulement une fois inscriptible. On pourra si nécessaire même marquer ce document avec un chronotimbre fiable et le laisser conserver sellé de façon notariale. Une protection plus importante contre des activités de l'administrateur n'est techniquement pas possible car quelqu'un avec les droits de l'administrateur pourrait à tout moment même effacer la base de données ou la remplacer par une qu'il avait falsifié ou par une ancienne version. Pour cette raison nous préconisons fortement de ne donner le droit d'accès en tant qu' administrateur sur l'ordinateur qui contient la base de donnée qu'à une seule personne.


\section{Comment éviter l'accès non autorisé à la base de données ?}

Une base de donné sert à enregistrer des informations, des les fournir au moment voulu et de permettre de les modifier. Malheureusement la base de données ne  \glqq sait pas\grqq chaque fois d'emblée si l'accès se fait par une personne autorisée ou non. Des accès non autorisés peuvent avoir lieu de façon ciblée (par exemple pour espionner des données, pour les détruire mais aussi pour les modifier de façon subtile chose qui peut parfois provoquer des dégâts nettement plus importants qu'une destruction directe qui sera au moins rapidement constatée.).
Des accès non autorisés peuvent aussi avoir lieu de façon aléatoire et non ciblée, provoqué par des logiciels nuisibles qui sont distribués largement et qui tentent d'attaquer n'importe quel système. Dans ce qui suit nous essayons de tracer quelques scénarios d'attaques. Ensuite nous mentionnons les mesures de défense possibles. Cette partie du manuel est très technique et ne devrait vous intéresser que lorsque vous n'avez pas confié l'installation et la maintenance de votre réseau à des professionnels externes.

\subsection{Attaque contre des ports ouverts}

Un ordinateur qui est connecté à Internet est comparable à une maison avec des portes qui servent à des différentes tâches. Au lieu d'avoir des escaliers de la cave, des entrées pour les livreurs, des  portes des balcons, porte principale et porte du garage, un ordinateur n'a que des Ports et de ceux exactement 65'535. Chacun de ces Ports peut être comme une porte : ouverte, fermée ou même scellée. Un port ouvert est comme une porte d'entrée ouverte. Dans un certain sens une invitation pour des voleurs de venir voir comment accéder à l'intérieur de la maison. Tout aussi peu qu'il y a un sens de sceller à la maison tout les portes et fenêtres, on ne peut pas se priver simplement de ces ports. Si on ne veut pas permettre de communication à travers certains ports on pourrait plus facilement tirer le câble du réseau ou du téléphone.
Heureusement un port ouvert ne correspond pas seulement à un  \glqq trou\grqq dans l'ordinateur mais il y a toujours un portier - un programme qui a ouvert ce port. Sans des tels programmes tout les ports seraient fermés de façon standardisée. Un agresseur va donc d'abord observer s'il trouve un port ouvert. Pour ceci il va tester tout les ports un après l'autre = Portscan. S'il trouve des ports ouverts, il va essayer de savoir quel programme avait ouvert le port et s'il s'agit d'un programme pour lequel une vulnerabilité est connue il va utiliser cette vulnerabilité de la sécurité pour une attaque.

Pour une telle attaque par Portscan, analyse du programme et intrusion, il ne faut malheureusement pas être un Hacker très intelligent qui est déterminé à tout, mais ils existent en masse des programmes déjà prêts pour produire par seconde des telles attaques contre des milliers d'ordinateurs et qui peuvent être distribués par exemple par des jeunes aventuriers ou simplement destructeurs (\glqq Script kiddies\grqq). En outre on constate dernièrement une professionnalisation de ces programmes qui est à prendre au sérieux car financée par des spammeurs dont le seul but est d'abuser des ordinateurs attaqués pour la distribution des spams et pour espionner des données confidentielles.

\medskip

Qu'est-ce qu'on peut faire contre ?
\begin{itemize}
  \item {Des ordinateurs avec des données critiques ne devraient pas être lié à l'Internet ni directement ni indirectement (par le LAN). Pour surfer ou pour les e-mails il faudrait utiliser de préférence un ordinateur à part qui n'est pas lié au réseau. Si le LAN doit pourtant être lié à l'Internet il faudra absolument avoir des connaissances sur la possibilité de se protéger ou il faudra déléguer cette tâche à un professionnel.}
  \item  {Ne laisser ouvrir que des ports qui seront effectivement utilisés. Pour cela il faudra contrôler en détail quels services sont démarrés d'office par le système d'exploitation et si ce services sont effectivement utilisés. Des ordinateurs avec le système d'exploitation Windows on la tendance d'ouvrir des ports NetBIOS vers l'extérieur ce qui libère inutilement les ressources mises à disposition dans le LAN aussi directement dans l'Internet. Par un simple test à travers le site http://www.security-check.ch vous pouvez savoir lesquels des ports sont ouverts chez vous.}
  \item{Mettre un routeur entre le LAN et l'accès Internet. Un routeur  \glqq cache\grqq{}les adresses internes des ordinateurs dans le LAN et une firewall\footnote{Nous aimerions vous prévenir de ne pas donner trop de confiance à une  \glqq Personal Firewall\grqq{}. Une telle Software est elle-même exposée à des attaques contre lesquelles elle devrait protéger l'ordinateur et en effet il y a beaucoup de virus et autres logiciels nuisibles qui mettent ce Personal Firewall directement hors service. Une Hardware-Firewall est nettement mieux protégé contre de telles attaques et difficilement à mettre hors service.

} peut contrôler (entre autres) à travers lesquels des ports une communication peut être permise. Mais même ceci ne peut vous protéger contre toute attaque!}
  \item {Il faut faire attention d'utiliser le moins possible des logiciels dont on a connaissance de problèmes de sécurité. Beaucoup de produits de Microsoft appartiennent malheureusement juste à cette catégorie - dû à leur universalisation des logiciels comme Internet Explorer et Outlook sont régulièrement des cibles de ces attaques. Dans ce contexte l'utilisation des logiciels alternatifs pour le Web et/ou le Mailing vaut quelques réflexions si on veut augmenter la sécurité. }
\end{itemize}

\subsection{Attaque par exploitation des failles de sécurité}

Pour augmenter le confort pour l'utilisateur de leurs logiciels, Microsoft en premier lieu a impliqué dans leurs produits beaucoup de fonctions qui permettent de régler certaines fonctionnements de façon automatique. Ceci est même possible sans ordre de l'utilisateur. Il est par exemple possible que dans un e-mail, une page web, un document Word ou un tableau Excel se trouvent des commandes invisibles qui ouvrent sans demande de précision supplémentaire le logiciel concerné (Outlook, Internet-Explorer, Word, Excel). Ces fonctions qui visaient le confort de l'utilisateur ont été détournées par les producteurs de logiciels nuisibles. De cette façon l'ordinateur peut être infecté par un virus ou autre logiciel nuisible lorsqu'on ouvre simplement un e-mail, si on surfe simplement sur une page Web spécifique ou si on ouvre un document Office. Pendant les dernières années Microsoft a reconnu ces désavantages de leur logiciels et a développé régulièrement des améliorations, mais on trouve toujours des nouvelles failles de sécurité. Naturellement ce problème de base concerne aussi d'autres producteurs de logiciels mais Microsoft est pourtant par son importance la cible principale des attaques.

\medskip

Quoi faire contre ?

\begin{itemize}
    \item{ Procurez-vous toujours les derniers Updates de votre système d'exploitation et de votre logiciel. Seulement dans ce cas là vous avez la garantie qu'au moins les failles de sécurité reconnus jusqu'alors ont été réparées. }
    \item{Pour surfer à tout hasard vous ne devrez pas utiliser l'ordinateur du cabinet médical. Ne visitez des sites douteux jamais par un ordinateur qui est lié au réseau de l'entreprise.}
    \item{N'ouvrez jamais vos mails à l'hasard. L'inondation virale la plus grande s'était produite car les gens ont ouvert par Microsoft Outlook un e-mail avec le titre " I love you " et puisque Outlook installait le virus automatiquement, sans demande de précision, dans le système d'exploitation. Si vous recevez un e-mail avec un fichier exécutable comme document joint, vous ne devriez l'exécuter que si vous savez de qui et pourquoi vous l'avez reçu. Si un mail contient un document Office comme document joint, vous ne devriez jamais l'ouvrir avec le logiciel Microsoft correspondant mais avec un des multiples programmes gratuites qui permettent de voir que les donnes. }
    \item{Dans beaucoup de cas on peut changer sans problèmes sur un logiciel alternatif. On peut utiliser sans problème au lieu du Internet-Explorer par exemple Firefox ou Opera, ou au lieu de Outlook Thunderbird ou Opera ou au lieu de Microsoft Office OpenOffice.}
    \item{Installez sur chaque ordinateur un scanner à virus et veillez qu'il soit toujours mis à jour. Vous devez par contre savoir qu'un scanner à virus n'est pas une protection complète. Dépendant du système, il ne peut reconnaître que des virus ou logiciels malveillants qui lui sont déjà connus ou dont il peut reconnaître le comportement comme suspect à travers des méthodes heuristique.  -- Il ne peut pas reconnaître des nouveaux logiciels malveillants ou ceux qui se portent spécifiquement sur lui et évidemment il ne peut les neutraliser encore moins
    .}

\end{itemize}

\subsection{Attaque par interception du trafic du réseau}

Il s'agit d'un problème relativement nouveau. Les lignes des réseaux sont relativement sur contre des interceptions de données. (Puisqu'il consistent de plusieurs câbles torsadées les émissions sont minimes). Avec l'introduction des réseaux sans câbles (WLAN) une grande surface d'attaque s'est établie. Par principe toute personne qui se trouve à portée des ondes radio peut se brancher sur un WLAN et par conséquent espionner ou utiliser des ordinateurs du réseau sans être bloqué par une Firewall. En plus, tout le monde qui se trouve à portée des ondes radio peut écouter tout le trafic entre les ordinateurs du réseau. Ceci n'est techniquement pas du tout difficile et peut être réalisé par un équipement standard. Pour maîtriser ce danger les producteurs du WLAN ont développé relativement tôt une méthode de cryptage qui s'appelle WEP. Le WEP contient par contre des graves erreurs d'implémentation et doit aujourd'hui être considéré comme rompu. Ceci implique que toute personne qui utilise un certain logiciel en plus téléchargeable gratuitement à l'Internet, pourra dans quelques heures atteindre le but d'écouter, de contourner le cryptage WEP et d'entrer dans le réseau comme dans un réseau non protégé. Comme réaction à cette menace les producteurs du WLAN ont développé dernièrement une meilleure procédure de cryptage et authentification qui s'appelle WPA. Celle-ci ne peut être déplombée qu'avec une dépense considérable, du Know-How et beaucoup de patience (mais n'est pas non plus impossible à déplomber).
Il y en a encore toujours des appareils WLAN qui ne maîtrisent pas le WPA et en plus, par manque de standardisation certains appareils des différents producteurs ne sont parfois pas capables de communiquer. L'état actuel de la technique est le WPA2 aussi nommé WPA-AES ou IEEE 802.11i. Ce cryptage ne peut être déplombé qu'avec force brute et il est en plus standardisé au niveau international de sorte que tout les appareils IEEE 802.11i devraient être capables de communiquer.


\medskip
Qu'est-ce qu'on peut faire contre ?

Par principe : Evitez d'utiliser un WLAN si vous avez dans votre réseau des données sensibles. Si vous n'avez absolument pas la possibilité de tirer les lignes, réfléchissez de plutôt choisir une Powerline. S'il faut tout de même utiliser un WLAN : Utilisez exclusivement des appareils qui maîtrisent le WAP2 (IEEE 802.11i) et mettez surtout en route ce cryptage. Si vous pouvez régler l'énergie d'émission de votre Access-Point, choisissez l'émission la plus faible possible pour que le trafic interne ne puisse pénétrer le moins possible à l'extérieur. Utilisez pour l'authentification des utilisateurs du réseau soit un serveur RADIUS ou si vous employez PSK, changez au moins toute les quelques semaines le mot de passe WPA2 et n'utilisez pas une clé trop simple.

\subsection{Attaques en profitant de la naïveté de l'utilisateur}

Souvent les attaquants essayent à convaincre l'utilisateur par des E-Mails bien formulés à exécuter un document joint qui contient un virus ou de révéler des données sensibles comme des mot de passe etc.

\medskip

Que faire contre ceci ?

\begin{itemize}
    \item{Ne réagissez jamais à des e-mails qui vous demandent des informations par mail ou en cliquant sur un lien. Appelez plutôt l'émetteur présumée du mail et demandez le si ce mail provient de lui.}
    \item{N'ouvrez jamais des documents joints à un mail si vous ne savez pas exactement pourquoi vous l'avez reçu. Il ne suffit pas de voir si l'émetteur vous est connu car beaucoup de virus sont capable d'extraire et de falsifier des informations des émetteurs de votre répertoire d'adresses. }
\end{itemize}


\section{Et qu'est-ce que tout ça a avoir avec Elexis?}

Elexis est au moins en ce qui concerne la variante multi-client un système client/serveur. Cela veut dire que le serveur doit ouvrir un port a travers lequel le client peut avoir accès. Autrement une communication à travers un réseau ne serait pas possible. Dans le cas d'une base de données MySQL le numéro du port est 3306. En principe toute personne depuis tout les pays du monde pourrait accéder à votre base de données, si l'ordinateur est directement ou indirectement lié à l'Internet car il n'est pas un secret que derrière le port 3306 se trouve en général un serveur MySQL. Par contre vous êtes en sécurité lorsque vous fermez dans votre Router / Firewall les ports qui sont utilisés par votre base de données. Par cela vous laissez apparaître ces ports depuis l'Internet comme fermés, tandis qu'il sont ouverts dans le LAN interne. Si vous voulez par contre accéder à Elexis depuis votre domicile une communication depuis l'extérieur sera indispensable. Pour cela vous pouvez installer dans votre routeur spécifiquement pour le port nécessaire une règle \glqq forward\grqq qui vous donnera accès sur l'ordinateur qui contient la base de données. En ce cas là il faudra par contre absolument veiller à ce que l'accès sur la base de données soit contrôlé par ses propres règles de sécurité. N'utilisez en aucun cas le mot de passe standard, veillez à ce que le root-account de la base de données soit protégé par un mot de passe et ne soit pas accessible depuis l'extérieur et limitez les droits des utilisateurs qui accèdent depuis l'extérieur au minimum. Veuillez pour ceci lire la documentation de la base de données et demandez un spécialiste pour cette installation. Puisque aussi pour les serveurs de base de données on n'a jamais la garanti d'être libre de toute faille de sécurité, il pourrait être utile de ne même pas donner une ouverture par un port de base de donnée mais de permettre l'accès à travers l'Internet que par des canaux sécurisés comme le SSH ou le VPN. Une explication de ces techniques dépasserait par contre définitivement le cadre de ce manuel. En cas de besoin nous vous proposons de vous laisser conseiller individuellement sur des prises de mesures de sécurité utiles.

\section{Last but not least: Attaque par accès direct sur le disque dur}
Si une personne non autorisée peut accéder au serveur, elle peut en règle général voir tout ce qui se trouve sur le serveur, donc aussi votre base de données. Ne vous laissez pas tromper par des conceptions de sécurité de votre système d'exploitation : Un aggresseur qui peut accéder à votre serveur peut par exemple sortir tout simplement le disque dur de votre serveur et le lire sur un autre ordinateur où il a les droits d'administrateur. Le même problème se pose lorsqu'un disque dur doit être vendu ou éliminé : Quiconque recevant le disque dur pourrait lire tout le contenu - la distribution des droits au mandants et utilisateurs qui avait été faite à travers le système d'exploitation ne joue pas dans ce cas. \footnote{Même pas l'effacement des données ne sert vraiment : En général on peut reconstruire des données effacées avec plus ou moins d'investissement de temps.}

\medskip

Qu'est-ce qu'on peut faire contre ? 
Il n'y a qu'un remède : La base de données doit être installée dans un répertoire crypté du serveur. Heureusement les systèmes d'exploitation de nos jours apportent déjà toutes les choses nécessaires pour une installation des partitions ou répertoires cryptés\footnote{Certains par contre seulement dans la variante 'Professional', 'Business' ou 'Server'}. En plus il existent des Tools OpenSource gratuits comme par ex. TrueCrypt qui sont apte à faire la même chose. Le désavantage d'un système de données ou des répertoires cryptés est une rapidité d'accès probablement un peu réduite et le fait que vous devez mémoriser (une fois de plus) un mot de passe et celui-ci, il faut vous le graver dans votre mémoire car en cas de perte il n'y aura aucune chance de récupérer les données.

\subsection{Matériel pour le Backup}
Pour ce thème les mêmes règles comptent comme mentionné ci-dessus. Le meilleure sécurité ne vaut pas grande chose si le matériel du Backup non crypté est facilement accessible à un personne non autorisée. Si vous faites un Backup d'un partition cryptée, le Backup ne sera normalement pas crypté. D'un autre côté il y a plusieurs arguments contre un cryptage de ce matériel de Backup : Vous voudrez pouvoir lire ces Backups encore après 10 ans. Mais il n'est pas certain que vous connaissez votre mot de passe encore dans 10 ans ou que le logiciel de décryptage sera encore fonctionnel sur un ordinateur de l'époque dans 10 ans…
Nous proposons plutôt de laisser les Backups non cryptés par contre de les garder dans un endroit sécurisé. 


\chapter{Index}
\printindex
\end{document}
