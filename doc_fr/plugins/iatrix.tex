% *******************************************************************************
% * Copyright (c) 2007 by Elexis
% * All rights reserved. This document and the accompanying materials
% * are made available under the terms of the Eclipse Public License v1.0
% * which accompanies this distribution, and is available at
% * http://www.eclipse.org/legal/epl-v10.html
% *
% *  $Id$
% *******************************************************************************
% !Mode:: "TeX:UTF-8" (encoding info for WinEdt)

\section{Iatrix}\label{Iatrix}
\index{Iatrix}
Stil altérnative du dossier électronique avec un point de départ orienté sur le problème du patient. Ce Plugin fait partie du programme Iatrix.

\subsection{Informations générales}

Le développement de Iatrix est financé par le Dr. med. P. Schönbucher, Luzern.
Le développeur principale est  \href{http://www.elexis.ch/jp/component/option,com_contact/task,view/contact_id,2/Itemid,32/}{Daniel Lutz}.
Par conséquent nous vous prions de vous tourner vers lui pour des questions concernant ce Plugin.

Des plus amples informations concernant Iatrix sous : \href{http://www.iatrix.org/}.

\subsection{Utilisation}

\subsubsection{ouvrir 'View' }

Veuillez séléctionner sous  \textit{Fenêtre/Affichage/Other...}. Introduisez dans le champ de recherche
\textit{KG Iatrix}. Séléctionnez l'affichage \textit{KG Iatrix} et cliquez ensuite sur OK. Vous pouvez maintenant positionner la fenêtre sur votre surface de travail.

Veuillez aussi ouvrir l'affichage \textit{problème}, pour afficher les caractères d'un problème en détail. Gehen Sie gleich vor wie oben, um dieses Fenster
zu öffnen.

\subsubsection{Utilisation de Iatrix}

La 'view' \textit{KG Iatrix} montre dans la partie supérieure la liste des problèmes.
Pour inscrire un problème vous pouvez simplement introduire les informations dans la dernière ligne vide.

\begin{description}

\item[Date]
La date permet de voir quand le problème en question s'était manifesté pour la première fois. On peut introduire des dates complètes (par ex. 19.08.2003) ou seulement l'année (par ex. 2003).

\item[No.]
Les problèmes peuvent être numérés. Ce numéro servira à s'orienter. Un problème peut être une partie d'un autre problème. En ce cas on peut numérer par ex. de façon suivante : \textit{1}, \textit{1.1}, \textit{1.2}, etc..

\item[problème/diagnostic]
On peut décrir ici le problème de façon plus détaillée. Normalement on introduit ici le diagnostic médical. (Ce diagnostic n'apparaîtera pas sur la facture).

\item[procedere]
On peut décrir ici la procédure à suivre par ex. \textit{physiothérapie}.
En plus on peut introduire la médication fixe. Ceci doit être fait dans la 'view' \textit{problème}.

\item[Rn-Dx]
Ici on peut introduire les diagnostics qui figureront sur la facture. Doubel-cliquez dans ce champ pour introduire des diagnostics. (Les diagnostics introduits ne pourront plus être effacés. Choisissez pour ceci la 'view'  \textit{problème}.)

Si un problème avait été attribué à une consultation, ces diagnostics apparaîteront sur la facture pour cette consultation.

\item[status]
Un problème peut être dans un état actif (bouton vert) ou inactif (bouton rouge). On peut activer ou désactiver un problème en cliquant sur ce point.

\end{description}

La consultation actuellement choisi figure en dessous de la liste des problèmes. Celle-ci sera en gros utilisée de la même façon que la 'view' \textit{consultation}.

Du côté gauche du champ de texte on peut choisir les problèmes qui doivent être attribués à cette consultation. On y introduit normalement tous les problèmes qu avaient été évoqués avec le patient pendant la consultation .

Du côté droit du champ de texte on peut introduire les prestations.

Dans la partie inférieure se trouve une liste des consultations anciennes.

La 'view' \textit{problème} permet d'adapter plusieurs propriétés d'un problème. On peut y ajouter ou enlever la médication fixe ou les diagnostics de la facture. En plus on peut voir toutes les consultations à lesquelles ce problème avait été attribuée. Selon besoin on peut aussi annuler cette attribution.
(De  séléctionner dans la 'view' \textit{KG Iatrix} la consultation et d'effacer le problème de la consultation a le même effet.)

