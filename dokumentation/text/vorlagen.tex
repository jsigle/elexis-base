% *******************************************************************************
% * Copyright (c) 2007 by Elexis
% * All rights reserved. This document and the accompanying materials
% * are made available under the terms of the Eclipse Public License v1.0
% * which accompanies this distribution, and is available at
% * http://www.eclipse.org/legal/epl-v10.html
% *
% * Contributors:
% *    G. Weirich - initial implementation
% *
% *  $Id: vorlagen.tex 2821 2007-07-16 14:51:35Z rgw_ch $
%  *******************************************************************************
% !Mode:: "TeX:UTF-8" (encoding info for WinEdt)


 \section{Vorlagen}
 \label{textvorlagen}
 \index{Textformatvorlagen}\index{Briefvorlagen}
In Elexis erstellte Dokumente basieren immer auf bestimmten Vorlagen. Eine
Vorlage enthält einerseits das Aussehen des Dokuments (Briefkopf etc.),
andererseits bestimmte Platzhalter, in die beim Erstellen des Dokuments dann die
entsprechenden Daten eingefügt werden.

Eine Vorlage ist einfach ein mit OpenOffice erstelltes Dokument mit dem
gewünschten Aussehen. Platzhalter werden als gewöhnlicher, in eckige Klammern
gesetzter Text der Form [Datentyp.Feld] eingetragen, wie z.B.
[Patient.Vorname]. Eine Auflistung der möglichen Platzhalter finden Sie auf
Seite \pageref{Platzhalter}

Es gibt zwei Typen von Vorlagen:
\begin{itemize}
  \item {Systemvorlagen} sind Vorlagen, die für bestimmte Programmfunktionen
  benötigt werden. So kann beispielsweise ein Rezept nur auf der Basis einer
  Systemvorlage namens \glqq Rezept\grqq{} gedruckt werden. Systemvorlagen
  müssen einen bestimmten Namen haben (eben z.B. \glqq Rezept\grqq{}), und
  meistens an einer Stelle einen speziellen Platzhalter, der angibt, wo der
  Inhalt eingefügt werden soll.
  \item {Benutzervorlagen} sind Vorlagen, die beliebig erstellt und benannt
  werden können, und die beliebige Felder (oder gar keine) enthalten können.
  Benutzervorlagen können beispielsweise für Konsiliarberichte, Zuweisungen
  etc. erstellt werden.
\end{itemize}


\subsection{Systemvorlagen}
\label{systemvorlagen}
Folgende Systemvorlagen werden im Basis-System genutzt (Plugins können ggf. auch
eigene Systemvorlagen definieren):
\begin{itemize}

  \item {Rezept} Ein Rezept, meist auf A5 oder A6 gedruckt. Die Gestaltung
  erfolgt nach eigene Geschmack bzw. gemäss gesetzlichen Vorgaben, falls
  vorhanden. An der Stelle, wo später die Medikamente eingefügt werden soll,
  muss ein Platzhalter [Rezeptzeilen] stehen.
  \item {AUF-Zeugnis} Ein Arbeitsunfähigkeitszeugnis, ebenfalls gemäss örtlichem
  Usus frei gestaltbar. Die Eckdaten können mit den Platzhaltern [AUF.von],
  [AUF.bis] und [AUF.Prozent] eingetragen werden.
  \item {Laborblatt} Dies dient zum Ausdrucken der im System vorhandenen
  Laborwerte. Das Blatt ist frei gestaltbar, an einer Stelle muss der
  Platzhalter [Laborwerte] stehen.
  \item {Einnahmeliste} Eine Medikamentenliste für den Patienten. Frei
  gestaltbar, an einer Stelle muss der Platzhalter [Medikamentenliste] stehen.
  \item {Bestellung} Eine Bestellung zur Übermittlung per Brief, Fax oder Mail.
  Frei gestaltbar, an einer Stelle muss der Platzhalter [Bestellung] stehen.
  \item {AgendaListe} Ein Ausdruck der Agenda eines Bereiches für einen Tage.
  Frei gestaltbar, an einer Stelle muss der Platzhalter [Termine] stehen
  \item {Abrechnungsliste} Eine Liste aller in einem bestimmten Zeitraum
  erfolgten Abrechnungen (s. \pageref{fig:konnd}). Frei gestaltbar, an einer
  Stelle muss der Platzhalter [Liste] stehen.
  \item {Terminkarte} Eine Liste der Termine eines Patienten. Frei gestaltbar,
  an einer Stelle muss der Platzhalter [Termine] stehen.
  \item {Tarmedrechnung\_EZ} (Alle Tarmedrechnung\_xx Vorlagen sind vom
  Arzttarife-Schweiz-Plugin beigesteuert) Rechnung mit Einzahlungsschein, meist
  der beim Patienten verbleibende Teil. Muss auf einem A4-Blatt sein. Die oberen
  zwei Drittel sind frei gestaltbar, an einer Stelle muss der Platzhalter
  [Leistungen] stehen- Das untere Drittel muss frei bleiben, dort wird der
  Einzahlungsschein gedruckt.
  \item {Tarmedrechnung\_M1} Erste Mahnung mit Einzahlungsschein. Gestaltung s. Tarmedrechnung\_EZ.
  \item {Tarmedrechnung\_M2}Zweite Mahnung.
  \item {Tarmedrechnung\_M3}Dritte Mahnung.
  \item {Tarmedrechnung\_S1}Erste Seite des Tarmed-Formulars. Die Gestaltung ist
  fix vorgegeben, nur die persönlichen Daten können geändert werden (Dabei darf
  das Layout nicht verschoben werden).
  \item {Tarmedrechnung\_S2} Folgeseite des Tarmed-Formulars. Die Gestaltung ist
  ebenfalls fix vorgegeben.
\end{itemize}


\subsection{Benutzervorlagen}
Benutzervorlagen können beliebig erstellt und benannt werden.
