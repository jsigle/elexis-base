% *******************************************************************************
% * Copyright (c) 2007 by Elexis
% * All rights reserved. This document and the accompanying materials
% * are made available under the terms of the Eclipse Public License v1.0
% * which accompanies this distribution, and is available at
% * http://www.eclipse.org/legal/epl-v10.html
% *
% *  $Id$
% *******************************************************************************
% !Mode:: "TeX:UTF-8" (encoding info for WinEdt)

\section{Externe Dokumente}\label{elexis-externedokumente}
\index{Externe Dokumente}
Dieses Plugin ermöglicht die Integration von beliebigen extern gespeicherten
Dokumenten in Elexis.

Insgesamt können bis zu drei Verzeichnisse im Dateisystem eingebunden werden.
Die Dateien aus diesen Verzeichnissen werden in einer Liste kombiniert dargestellt.
Durch einen Doppelklick auf eine Datei wird das Dokument geöffnet. Ein so geöffnetes
Dokument kann normal bearbeitet werden.

In einer Netzwerkumgebung müssen die Dokumente auf einem Netzwerk-Laufwerk gespeichert
sein, damit die Dokumente von allen Arbeitsstationen aus verfügbar sind.

\subsection{Konfiguration}

Um das Plugin zu verwenden, muss es zuerst sinnvoll Konfiguriert werden.
In den Einstellungen gibt es einen Abschnitt \textit{Externe Dokumente}.
Dort können bis zu drei Verzeichnisse im Dateisystem ausgewählt werden.
(Werden diese Einstellungen geändert, muss Elexis neu gestartet werden.)

\subsection{Dokumente in Elexis anzeigen}

Wählen Sie im Menü \textit{Fenster/Ansicht/Other...}. Geben Sie im Suchfeld
\textit{Externe Dokumente} ein. Wählen Sie die Ansicht \textit{Externe Dokumente} aus
und klicken Sie auf OK. Nun können Sie das Fenster an der gewünschten Position
anordnen.

Ist ein Patient ausgewählt, sucht Elexis nach passenden Dokumenten in den
konfigurierten Verzeichnissen und zeigt sie in der Liste an. Elexis blendet
die Dokumente von allen Verzeichnissen ein. Sie können bei Bedarf ein Verzeichnis
ein- oder ausblenden, indem Sie oberhalb der Liste das Verzeichnis entsprechend
markieren.

Um ein Dokument zu öffnen, doppelklicken Sie auf den Dateinamen. Sie können
auch den Dateinamen oder das Datum des Dokuments öffnen, indem Sie es mit
der rechten Maustaste anklicken und im erscheinenden Menü \textit{Eigenschaften}
wählen.

\subsection{Namensgebung für Dokumente}

Extern gespeicherte Dokumente können nicht zuverlässig automatisch mit
den Patientendaten in Elexis verknüpft werden, da die extern gespeicherten
Dateien keine eindeutige Kennung haben. Die Dokumente müssen anhand des
Dateinamens mit den Patienten in Elexis verknüpft werden. Die Dateiname der
Dokumente müssen deshalb einem bestimmten Muster entsprechen, damit Elexis
die Dokumente einem Patienten zuordnen kann.

Das Plugin unterstützt folgendes Schema:

\begin{itemize}
\item Der Dateiname muss mit den ersten 6 Buchstaben des Nachnamens des Patienten beginnen.
      Falls der Nachname kürzer als 6 Zeichen ist, müssen entsprechend viele
      Leerzeichen angefügt werden, bis 6 Zeichen erreicht werden.
\item Danach folgt der komplette Vorname.
\item Der Rest kann aus beliebigem Text bestehen. Idealerweise wird der Inhalt
      des Dokuments beschrieben.
\end{itemize}

Beispiele:

\begin{description}
\item[MusterPeter Labor.pdf]
Diese Datei wird dem Patienten Muster Peter zugeordnet.
\item[Kunz  Rolf Überweisung Dr. Meier.doc]
Diese Datei wird dem Patienten Kunz Rolf zugeordnet. Der Nachname (Kunz) muss
mit 2 Leerzeichen ergänzt werden.
\end{description}
