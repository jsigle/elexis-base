% *******************************************************************************
% * Copyright (c) 2007 by Elexis
% * All rights reserved. This document and the accompanying materials
% * are made available under the terms of the Eclipse Public License v1.0
% * which accompanies this distribution, and is available at
% * http://www.eclipse.org/legal/epl-v10.html
% *
% *  $Id$
% *******************************************************************************
% !Mode:: "TeX:UTF-8" (encoding info for WinEdt)

\section{Iatrix}\label{Iatrix}
\index{Iatrix}
Alternativer KG-Stil mit problemorientiertem Ansatz. Dieses Plugin ist Teil von Iatrix.

\subsection{Allgemeine Informationen}

Die Iatrix-Entwicklung wird von Dr. med. P. Schönbucher, Luzern finanziert.
Hauptentwickler ist \href{http://www.elexis.ch/jp/component/option,com_contact/task,view/contact_id,2/Itemid,32/}{Daniel Lutz}.
Anfragen zu diesem Plugin deshalb bitte an ihn.

Weitere Informationen zu Iatrix finden Sie unter \href{http://www.iatrix.org/}.

\subsection{Verwendung}

\subsubsection{View öffnen}

Wählen Sie im Menü \textit{Fenster/Ansicht/Other...}. Geben Sie im Suchfeld
\textit{KG Iatrix} ein. Wählen Sie die Ansicht \textit{KG Iatrix} aus
und klicken Sie auf OK. Nun können Sie das Fenster an der gewünschten Position
anordnen.

Öffnen Sie auch die Ansicht \textit{Problem}, um die Eigenschaften eines
Problems im Detail anzuzeigen. Gehen Sie gleich vor wie oben, um dieses Fenster
zu öffnen.

\subsubsection{Iatrix verwenden}

Die Ansicht \textit{KG Iatrix} zeigt im oberen Bereich die Problemliste an.
Um ein Problem zu erstellen, können Sie einfach die gewünschten Informationen
in der letzten, leeren Zeile eingeben.

\begin{description}

\item[Datum]
Das Datum gibt an, wann das Problem das erste Mal aufgetreten ist. Es kann ein
komplettes Datum verwendet werden (z. B. 19.08.2003) oder nur eine Jahreszahl
(z. B. 2003).

\item[Nr.]
Die Probleme können nummeriert werden. Diese Nummerr
dienen zur eigenen Orientierung. Probleme können Teil eines anderen Problems
sein. Dann kann z. B. eine Nummerierung wie \textit{1}, \textit{1.1}, \textit{1.2}, etc.
verwendet werden.

\item[Problem/Diagnose]
Hier kann das Problem genauer beschrieben werden. Normalwerweise wird hier
die medizinische Diagnose angegeben. (Diese Diagnose erscheint nicht auf der
Rechnung).

\item[Procedere]
Hier kann eine Beschreibung des Procedere eingegeben werden, z. B. \textit{Physiotherapie}.
Zusäztlich können Fixmedikamente angegeben werden. Dies muss in der Ansicht \textit{Problem}
erfolgen.

\item[Rn-Dx]
Hier können die Diagnosen für die Rechnung erfasst werden. Doppelklicken Sie
auf dieses Feld, um Diagnosen hinzuzufügen. (Diagnosen können hier nicht wieder
entfernt werden. Wählen Sie hierfür die Ansicht \textit{Problem}.)

Wir ein Problem einer Konsultation zugeordnet, erscheinen diese Diagnosen
auf der Rechnung zu dieser Konsultation.

\item[Status]
Ein Problem kann aktiv oder inaktiv sein. Dies wird durch einen grünen oder
roten Punkt angezeigt. Sie können ein Problem aktivieren oder inaktivieren,
indem Sie auf diesen Punkt doppelklicken.

\end{description}

Unterhalb der Problemliste wird die aktuell ausgewählte Konsultation angezeigt.
Diese wird im Wesentlichen gleich bedient wie die Ansicht \textit{Konsultation}.

Links des Textfeldes können die Probleme ausgewählt werden, die dieser Konsultation
zugeordnet werden sollen. Normalerweise wählen Sie alle Probleme, die Sie
während der Konsultation mit dem Patienten besprochen haben.

Rechts des Textfeldes können Sie die Leistungen Verrechnen.

Im unteren Bereich wird die Liste der vergangenen Konsultationen angezeigt.

In der Ansicht \textit{Problem} Können Sie weitere Eigenschaften eines Problems
anpassen. Sie können hier Fixmedikamente und Rechnungsdiagnosen hinzufügen
oder entfernen. Des weiteren sehen Sie alle Konsultationen, denen dieses
Problem zugeordnet ist. Bei bedarf könne Sie hier diese Zuordnung aufheben.
(Dies hat denselben Effekt, wie wenn Sie in der Ansicht \textit{KG Iatrix}
die Konsultation auswählen und das Problem von der Konsultation entfernen.)

