% !Mode:: "TeX:UTF-8" (encoding info for WinEdt)
\section{MacOS X (Tiger) }
Elexis selbst lässt sich ziemlich einfach installieren: Einfach das Verzeichnis herunterladen und an geeigneter Stelle entpacken. Schwieriger ist die Installation von OpenOffice, und, das muss leider gesagt werden, noch nicht korrekt möglich.

Es gibt zwei Schwierigkeiten:
\begin{itemize}
 \item

Die \textit{erste}: OpenOffice benötigt auf dem Mac X11. Und X11 wird bei Tiger zwar mitgeliefert, aber ziemlich versteckt auf der System-DVD 1. Legen Sie also die System-DVD 1 ein. Ignorieren Sie die Aufforderung, irgendetwas zu installieren (denn X11 wird da nicht angeboten), sondern benutzen Sie \textit{spotlight}, um X11User.pgk zu finden. Starten Sie dann diesen Installer (braucht Administratorrechte). Der Rest geht weitgehend automatisch.

Sodann können Sie von \href{http://www.openoffice.org}{OpenOffice.org} (deutsch: http://de.openoffice.org/downloads/quick.html) die geeignete Datei für Ihr System herunterladen. Es wird eine Datei mit der Endung .dmg. Ein Doppelklick auf das Archiv entpackt es und mountet die Image-Datei, die das fertige OpenOffice.org enthält. Es erscheint eine Datei mit dem OpenOffice.org 2.0 - Symbol. Ziehen Sie die Datei in den Ordner \textit{Programme} (auf der linken Seite). Damit ist die OpenOffice-Installation beendet.

Damit ist OpenOffice auf dem Mac als Standalone-Programm lauffähig.

\item Die \textit{zweite Schwierigkeit} ist aber: OpenOffice kann noch nicht in Elexis eingebettet werden.
\item  Dies geschieht mit einer Komponente namens SWT\_AWT-Bridge, welche unter
Windows und Linux gut funktioniert, unter MacOS aber leider noch nicht. Man hofft aber, das mit einer der nächsten Releases
von MacOS bzw. Java für MacOS zu beheben. Mehr dazu steht \href{https://bugs.eclipse.org/bugs/show_bug.cgi?id=67384}{hier}
\end{itemize}

Die bottomline ist: Elexis läuft auf Macintosh nur begrenzt, alle Textfunktionen (Briefe, Rezepte, Zeugnisse, Rechnungen) sind deaktiviert.
